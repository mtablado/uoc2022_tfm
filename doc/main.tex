\documentclass[11pt,a4paper,openany]{book}

\usepackage{amsmath}
\usepackage{graphicx}
\usepackage{hyperref}
\usepackage[utf8]{inputenc}
\usepackage[acronym, toc]{glossaries}

\usepackage{setspace}	%double spacing for text, single for captions, footnotes, etc.
\usepackage{natbib}		% substituye a 'hypernat' que funciona en Windows.
\usepackage[english]{babel}
\usepackage[utf8]{inputenc}
\usepackage{color}
\usepackage{hhline} 		% extended styles for tables
\usepackage{multirow}
\usepackage{subfigure}
\usepackage{amsmath,amsmath,amssymb} 
\usepackage{fancyhdr}
\usepackage{epsfig, amsmath}
\usepackage{algorithm}
\usepackage{algorithmic}

\usepackage[section]{placeins}

% general settings
\hypersetup{
	linktocpage=true,
	colorlinks=true,
	linkcolor=blue,
	citecolor=blue,
}
\definecolor{Hgray}{gray}{0.6}

\newenvironment{definition}[1][Definition]{\begin{trivlist}
\item[\hskip \labelsep {\bfseries #1}]}{\end{trivlist}}

\setlength{\topmargin}{0cm}
\setlength{\textheight}{23cm}
\setlength{\textwidth}{17cm}
\setlength{\oddsidemargin}{0cm}
\setlength{\evensidemargin}{0cm}
\setlength{\headheight}{1cm}

% indica que las 'sub-sub-sections' sean numeradas y aparezcan en el indice
\setcounter{secnumdepth}{3}
\setcounter{tocdepth}{2}

% settings for code
\renewcommand{\algorithmicrequire}{\textbf{Entrada: }}
\renewcommand{\algorithmicensure}{\textbf{Salida: }}

\makeglossaries
\newglossaryentry{kpi}
{
    name=KPI,
    description={Key Performance Indicator}
}

\newglossaryentry{mnist}
{
    name=MNIST,
    description={The MNIST database of handwritten digits used for training models}
}

\newglossaryentry{kdd}
{
    name=KDD,
    description={KDD cup 1999 network intrusion dataset}
}

\newglossaryentry{ixi}
{
    name=IXI,
    description={IXI dataset is a collection of 600 Brain MR Images, in 3-D, collected from 3 different hospitals in London. The images are scanned in 1.5 and 3 Tesla which is the range of the quality that we could usually find in hospitals around the world}
}

\newacronym{pca}{PCA}{Principal Component Analysis}

\newacronym{mri}{MRI}{Magnetic Resonance Image}

\newacronym{vqvae}{VQ-VAE}{Vector-Quantized Variational Autoencoder}

\newacronym{cevae}{ceVAE}{Context-Encoder Variational Autoencoder}

\newacronym{strega}{StRegA}{Segmentation Regularised Anomaly}


\begin{document}

% Portada
\newpage
\thispagestyle{empty}

\baselineskip 2em

%\vspace*{1cm}

\centerline{\includegraphics[width=0.6\textwidth]{images/UOC-logo}}
\begin{center}
\textsc{Universitat Oberta de Catalunya (UOC) \\
 Máster Universitario en Ciencia de Datos (\textit{Data Science})\\}

%\centerline {\pic{UOC}{4cm}}

\vspace*{1.5cm}

\textsc{\Large TRABAJO FINAL DE MÁSTER}

\vspace*{0.5cm}

\textsc{\large Área: YYY}


%\textbf{\Huge VirtualTechLab Model: }

\vspace*{2.0cm}

\title{\Large Artificial MRI brain images creation with Variational Autoencoders}

\vspace{2.5cm}
\baselineskip 1em

\baselineskip 2em
-----------------------------------------------------------------------------\\
Autor:      Miguel Tablado\\
Tutor:      Baris Kanber\\
Profesor:   Nombre del profesor responsable del área de TF\\
-----------------------------------------------------------------------------\\
\vspace*{1.5cm}
Barcelona, \today

\author{Miguel Tablado}

\end{center}



%\tableofcontents

\pagenumbering{roman} 
\setcounter{page}{1} 
\pagestyle{plain}

%%%%%%%%%%%%%
%%% FICHA %%%
%%%%%%%%%%%%%
\chapter*{FICHA DEL TRABAJO FINAL}
\begin{table}[ht]
	\centering{}
	\renewcommand{\arraystretch}{2}
	\begin{tabular}{r | l}
		\hline
		Título del trabajo: & Artificial MRI brain images creation with Variational Autoencoders\\
		\hline
        Nombre del autor: & Miguel Tablado León\\
		\hline
        Nombre del Tutor/a de TF: & Baris Kanber\\
		\hline
        Nombre del/de la PRA: & Ferran Prados Carrasco\\
		\hline
        Fecha de entrega: & 02/2023\\
		\hline
        Titulación o programa: & Máster en Ciencia de Datos\\
		\hline
        Área del Trabajo Final: & Medicine Area (TFM-Med)\\
		\hline
        Language: & English\\
		\hline
        Keywords & Deep Learning, Brain MRI, Variational Autoencoder\\
		\hline
	\end{tabular}
\end{table}

\pagestyle{fancy}
\renewcommand{\chaptermark}[1]{ \markboth{#1}{}}
\renewcommand{\sectionmark}[1]{\markright{ \thesection.\ #1}}
\lhead[\fancyplain{}{\bfseries\thepage}]{\fancyplain{}{\bfseries\rightmark}}
\rhead[\fancyplain{}{\bfseries\leftmark}]{\fancyplain{}{\bfseries\thepage}}
\cfoot{}

% indice
\cleardoublepage
\phantomsection
\addcontentsline{toc}{chapter}{Index}
\tableofcontents
% listado de figuras
\cleardoublepage
\phantomsection
\addcontentsline{toc}{chapter}{Figure List}
\listoffigures
% listado de tablas
\cleardoublepage
\phantomsection
\addcontentsline{toc}{chapter}{Table List}
\listoftables

\thispagestyle{empty}

\pagenumbering{arabic}

\pagestyle{fancy}
\renewcommand{\chaptermark}[1]{ \markboth{#1}{}}
\renewcommand{\sectionmark}[1]{\markright{ \thesection.\ #1}}
\lhead[\fancyplain{}{\bfseries\thepage}]{\fancyplain{}{\bfseries\rightmark}}
\rhead[\fancyplain{}{\bfseries\leftmark}]{\fancyplain{}{\bfseries\thepage}}
\cfoot{}

\onehalfspacing

\input{introduction/0_abstract.tex}

\chapter{Introduction}

\section{Context and project justification}

Artificial Intelligence has arrived to change the world in almost (if not all) any field. Today, we are surrounded by (and we are using many) AI products like smartphones’ face recognition capabilities, home cleaning robots or cars with autopilot options.

From the different AI fields, computer vision is maybe the most popular one and the one which is usually used for explaining AI capabilities to general public. Identifying a cat in a picture could perfectly be the example used in every AI presentation to welcome people to AI.

Image processing is intuitively matched with medical diagnosis by anyone having or not any expertise on the field. Every single citizen will have heard of magnetic resonance imaging, and anyone easily transposes ‘cat detection’ to ‘anomaly detection’, being the anomaly a tumor or anything else.

There are different techniques to scan people and create images for clinical diagnosis like X-Ray or Magnetic Resonance Imaging. In this project, we will work with Magnetic Resonance Images (MRI) which are images created by a machine with a large bore that scans people lying inside it. The MR technique is non-invasive, it produces no radiation, and is used to scan almost any part of the body from which we will focus on brain images.

Combining AI diagnosis capabilities on image processing and MRI images, we can think of helping doctors to identify the presence of anomalies or looking for concrete diagnosis for a specific disease.

Obviously, these projects are not easy at all and they face a lot of challenges. One of the first challenges that such AI project faces is the difficulty to obtain a large set of brain images that are needed to train an accurate AI model. Scanning people is too costly and requires a lot of time. and the challenge of having a large dataset gets harder once we understand that brain images may differ depending on age or gender.

Today, there is a clear limitation on how to reproduce or obtain healthy brain images for AI-based diagnosis projects while the appearance of new AI techniques known as Autoencoders and Variational Autoencoders introduces a new area of investigation to mitigate the gap.

This project aims to be a proof of concept to create AI-created images with new variational autoencoders that would serve to augment any existing MRI dataset and that will help to improve the accuracy of brain anomaly detection projects, including those which could be used for overall anomaly detection to others more specific which would help on concrete disease diagnosis.

Personal motivation comes from various angles:

\begin{itemize}
    \item One is to prove myself that I can work with new AI architectures demonstrating that I have acquired the knowledge needed (deep enough) to be productive and to be able to innovate in the health sector.
    \item Deep Learning has been the subject which I enjoyed the most, hence continuing with Autoencoders seems natural to me as the next step on AI adoption
    \item At no doubts, if I can contribute to help on brain issue detection or diagnosis, I will feel my life been completely fulfilled
\end{itemize}
\section{Aim of the project}

The aim of this project is to serve as a Proof of Concept on how MRI brain images can be artificially generated with Variational Autoencoders which ultimately would serve to enhance existing or new datasets to improve model accuracy by having larger samples of data.

Foreseen projects objectives are:

\begin{itemize}
    \item Obtain basic knowledge about MRI images and the NIFTI file format
    \item Obtain and visualise 2-D images from 3-D images in the dataset
    \item Select what range of slices (2-D images) from brain to be created
    \item Test different existing networks and choose the one to be used
    \item Tune network parameters
    \item Compare generated brain images against real ones, qualitatively
\end{itemize}
\section{Project Plan}

\subsection{Resources}

\begin{enumerate}
    \item 1 Data Scientist: Miguel Tablado will be playing this role and will dedicate 300h
    \item 1 Coach/Tutor: Baris Kanber will be assisting Miguel Tablado during the project
    \item MRI images and demographic information from IXI Dataset
    \item GPU resources are needed to train and test the network
\end{enumerate}

\subsection{High Level Plan}

The plan will be executed in 3 different phases with the listed tasks:
\begin{enumerate}
    \item Phase 1: Analysis
    \begin{enumerate}
        \item Gain knowledge on MRI and NIFTI protocol
        \item Describe images and dataset
        \item Extract 2D images
        \item Pre-processing images
    \end{enumerate}
    \item Phase 2: MR Images creation
    \begin{enumerate}
        \item Test different Network architectures
        \item Tune-up architectural network
    \end{enumerate}
    \item Phase 3: Project documentation
    \begin{enumerate}
        \item Write conclusions
        \item Create project documentation
        \item Create project presentation
    \end{enumerate}
\end{enumerate}

\begin{figure}[ht]
    \hspace*{-1.2in}
    \centering
    \includegraphics[width = 20cm, height = 6cm]{images/project-plan.png}
    \caption[]{Project Plan. source: https://www.onlinegantt.com}
    \label{fig:project-plan}
\end{figure}

\newpage
\subsection{Tasks}
\subsubsection*{Phase 1: Analysis}

During this face the different tasks will be executed to prepare the work and includes:
\begin{enumerate}
    \item Gain knowledge on MRI and the NIFTI file format: This activity consists of reading papers and documents to gain sufficient knowledge to execute the project. There is no need to become an expert on the matter but understanding how those files are and how to process them.
    \item Describe images and dataset: During this activity, a description of the dataset will be generated with a view of the quality of the dataset for the aim of the project and any findings which could result.
    \item Extract 2D images: Load 3D images and extract 2D slices from the original dataset, which will be depicted with code.
    \item Pre-processing images: Decide which transformations on the 2D images would help the project like pixel changes or applying gray-scale transformations.
\end{enumerate}

\subsubsection*{Phase 2: MR Images creation}

\begin{enumerate}
    \item Test different Network architectures: This task will take few existing network architectures and be tested with the dataset so that one of them will be selected to be improved and used as the project architecture.
    \item Tune-up architectural network: Tune the selected architecture with useful techniques like changing network layers
\end{enumerate}

\documentclass[11pt,a4paper,openany]{book}

\usepackage{amsmath}
\usepackage{graphicx}
\usepackage{hyperref}
\usepackage[utf8]{inputenc}
\usepackage[acronym, toc]{glossaries}

\usepackage{setspace}	%double spacing for text, single for captions, footnotes, etc.
\usepackage{natbib}		% substituye a 'hypernat' que funciona en Windows.
\usepackage[english]{babel}
\usepackage[utf8]{inputenc}
\usepackage{color}
\usepackage{hhline} 		% extended styles for tables
\usepackage{multirow}
\usepackage{subfigure}
\usepackage{amsmath,amsmath,amssymb} 
\usepackage{fancyhdr}
\usepackage{epsfig, amsmath}
\usepackage{algorithm}
\usepackage{algorithmic}

\usepackage[section]{placeins}

% general settings
\hypersetup{
	linktocpage=true,
	colorlinks=true,
	linkcolor=blue,
	citecolor=blue,
}
\definecolor{Hgray}{gray}{0.6}

\newenvironment{definition}[1][Definition]{\begin{trivlist}
\item[\hskip \labelsep {\bfseries #1}]}{\end{trivlist}}

\setlength{\topmargin}{0cm}
\setlength{\textheight}{23cm}
\setlength{\textwidth}{17cm}
\setlength{\oddsidemargin}{0cm}
\setlength{\evensidemargin}{0cm}
\setlength{\headheight}{1cm}

% indica que las 'sub-sub-sections' sean numeradas y aparezcan en el indice
\setcounter{secnumdepth}{3}
\setcounter{tocdepth}{2}

% settings for code
\renewcommand{\algorithmicrequire}{\textbf{Entrada: }}
\renewcommand{\algorithmicensure}{\textbf{Salida: }}

\makeglossaries
\newglossaryentry{kpi}
{
    name=KPI,
    description={Key Performance Indicator}
}

\newglossaryentry{mnist}
{
    name=MNIST,
    description={The MNIST database of handwritten digits used for training models}
}

\newglossaryentry{kdd}
{
    name=KDD,
    description={KDD cup 1999 network intrusion dataset}
}

\newglossaryentry{ixi}
{
    name=IXI,
    description={IXI dataset is a collection of 600 Brain MR Images, in 3-D, collected from 3 different hospitals in London. The images are scanned in 1.5 and 3 Tesla which is the range of the quality that we could usually find in hospitals around the world}
}

\newacronym{pca}{PCA}{Principal Component Analysis}

\newacronym{mri}{MRI}{Magnetic Resonance Image}

\newacronym{vqvae}{VQ-VAE}{Vector-Quantized Variational Autoencoder}

\newacronym{cevae}{ceVAE}{Context-Encoder Variational Autoencoder}

\newacronym{strega}{StRegA}{Segmentation Regularised Anomaly}


\begin{document}

% Portada
\newpage
\thispagestyle{empty}

\baselineskip 2em

%\vspace*{1cm}

\centerline{\includegraphics[width=0.6\textwidth]{images/UOC-logo}}
\begin{center}
\textsc{Universitat Oberta de Catalunya (UOC) \\
 Máster Universitario en Ciencia de Datos (\textit{Data Science})\\}

%\centerline {\pic{UOC}{4cm}}

\vspace*{1.5cm}

\textsc{\Large TRABAJO FINAL DE MÁSTER}

\vspace*{0.5cm}

\textsc{\large Área: YYY}


%\textbf{\Huge VirtualTechLab Model: }

\vspace*{2.0cm}

\title{\Large Artificial MRI brain images creation with Variational Autoencoders}

\vspace{2.5cm}
\baselineskip 1em

\baselineskip 2em
-----------------------------------------------------------------------------\\
Autor:      Miguel Tablado\\
Tutor:      Baris Kanber\\
Profesor:   Nombre del profesor responsable del área de TF\\
-----------------------------------------------------------------------------\\
\vspace*{1.5cm}
Barcelona, \today

\author{Miguel Tablado}

\end{center}



%\tableofcontents

\pagenumbering{roman} 
\setcounter{page}{1} 
\pagestyle{plain}

%%%%%%%%%%%%%
%%% FICHA %%%
%%%%%%%%%%%%%
\chapter*{FICHA DEL TRABAJO FINAL}
\begin{table}[ht]
	\centering{}
	\renewcommand{\arraystretch}{2}
	\begin{tabular}{r | l}
		\hline
		Título del trabajo: & Artificial MRI brain images creation with Variational Autoencoders\\
		\hline
        Nombre del autor: & Miguel Tablado León\\
		\hline
        Nombre del Tutor/a de TF: & Baris Kanber\\
		\hline
        Nombre del/de la PRA: & Ferran Prados Carrasco\\
		\hline
        Fecha de entrega: & 02/2023\\
		\hline
        Titulación o programa: & Máster en Ciencia de Datos\\
		\hline
        Área del Trabajo Final: & Medicine Area (TFM-Med)\\
		\hline
        Language: & English\\
		\hline
        Keywords & Deep Learning, Brain MRI, Variational Autoencoder\\
		\hline
	\end{tabular}
\end{table}

\pagestyle{fancy}
\renewcommand{\chaptermark}[1]{ \markboth{#1}{}}
\renewcommand{\sectionmark}[1]{\markright{ \thesection.\ #1}}
\lhead[\fancyplain{}{\bfseries\thepage}]{\fancyplain{}{\bfseries\rightmark}}
\rhead[\fancyplain{}{\bfseries\leftmark}]{\fancyplain{}{\bfseries\thepage}}
\cfoot{}

% indice
\cleardoublepage
\phantomsection
\addcontentsline{toc}{chapter}{Index}
\tableofcontents
% listado de figuras
\cleardoublepage
\phantomsection
\addcontentsline{toc}{chapter}{Figure List}
\listoffigures
% listado de tablas
\cleardoublepage
\phantomsection
\addcontentsline{toc}{chapter}{Table List}
\listoftables

\thispagestyle{empty}

\pagenumbering{arabic}

\pagestyle{fancy}
\renewcommand{\chaptermark}[1]{ \markboth{#1}{}}
\renewcommand{\sectionmark}[1]{\markright{ \thesection.\ #1}}
\lhead[\fancyplain{}{\bfseries\thepage}]{\fancyplain{}{\bfseries\rightmark}}
\rhead[\fancyplain{}{\bfseries\leftmark}]{\fancyplain{}{\bfseries\thepage}}
\cfoot{}

\onehalfspacing

\input{introduction/0_abstract.tex}

\chapter{Introduction}

\section{Context and project justification}

Artificial Intelligence has arrived to change the world in almost (if not all) any field. Today, we are surrounded by (and we are using many) AI products like smartphones’ face recognition capabilities, home cleaning robots or cars with autopilot options.

From the different AI fields, computer vision is maybe the most popular one and the one which is usually used for explaining AI capabilities to general public. Identifying a cat in a picture could perfectly be the example used in every AI presentation to welcome people to AI.

Image processing is intuitively matched with medical diagnosis by anyone having or not any expertise on the field. Every single citizen will have heard of magnetic resonance imaging, and anyone easily transposes ‘cat detection’ to ‘anomaly detection’, being the anomaly a tumor or anything else.

There are different techniques to scan people and create images for clinical diagnosis like X-Ray or Magnetic Resonance Imaging. In this project, we will work with Magnetic Resonance Images (MRI) which are images created by a machine with a large bore that scans people lying inside it. The MR technique is non-invasive, it produces no radiation, and is used to scan almost any part of the body from which we will focus on brain images.

Combining AI diagnosis capabilities on image processing and MRI images, we can think of helping doctors to identify the presence of anomalies or looking for concrete diagnosis for a specific disease.

Obviously, these projects are not easy at all and they face a lot of challenges. One of the first challenges that such AI project faces is the difficulty to obtain a large set of brain images that are needed to train an accurate AI model. Scanning people is too costly and requires a lot of time. and the challenge of having a large dataset gets harder once we understand that brain images may differ depending on age or gender.

Today, there is a clear limitation on how to reproduce or obtain healthy brain images for AI-based diagnosis projects while the appearance of new AI techniques known as Autoencoders and Variational Autoencoders introduces a new area of investigation to mitigate the gap.

This project aims to be a proof of concept to create AI-created images with new variational autoencoders that would serve to augment any existing MRI dataset and that will help to improve the accuracy of brain anomaly detection projects, including those which could be used for overall anomaly detection to others more specific which would help on concrete disease diagnosis.

Personal motivation comes from various angles:

\begin{itemize}
    \item One is to prove myself that I can work with new AI architectures demonstrating that I have acquired the knowledge needed (deep enough) to be productive and to be able to innovate in the health sector.
    \item Deep Learning has been the subject which I enjoyed the most, hence continuing with Autoencoders seems natural to me as the next step on AI adoption
    \item At no doubts, if I can contribute to help on brain issue detection or diagnosis, I will feel my life been completely fulfilled
\end{itemize}
\section{Aim of the project}

The aim of this project is to serve as a Proof of Concept on how MRI brain images can be artificially generated with Variational Autoencoders which ultimately would serve to enhance existing or new datasets to improve model accuracy by having larger samples of data.

Foreseen projects objectives are:

\begin{itemize}
    \item Obtain basic knowledge about MRI images and the NIFTI file format
    \item Obtain and visualise 2-D images from 3-D images in the dataset
    \item Select what range of slices (2-D images) from brain to be created
    \item Test different existing networks and choose the one to be used
    \item Tune network parameters
    \item Compare generated brain images against real ones, qualitatively
\end{itemize}
\section{Project Plan}

\subsection{Resources}

\begin{enumerate}
    \item 1 Data Scientist: Miguel Tablado will be playing this role and will dedicate 300h
    \item 1 Coach/Tutor: Baris Kanber will be assisting Miguel Tablado during the project
    \item MRI images and demographic information from IXI Dataset
    \item GPU resources are needed to train and test the network
\end{enumerate}

\subsection{High Level Plan}

The plan will be executed in 3 different phases with the listed tasks:
\begin{enumerate}
    \item Phase 1: Analysis
    \begin{enumerate}
        \item Gain knowledge on MRI and NIFTI protocol
        \item Describe images and dataset
        \item Extract 2D images
        \item Pre-processing images
    \end{enumerate}
    \item Phase 2: MR Images creation
    \begin{enumerate}
        \item Test different Network architectures
        \item Tune-up architectural network
    \end{enumerate}
    \item Phase 3: Project documentation
    \begin{enumerate}
        \item Write conclusions
        \item Create project documentation
        \item Create project presentation
    \end{enumerate}
\end{enumerate}

\begin{figure}[ht]
    \hspace*{-1.2in}
    \centering
    \includegraphics[width = 20cm, height = 6cm]{images/project-plan.png}
    \caption[]{Project Plan. source: https://www.onlinegantt.com}
    \label{fig:project-plan}
\end{figure}

\newpage
\subsection{Tasks}
\subsubsection*{Phase 1: Analysis}

During this face the different tasks will be executed to prepare the work and includes:
\begin{enumerate}
    \item Gain knowledge on MRI and the NIFTI file format: This activity consists of reading papers and documents to gain sufficient knowledge to execute the project. There is no need to become an expert on the matter but understanding how those files are and how to process them.
    \item Describe images and dataset: During this activity, a description of the dataset will be generated with a view of the quality of the dataset for the aim of the project and any findings which could result.
    \item Extract 2D images: Load 3D images and extract 2D slices from the original dataset, which will be depicted with code.
    \item Pre-processing images: Decide which transformations on the 2D images would help the project like pixel changes or applying gray-scale transformations.
\end{enumerate}

\subsubsection*{Phase 2: MR Images creation}

\begin{enumerate}
    \item Test different Network architectures: This task will take few existing network architectures and be tested with the dataset so that one of them will be selected to be improved and used as the project architecture.
    \item Tune-up architectural network: Tune the selected architecture with useful techniques like changing network layers
\end{enumerate}

\documentclass[11pt,a4paper,openany]{book}

\usepackage{amsmath}
\usepackage{graphicx}
\usepackage{hyperref}
\usepackage[utf8]{inputenc}
\usepackage[acronym, toc]{glossaries}

\usepackage{setspace}	%double spacing for text, single for captions, footnotes, etc.
\usepackage{natbib}		% substituye a 'hypernat' que funciona en Windows.
\usepackage[english]{babel}
\usepackage[utf8]{inputenc}
\usepackage{color}
\usepackage{hhline} 		% extended styles for tables
\usepackage{multirow}
\usepackage{subfigure}
\usepackage{amsmath,amsmath,amssymb} 
\usepackage{fancyhdr}
\usepackage{epsfig, amsmath}
\usepackage{algorithm}
\usepackage{algorithmic}

\usepackage[section]{placeins}

% general settings
\hypersetup{
	linktocpage=true,
	colorlinks=true,
	linkcolor=blue,
	citecolor=blue,
}
\definecolor{Hgray}{gray}{0.6}

\newenvironment{definition}[1][Definition]{\begin{trivlist}
\item[\hskip \labelsep {\bfseries #1}]}{\end{trivlist}}

\setlength{\topmargin}{0cm}
\setlength{\textheight}{23cm}
\setlength{\textwidth}{17cm}
\setlength{\oddsidemargin}{0cm}
\setlength{\evensidemargin}{0cm}
\setlength{\headheight}{1cm}

% indica que las 'sub-sub-sections' sean numeradas y aparezcan en el indice
\setcounter{secnumdepth}{3}
\setcounter{tocdepth}{2}

% settings for code
\renewcommand{\algorithmicrequire}{\textbf{Entrada: }}
\renewcommand{\algorithmicensure}{\textbf{Salida: }}

\makeglossaries
\newglossaryentry{kpi}
{
    name=KPI,
    description={Key Performance Indicator}
}

\newglossaryentry{mnist}
{
    name=MNIST,
    description={The MNIST database of handwritten digits used for training models}
}

\newglossaryentry{kdd}
{
    name=KDD,
    description={KDD cup 1999 network intrusion dataset}
}

\newglossaryentry{ixi}
{
    name=IXI,
    description={IXI dataset is a collection of 600 Brain MR Images, in 3-D, collected from 3 different hospitals in London. The images are scanned in 1.5 and 3 Tesla which is the range of the quality that we could usually find in hospitals around the world}
}

\newacronym{pca}{PCA}{Principal Component Analysis}

\newacronym{mri}{MRI}{Magnetic Resonance Image}

\newacronym{vqvae}{VQ-VAE}{Vector-Quantized Variational Autoencoder}

\newacronym{cevae}{ceVAE}{Context-Encoder Variational Autoencoder}

\newacronym{strega}{StRegA}{Segmentation Regularised Anomaly}


\begin{document}

% Portada
\newpage
\thispagestyle{empty}

\baselineskip 2em

%\vspace*{1cm}

\centerline{\includegraphics[width=0.6\textwidth]{images/UOC-logo}}
\begin{center}
\textsc{Universitat Oberta de Catalunya (UOC) \\
 Máster Universitario en Ciencia de Datos (\textit{Data Science})\\}

%\centerline {\pic{UOC}{4cm}}

\vspace*{1.5cm}

\textsc{\Large TRABAJO FINAL DE MÁSTER}

\vspace*{0.5cm}

\textsc{\large Área: YYY}


%\textbf{\Huge VirtualTechLab Model: }

\vspace*{2.0cm}

\title{\Large Artificial MRI brain images creation with Variational Autoencoders}

\vspace{2.5cm}
\baselineskip 1em

\baselineskip 2em
-----------------------------------------------------------------------------\\
Autor:      Miguel Tablado\\
Tutor:      Baris Kanber\\
Profesor:   Nombre del profesor responsable del área de TF\\
-----------------------------------------------------------------------------\\
\vspace*{1.5cm}
Barcelona, \today

\author{Miguel Tablado}

\end{center}



%\tableofcontents

\pagenumbering{roman} 
\setcounter{page}{1} 
\pagestyle{plain}

%%%%%%%%%%%%%
%%% FICHA %%%
%%%%%%%%%%%%%
\chapter*{FICHA DEL TRABAJO FINAL}
\begin{table}[ht]
	\centering{}
	\renewcommand{\arraystretch}{2}
	\begin{tabular}{r | l}
		\hline
		Título del trabajo: & Artificial MRI brain images creation with Variational Autoencoders\\
		\hline
        Nombre del autor: & Miguel Tablado León\\
		\hline
        Nombre del Tutor/a de TF: & Baris Kanber\\
		\hline
        Nombre del/de la PRA: & Ferran Prados Carrasco\\
		\hline
        Fecha de entrega: & 02/2023\\
		\hline
        Titulación o programa: & Máster en Ciencia de Datos\\
		\hline
        Área del Trabajo Final: & Medicine Area (TFM-Med)\\
		\hline
        Language: & English\\
		\hline
        Keywords & Deep Learning, Brain MRI, Variational Autoencoder\\
		\hline
	\end{tabular}
\end{table}

\pagestyle{fancy}
\renewcommand{\chaptermark}[1]{ \markboth{#1}{}}
\renewcommand{\sectionmark}[1]{\markright{ \thesection.\ #1}}
\lhead[\fancyplain{}{\bfseries\thepage}]{\fancyplain{}{\bfseries\rightmark}}
\rhead[\fancyplain{}{\bfseries\leftmark}]{\fancyplain{}{\bfseries\thepage}}
\cfoot{}

% indice
\cleardoublepage
\phantomsection
\addcontentsline{toc}{chapter}{Index}
\tableofcontents
% listado de figuras
\cleardoublepage
\phantomsection
\addcontentsline{toc}{chapter}{Figure List}
\listoffigures
% listado de tablas
\cleardoublepage
\phantomsection
\addcontentsline{toc}{chapter}{Table List}
\listoftables

\thispagestyle{empty}

\pagenumbering{arabic}

\pagestyle{fancy}
\renewcommand{\chaptermark}[1]{ \markboth{#1}{}}
\renewcommand{\sectionmark}[1]{\markright{ \thesection.\ #1}}
\lhead[\fancyplain{}{\bfseries\thepage}]{\fancyplain{}{\bfseries\rightmark}}
\rhead[\fancyplain{}{\bfseries\leftmark}]{\fancyplain{}{\bfseries\thepage}}
\cfoot{}

\onehalfspacing

\input{introduction/0_abstract.tex}

\chapter{Introduction}

\section{Context and project justification}

Artificial Intelligence has arrived to change the world in almost (if not all) any field. Today, we are surrounded by (and we are using many) AI products like smartphones’ face recognition capabilities, home cleaning robots or cars with autopilot options.

From the different AI fields, computer vision is maybe the most popular one and the one which is usually used for explaining AI capabilities to general public. Identifying a cat in a picture could perfectly be the example used in every AI presentation to welcome people to AI.

Image processing is intuitively matched with medical diagnosis by anyone having or not any expertise on the field. Every single citizen will have heard of magnetic resonance imaging, and anyone easily transposes ‘cat detection’ to ‘anomaly detection’, being the anomaly a tumor or anything else.

There are different techniques to scan people and create images for clinical diagnosis like X-Ray or Magnetic Resonance Imaging. In this project, we will work with Magnetic Resonance Images (MRI) which are images created by a machine with a large bore that scans people lying inside it. The MR technique is non-invasive, it produces no radiation, and is used to scan almost any part of the body from which we will focus on brain images.

Combining AI diagnosis capabilities on image processing and MRI images, we can think of helping doctors to identify the presence of anomalies or looking for concrete diagnosis for a specific disease.

Obviously, these projects are not easy at all and they face a lot of challenges. One of the first challenges that such AI project faces is the difficulty to obtain a large set of brain images that are needed to train an accurate AI model. Scanning people is too costly and requires a lot of time. and the challenge of having a large dataset gets harder once we understand that brain images may differ depending on age or gender.

Today, there is a clear limitation on how to reproduce or obtain healthy brain images for AI-based diagnosis projects while the appearance of new AI techniques known as Autoencoders and Variational Autoencoders introduces a new area of investigation to mitigate the gap.

This project aims to be a proof of concept to create AI-created images with new variational autoencoders that would serve to augment any existing MRI dataset and that will help to improve the accuracy of brain anomaly detection projects, including those which could be used for overall anomaly detection to others more specific which would help on concrete disease diagnosis.

Personal motivation comes from various angles:

\begin{itemize}
    \item One is to prove myself that I can work with new AI architectures demonstrating that I have acquired the knowledge needed (deep enough) to be productive and to be able to innovate in the health sector.
    \item Deep Learning has been the subject which I enjoyed the most, hence continuing with Autoencoders seems natural to me as the next step on AI adoption
    \item At no doubts, if I can contribute to help on brain issue detection or diagnosis, I will feel my life been completely fulfilled
\end{itemize}
\section{Aim of the project}

The aim of this project is to serve as a Proof of Concept on how MRI brain images can be artificially generated with Variational Autoencoders which ultimately would serve to enhance existing or new datasets to improve model accuracy by having larger samples of data.

Foreseen projects objectives are:

\begin{itemize}
    \item Obtain basic knowledge about MRI images and the NIFTI file format
    \item Obtain and visualise 2-D images from 3-D images in the dataset
    \item Select what range of slices (2-D images) from brain to be created
    \item Test different existing networks and choose the one to be used
    \item Tune network parameters
    \item Compare generated brain images against real ones, qualitatively
\end{itemize}
\section{Project Plan}

\subsection{Resources}

\begin{enumerate}
    \item 1 Data Scientist: Miguel Tablado will be playing this role and will dedicate 300h
    \item 1 Coach/Tutor: Baris Kanber will be assisting Miguel Tablado during the project
    \item MRI images and demographic information from IXI Dataset
    \item GPU resources are needed to train and test the network
\end{enumerate}

\subsection{High Level Plan}

The plan will be executed in 3 different phases with the listed tasks:
\begin{enumerate}
    \item Phase 1: Analysis
    \begin{enumerate}
        \item Gain knowledge on MRI and NIFTI protocol
        \item Describe images and dataset
        \item Extract 2D images
        \item Pre-processing images
    \end{enumerate}
    \item Phase 2: MR Images creation
    \begin{enumerate}
        \item Test different Network architectures
        \item Tune-up architectural network
    \end{enumerate}
    \item Phase 3: Project documentation
    \begin{enumerate}
        \item Write conclusions
        \item Create project documentation
        \item Create project presentation
    \end{enumerate}
\end{enumerate}

\begin{figure}[ht]
    \hspace*{-1.2in}
    \centering
    \includegraphics[width = 20cm, height = 6cm]{images/project-plan.png}
    \caption[]{Project Plan. source: https://www.onlinegantt.com}
    \label{fig:project-plan}
\end{figure}

\newpage
\subsection{Tasks}
\subsubsection*{Phase 1: Analysis}

During this face the different tasks will be executed to prepare the work and includes:
\begin{enumerate}
    \item Gain knowledge on MRI and the NIFTI file format: This activity consists of reading papers and documents to gain sufficient knowledge to execute the project. There is no need to become an expert on the matter but understanding how those files are and how to process them.
    \item Describe images and dataset: During this activity, a description of the dataset will be generated with a view of the quality of the dataset for the aim of the project and any findings which could result.
    \item Extract 2D images: Load 3D images and extract 2D slices from the original dataset, which will be depicted with code.
    \item Pre-processing images: Decide which transformations on the 2D images would help the project like pixel changes or applying gray-scale transformations.
\end{enumerate}

\subsubsection*{Phase 2: MR Images creation}

\begin{enumerate}
    \item Test different Network architectures: This task will take few existing network architectures and be tested with the dataset so that one of them will be selected to be improved and used as the project architecture.
    \item Tune-up architectural network: Tune the selected architecture with useful techniques like changing network layers
\end{enumerate}

\documentclass[11pt,a4paper,openany]{book}

\usepackage{amsmath}
\usepackage{graphicx}
\usepackage{hyperref}
\usepackage[utf8]{inputenc}
\usepackage[acronym, toc]{glossaries}

\usepackage{setspace}	%double spacing for text, single for captions, footnotes, etc.
\usepackage{natbib}		% substituye a 'hypernat' que funciona en Windows.
\usepackage[english]{babel}
\usepackage[utf8]{inputenc}
\usepackage{color}
\usepackage{hhline} 		% extended styles for tables
\usepackage{multirow}
\usepackage{subfigure}
\usepackage{amsmath,amsmath,amssymb} 
\usepackage{fancyhdr}
\usepackage{epsfig, amsmath}
\usepackage{algorithm}
\usepackage{algorithmic}

\usepackage[section]{placeins}

% general settings
\hypersetup{
	linktocpage=true,
	colorlinks=true,
	linkcolor=blue,
	citecolor=blue,
}
\definecolor{Hgray}{gray}{0.6}

\newenvironment{definition}[1][Definition]{\begin{trivlist}
\item[\hskip \labelsep {\bfseries #1}]}{\end{trivlist}}

\setlength{\topmargin}{0cm}
\setlength{\textheight}{23cm}
\setlength{\textwidth}{17cm}
\setlength{\oddsidemargin}{0cm}
\setlength{\evensidemargin}{0cm}
\setlength{\headheight}{1cm}

% indica que las 'sub-sub-sections' sean numeradas y aparezcan en el indice
\setcounter{secnumdepth}{3}
\setcounter{tocdepth}{2}

% settings for code
\renewcommand{\algorithmicrequire}{\textbf{Entrada: }}
\renewcommand{\algorithmicensure}{\textbf{Salida: }}

\makeglossaries
\input{closing/glossary.tex}

\begin{document}

% Portada
\input{0_tittle.tex}


%\tableofcontents

\pagenumbering{roman} 
\setcounter{page}{1} 
\pagestyle{plain}

%%%%%%%%%%%%%
%%% FICHA %%%
%%%%%%%%%%%%%
\chapter*{FICHA DEL TRABAJO FINAL}
\begin{table}[ht]
	\centering{}
	\renewcommand{\arraystretch}{2}
	\begin{tabular}{r | l}
		\hline
		Título del trabajo: & Artificial MRI brain images creation with Variational Autoencoders\\
		\hline
        Nombre del autor: & Miguel Tablado León\\
		\hline
        Nombre del Tutor/a de TF: & Baris Kanber\\
		\hline
        Nombre del/de la PRA: & Ferran Prados Carrasco\\
		\hline
        Fecha de entrega: & 02/2023\\
		\hline
        Titulación o programa: & Máster en Ciencia de Datos\\
		\hline
        Área del Trabajo Final: & Medicine Area (TFM-Med)\\
		\hline
        Language: & English\\
		\hline
        Keywords & Deep Learning, Brain MRI, Variational Autoencoder\\
		\hline
	\end{tabular}
\end{table}

\pagestyle{fancy}
\renewcommand{\chaptermark}[1]{ \markboth{#1}{}}
\renewcommand{\sectionmark}[1]{\markright{ \thesection.\ #1}}
\lhead[\fancyplain{}{\bfseries\thepage}]{\fancyplain{}{\bfseries\rightmark}}
\rhead[\fancyplain{}{\bfseries\leftmark}]{\fancyplain{}{\bfseries\thepage}}
\cfoot{}

% indice
\cleardoublepage
\phantomsection
\addcontentsline{toc}{chapter}{Index}
\tableofcontents
% listado de figuras
\cleardoublepage
\phantomsection
\addcontentsline{toc}{chapter}{Figure List}
\listoffigures
% listado de tablas
\cleardoublepage
\phantomsection
\addcontentsline{toc}{chapter}{Table List}
\listoftables

\thispagestyle{empty}

\pagenumbering{arabic}

\pagestyle{fancy}
\renewcommand{\chaptermark}[1]{ \markboth{#1}{}}
\renewcommand{\sectionmark}[1]{\markright{ \thesection.\ #1}}
\lhead[\fancyplain{}{\bfseries\thepage}]{\fancyplain{}{\bfseries\rightmark}}
\rhead[\fancyplain{}{\bfseries\leftmark}]{\fancyplain{}{\bfseries\thepage}}
\cfoot{}

\onehalfspacing

\input{introduction/0_abstract.tex}

\chapter{Introduction}

\input{introduction/1_context.tex}
\input{introduction/2_aim.tex}
\input{introduction/5_plan.tex}

\input{state_of_art/main.tex}
\input{impl/main.tex}
\input{conclusions/main.tex}

\clearpage
\printglossary[type=\acronymtype]
\printglossary

\bibliographystyle{plain} % We choose the "plain" reference style
\bibliography{refs}

\end{document}

\documentclass[11pt,a4paper,openany]{book}

\usepackage{amsmath}
\usepackage{graphicx}
\usepackage{hyperref}
\usepackage[utf8]{inputenc}
\usepackage[acronym, toc]{glossaries}

\usepackage{setspace}	%double spacing for text, single for captions, footnotes, etc.
\usepackage{natbib}		% substituye a 'hypernat' que funciona en Windows.
\usepackage[english]{babel}
\usepackage[utf8]{inputenc}
\usepackage{color}
\usepackage{hhline} 		% extended styles for tables
\usepackage{multirow}
\usepackage{subfigure}
\usepackage{amsmath,amsmath,amssymb} 
\usepackage{fancyhdr}
\usepackage{epsfig, amsmath}
\usepackage{algorithm}
\usepackage{algorithmic}

\usepackage[section]{placeins}

% general settings
\hypersetup{
	linktocpage=true,
	colorlinks=true,
	linkcolor=blue,
	citecolor=blue,
}
\definecolor{Hgray}{gray}{0.6}

\newenvironment{definition}[1][Definition]{\begin{trivlist}
\item[\hskip \labelsep {\bfseries #1}]}{\end{trivlist}}

\setlength{\topmargin}{0cm}
\setlength{\textheight}{23cm}
\setlength{\textwidth}{17cm}
\setlength{\oddsidemargin}{0cm}
\setlength{\evensidemargin}{0cm}
\setlength{\headheight}{1cm}

% indica que las 'sub-sub-sections' sean numeradas y aparezcan en el indice
\setcounter{secnumdepth}{3}
\setcounter{tocdepth}{2}

% settings for code
\renewcommand{\algorithmicrequire}{\textbf{Entrada: }}
\renewcommand{\algorithmicensure}{\textbf{Salida: }}

\makeglossaries
\input{closing/glossary.tex}

\begin{document}

% Portada
\input{0_tittle.tex}


%\tableofcontents

\pagenumbering{roman} 
\setcounter{page}{1} 
\pagestyle{plain}

%%%%%%%%%%%%%
%%% FICHA %%%
%%%%%%%%%%%%%
\chapter*{FICHA DEL TRABAJO FINAL}
\begin{table}[ht]
	\centering{}
	\renewcommand{\arraystretch}{2}
	\begin{tabular}{r | l}
		\hline
		Título del trabajo: & Artificial MRI brain images creation with Variational Autoencoders\\
		\hline
        Nombre del autor: & Miguel Tablado León\\
		\hline
        Nombre del Tutor/a de TF: & Baris Kanber\\
		\hline
        Nombre del/de la PRA: & Ferran Prados Carrasco\\
		\hline
        Fecha de entrega: & 02/2023\\
		\hline
        Titulación o programa: & Máster en Ciencia de Datos\\
		\hline
        Área del Trabajo Final: & Medicine Area (TFM-Med)\\
		\hline
        Language: & English\\
		\hline
        Keywords & Deep Learning, Brain MRI, Variational Autoencoder\\
		\hline
	\end{tabular}
\end{table}

\pagestyle{fancy}
\renewcommand{\chaptermark}[1]{ \markboth{#1}{}}
\renewcommand{\sectionmark}[1]{\markright{ \thesection.\ #1}}
\lhead[\fancyplain{}{\bfseries\thepage}]{\fancyplain{}{\bfseries\rightmark}}
\rhead[\fancyplain{}{\bfseries\leftmark}]{\fancyplain{}{\bfseries\thepage}}
\cfoot{}

% indice
\cleardoublepage
\phantomsection
\addcontentsline{toc}{chapter}{Index}
\tableofcontents
% listado de figuras
\cleardoublepage
\phantomsection
\addcontentsline{toc}{chapter}{Figure List}
\listoffigures
% listado de tablas
\cleardoublepage
\phantomsection
\addcontentsline{toc}{chapter}{Table List}
\listoftables

\thispagestyle{empty}

\pagenumbering{arabic}

\pagestyle{fancy}
\renewcommand{\chaptermark}[1]{ \markboth{#1}{}}
\renewcommand{\sectionmark}[1]{\markright{ \thesection.\ #1}}
\lhead[\fancyplain{}{\bfseries\thepage}]{\fancyplain{}{\bfseries\rightmark}}
\rhead[\fancyplain{}{\bfseries\leftmark}]{\fancyplain{}{\bfseries\thepage}}
\cfoot{}

\onehalfspacing

\input{introduction/0_abstract.tex}

\chapter{Introduction}

\input{introduction/1_context.tex}
\input{introduction/2_aim.tex}
\input{introduction/5_plan.tex}

\input{state_of_art/main.tex}
\input{impl/main.tex}
\input{conclusions/main.tex}

\clearpage
\printglossary[type=\acronymtype]
\printglossary

\bibliographystyle{plain} % We choose the "plain" reference style
\bibliography{refs}

\end{document}

\documentclass[11pt,a4paper,openany]{book}

\usepackage{amsmath}
\usepackage{graphicx}
\usepackage{hyperref}
\usepackage[utf8]{inputenc}
\usepackage[acronym, toc]{glossaries}

\usepackage{setspace}	%double spacing for text, single for captions, footnotes, etc.
\usepackage{natbib}		% substituye a 'hypernat' que funciona en Windows.
\usepackage[english]{babel}
\usepackage[utf8]{inputenc}
\usepackage{color}
\usepackage{hhline} 		% extended styles for tables
\usepackage{multirow}
\usepackage{subfigure}
\usepackage{amsmath,amsmath,amssymb} 
\usepackage{fancyhdr}
\usepackage{epsfig, amsmath}
\usepackage{algorithm}
\usepackage{algorithmic}

\usepackage[section]{placeins}

% general settings
\hypersetup{
	linktocpage=true,
	colorlinks=true,
	linkcolor=blue,
	citecolor=blue,
}
\definecolor{Hgray}{gray}{0.6}

\newenvironment{definition}[1][Definition]{\begin{trivlist}
\item[\hskip \labelsep {\bfseries #1}]}{\end{trivlist}}

\setlength{\topmargin}{0cm}
\setlength{\textheight}{23cm}
\setlength{\textwidth}{17cm}
\setlength{\oddsidemargin}{0cm}
\setlength{\evensidemargin}{0cm}
\setlength{\headheight}{1cm}

% indica que las 'sub-sub-sections' sean numeradas y aparezcan en el indice
\setcounter{secnumdepth}{3}
\setcounter{tocdepth}{2}

% settings for code
\renewcommand{\algorithmicrequire}{\textbf{Entrada: }}
\renewcommand{\algorithmicensure}{\textbf{Salida: }}

\makeglossaries
\input{closing/glossary.tex}

\begin{document}

% Portada
\input{0_tittle.tex}


%\tableofcontents

\pagenumbering{roman} 
\setcounter{page}{1} 
\pagestyle{plain}

%%%%%%%%%%%%%
%%% FICHA %%%
%%%%%%%%%%%%%
\chapter*{FICHA DEL TRABAJO FINAL}
\begin{table}[ht]
	\centering{}
	\renewcommand{\arraystretch}{2}
	\begin{tabular}{r | l}
		\hline
		Título del trabajo: & Artificial MRI brain images creation with Variational Autoencoders\\
		\hline
        Nombre del autor: & Miguel Tablado León\\
		\hline
        Nombre del Tutor/a de TF: & Baris Kanber\\
		\hline
        Nombre del/de la PRA: & Ferran Prados Carrasco\\
		\hline
        Fecha de entrega: & 02/2023\\
		\hline
        Titulación o programa: & Máster en Ciencia de Datos\\
		\hline
        Área del Trabajo Final: & Medicine Area (TFM-Med)\\
		\hline
        Language: & English\\
		\hline
        Keywords & Deep Learning, Brain MRI, Variational Autoencoder\\
		\hline
	\end{tabular}
\end{table}

\pagestyle{fancy}
\renewcommand{\chaptermark}[1]{ \markboth{#1}{}}
\renewcommand{\sectionmark}[1]{\markright{ \thesection.\ #1}}
\lhead[\fancyplain{}{\bfseries\thepage}]{\fancyplain{}{\bfseries\rightmark}}
\rhead[\fancyplain{}{\bfseries\leftmark}]{\fancyplain{}{\bfseries\thepage}}
\cfoot{}

% indice
\cleardoublepage
\phantomsection
\addcontentsline{toc}{chapter}{Index}
\tableofcontents
% listado de figuras
\cleardoublepage
\phantomsection
\addcontentsline{toc}{chapter}{Figure List}
\listoffigures
% listado de tablas
\cleardoublepage
\phantomsection
\addcontentsline{toc}{chapter}{Table List}
\listoftables

\thispagestyle{empty}

\pagenumbering{arabic}

\pagestyle{fancy}
\renewcommand{\chaptermark}[1]{ \markboth{#1}{}}
\renewcommand{\sectionmark}[1]{\markright{ \thesection.\ #1}}
\lhead[\fancyplain{}{\bfseries\thepage}]{\fancyplain{}{\bfseries\rightmark}}
\rhead[\fancyplain{}{\bfseries\leftmark}]{\fancyplain{}{\bfseries\thepage}}
\cfoot{}

\onehalfspacing

\input{introduction/0_abstract.tex}

\chapter{Introduction}

\input{introduction/1_context.tex}
\input{introduction/2_aim.tex}
\input{introduction/5_plan.tex}

\input{state_of_art/main.tex}
\input{impl/main.tex}
\input{conclusions/main.tex}

\clearpage
\printglossary[type=\acronymtype]
\printglossary

\bibliographystyle{plain} % We choose the "plain" reference style
\bibliography{refs}

\end{document}


\clearpage
\printglossary[type=\acronymtype]
\printglossary

\bibliographystyle{plain} % We choose the "plain" reference style
\bibliography{refs}

\end{document}

\documentclass[11pt,a4paper,openany]{book}

\usepackage{amsmath}
\usepackage{graphicx}
\usepackage{hyperref}
\usepackage[utf8]{inputenc}
\usepackage[acronym, toc]{glossaries}

\usepackage{setspace}	%double spacing for text, single for captions, footnotes, etc.
\usepackage{natbib}		% substituye a 'hypernat' que funciona en Windows.
\usepackage[english]{babel}
\usepackage[utf8]{inputenc}
\usepackage{color}
\usepackage{hhline} 		% extended styles for tables
\usepackage{multirow}
\usepackage{subfigure}
\usepackage{amsmath,amsmath,amssymb} 
\usepackage{fancyhdr}
\usepackage{epsfig, amsmath}
\usepackage{algorithm}
\usepackage{algorithmic}

\usepackage[section]{placeins}

% general settings
\hypersetup{
	linktocpage=true,
	colorlinks=true,
	linkcolor=blue,
	citecolor=blue,
}
\definecolor{Hgray}{gray}{0.6}

\newenvironment{definition}[1][Definition]{\begin{trivlist}
\item[\hskip \labelsep {\bfseries #1}]}{\end{trivlist}}

\setlength{\topmargin}{0cm}
\setlength{\textheight}{23cm}
\setlength{\textwidth}{17cm}
\setlength{\oddsidemargin}{0cm}
\setlength{\evensidemargin}{0cm}
\setlength{\headheight}{1cm}

% indica que las 'sub-sub-sections' sean numeradas y aparezcan en el indice
\setcounter{secnumdepth}{3}
\setcounter{tocdepth}{2}

% settings for code
\renewcommand{\algorithmicrequire}{\textbf{Entrada: }}
\renewcommand{\algorithmicensure}{\textbf{Salida: }}

\makeglossaries
\newglossaryentry{kpi}
{
    name=KPI,
    description={Key Performance Indicator}
}

\newglossaryentry{mnist}
{
    name=MNIST,
    description={The MNIST database of handwritten digits used for training models}
}

\newglossaryentry{kdd}
{
    name=KDD,
    description={KDD cup 1999 network intrusion dataset}
}

\newglossaryentry{ixi}
{
    name=IXI,
    description={IXI dataset is a collection of 600 Brain MR Images, in 3-D, collected from 3 different hospitals in London. The images are scanned in 1.5 and 3 Tesla which is the range of the quality that we could usually find in hospitals around the world}
}

\newacronym{pca}{PCA}{Principal Component Analysis}

\newacronym{mri}{MRI}{Magnetic Resonance Image}

\newacronym{vqvae}{VQ-VAE}{Vector-Quantized Variational Autoencoder}

\newacronym{cevae}{ceVAE}{Context-Encoder Variational Autoencoder}

\newacronym{strega}{StRegA}{Segmentation Regularised Anomaly}


\begin{document}

% Portada
\newpage
\thispagestyle{empty}

\baselineskip 2em

%\vspace*{1cm}

\centerline{\includegraphics[width=0.6\textwidth]{images/UOC-logo}}
\begin{center}
\textsc{Universitat Oberta de Catalunya (UOC) \\
 Máster Universitario en Ciencia de Datos (\textit{Data Science})\\}

%\centerline {\pic{UOC}{4cm}}

\vspace*{1.5cm}

\textsc{\Large TRABAJO FINAL DE MÁSTER}

\vspace*{0.5cm}

\textsc{\large Área: YYY}


%\textbf{\Huge VirtualTechLab Model: }

\vspace*{2.0cm}

\title{\Large Artificial MRI brain images creation with Variational Autoencoders}

\vspace{2.5cm}
\baselineskip 1em

\baselineskip 2em
-----------------------------------------------------------------------------\\
Autor:      Miguel Tablado\\
Tutor:      Baris Kanber\\
Profesor:   Nombre del profesor responsable del área de TF\\
-----------------------------------------------------------------------------\\
\vspace*{1.5cm}
Barcelona, \today

\author{Miguel Tablado}

\end{center}



%\tableofcontents

\pagenumbering{roman} 
\setcounter{page}{1} 
\pagestyle{plain}

%%%%%%%%%%%%%
%%% FICHA %%%
%%%%%%%%%%%%%
\chapter*{FICHA DEL TRABAJO FINAL}
\begin{table}[ht]
	\centering{}
	\renewcommand{\arraystretch}{2}
	\begin{tabular}{r | l}
		\hline
		Título del trabajo: & Artificial MRI brain images creation with Variational Autoencoders\\
		\hline
        Nombre del autor: & Miguel Tablado León\\
		\hline
        Nombre del Tutor/a de TF: & Baris Kanber\\
		\hline
        Nombre del/de la PRA: & Ferran Prados Carrasco\\
		\hline
        Fecha de entrega: & 02/2023\\
		\hline
        Titulación o programa: & Máster en Ciencia de Datos\\
		\hline
        Área del Trabajo Final: & Medicine Area (TFM-Med)\\
		\hline
        Language: & English\\
		\hline
        Keywords & Deep Learning, Brain MRI, Variational Autoencoder\\
		\hline
	\end{tabular}
\end{table}

\pagestyle{fancy}
\renewcommand{\chaptermark}[1]{ \markboth{#1}{}}
\renewcommand{\sectionmark}[1]{\markright{ \thesection.\ #1}}
\lhead[\fancyplain{}{\bfseries\thepage}]{\fancyplain{}{\bfseries\rightmark}}
\rhead[\fancyplain{}{\bfseries\leftmark}]{\fancyplain{}{\bfseries\thepage}}
\cfoot{}

% indice
\cleardoublepage
\phantomsection
\addcontentsline{toc}{chapter}{Index}
\tableofcontents
% listado de figuras
\cleardoublepage
\phantomsection
\addcontentsline{toc}{chapter}{Figure List}
\listoffigures
% listado de tablas
\cleardoublepage
\phantomsection
\addcontentsline{toc}{chapter}{Table List}
\listoftables

\thispagestyle{empty}

\pagenumbering{arabic}

\pagestyle{fancy}
\renewcommand{\chaptermark}[1]{ \markboth{#1}{}}
\renewcommand{\sectionmark}[1]{\markright{ \thesection.\ #1}}
\lhead[\fancyplain{}{\bfseries\thepage}]{\fancyplain{}{\bfseries\rightmark}}
\rhead[\fancyplain{}{\bfseries\leftmark}]{\fancyplain{}{\bfseries\thepage}}
\cfoot{}

\onehalfspacing

\input{introduction/0_abstract.tex}

\chapter{Introduction}

\section{Context and project justification}

Artificial Intelligence has arrived to change the world in almost (if not all) any field. Today, we are surrounded by (and we are using many) AI products like smartphones’ face recognition capabilities, home cleaning robots or cars with autopilot options.

From the different AI fields, computer vision is maybe the most popular one and the one which is usually used for explaining AI capabilities to general public. Identifying a cat in a picture could perfectly be the example used in every AI presentation to welcome people to AI.

Image processing is intuitively matched with medical diagnosis by anyone having or not any expertise on the field. Every single citizen will have heard of magnetic resonance imaging, and anyone easily transposes ‘cat detection’ to ‘anomaly detection’, being the anomaly a tumor or anything else.

There are different techniques to scan people and create images for clinical diagnosis like X-Ray or Magnetic Resonance Imaging. In this project, we will work with Magnetic Resonance Images (MRI) which are images created by a machine with a large bore that scans people lying inside it. The MR technique is non-invasive, it produces no radiation, and is used to scan almost any part of the body from which we will focus on brain images.

Combining AI diagnosis capabilities on image processing and MRI images, we can think of helping doctors to identify the presence of anomalies or looking for concrete diagnosis for a specific disease.

Obviously, these projects are not easy at all and they face a lot of challenges. One of the first challenges that such AI project faces is the difficulty to obtain a large set of brain images that are needed to train an accurate AI model. Scanning people is too costly and requires a lot of time. and the challenge of having a large dataset gets harder once we understand that brain images may differ depending on age or gender.

Today, there is a clear limitation on how to reproduce or obtain healthy brain images for AI-based diagnosis projects while the appearance of new AI techniques known as Autoencoders and Variational Autoencoders introduces a new area of investigation to mitigate the gap.

This project aims to be a proof of concept to create AI-created images with new variational autoencoders that would serve to augment any existing MRI dataset and that will help to improve the accuracy of brain anomaly detection projects, including those which could be used for overall anomaly detection to others more specific which would help on concrete disease diagnosis.

Personal motivation comes from various angles:

\begin{itemize}
    \item One is to prove myself that I can work with new AI architectures demonstrating that I have acquired the knowledge needed (deep enough) to be productive and to be able to innovate in the health sector.
    \item Deep Learning has been the subject which I enjoyed the most, hence continuing with Autoencoders seems natural to me as the next step on AI adoption
    \item At no doubts, if I can contribute to help on brain issue detection or diagnosis, I will feel my life been completely fulfilled
\end{itemize}
\section{Aim of the project}

The aim of this project is to serve as a Proof of Concept on how MRI brain images can be artificially generated with Variational Autoencoders which ultimately would serve to enhance existing or new datasets to improve model accuracy by having larger samples of data.

Foreseen projects objectives are:

\begin{itemize}
    \item Obtain basic knowledge about MRI images and the NIFTI file format
    \item Obtain and visualise 2-D images from 3-D images in the dataset
    \item Select what range of slices (2-D images) from brain to be created
    \item Test different existing networks and choose the one to be used
    \item Tune network parameters
    \item Compare generated brain images against real ones, qualitatively
\end{itemize}
\section{Project Plan}

\subsection{Resources}

\begin{enumerate}
    \item 1 Data Scientist: Miguel Tablado will be playing this role and will dedicate 300h
    \item 1 Coach/Tutor: Baris Kanber will be assisting Miguel Tablado during the project
    \item MRI images and demographic information from IXI Dataset
    \item GPU resources are needed to train and test the network
\end{enumerate}

\subsection{High Level Plan}

The plan will be executed in 3 different phases with the listed tasks:
\begin{enumerate}
    \item Phase 1: Analysis
    \begin{enumerate}
        \item Gain knowledge on MRI and NIFTI protocol
        \item Describe images and dataset
        \item Extract 2D images
        \item Pre-processing images
    \end{enumerate}
    \item Phase 2: MR Images creation
    \begin{enumerate}
        \item Test different Network architectures
        \item Tune-up architectural network
    \end{enumerate}
    \item Phase 3: Project documentation
    \begin{enumerate}
        \item Write conclusions
        \item Create project documentation
        \item Create project presentation
    \end{enumerate}
\end{enumerate}

\begin{figure}[ht]
    \hspace*{-1.2in}
    \centering
    \includegraphics[width = 20cm, height = 6cm]{images/project-plan.png}
    \caption[]{Project Plan. source: https://www.onlinegantt.com}
    \label{fig:project-plan}
\end{figure}

\newpage
\subsection{Tasks}
\subsubsection*{Phase 1: Analysis}

During this face the different tasks will be executed to prepare the work and includes:
\begin{enumerate}
    \item Gain knowledge on MRI and the NIFTI file format: This activity consists of reading papers and documents to gain sufficient knowledge to execute the project. There is no need to become an expert on the matter but understanding how those files are and how to process them.
    \item Describe images and dataset: During this activity, a description of the dataset will be generated with a view of the quality of the dataset for the aim of the project and any findings which could result.
    \item Extract 2D images: Load 3D images and extract 2D slices from the original dataset, which will be depicted with code.
    \item Pre-processing images: Decide which transformations on the 2D images would help the project like pixel changes or applying gray-scale transformations.
\end{enumerate}

\subsubsection*{Phase 2: MR Images creation}

\begin{enumerate}
    \item Test different Network architectures: This task will take few existing network architectures and be tested with the dataset so that one of them will be selected to be improved and used as the project architecture.
    \item Tune-up architectural network: Tune the selected architecture with useful techniques like changing network layers
\end{enumerate}

\documentclass[11pt,a4paper,openany]{book}

\usepackage{amsmath}
\usepackage{graphicx}
\usepackage{hyperref}
\usepackage[utf8]{inputenc}
\usepackage[acronym, toc]{glossaries}

\usepackage{setspace}	%double spacing for text, single for captions, footnotes, etc.
\usepackage{natbib}		% substituye a 'hypernat' que funciona en Windows.
\usepackage[english]{babel}
\usepackage[utf8]{inputenc}
\usepackage{color}
\usepackage{hhline} 		% extended styles for tables
\usepackage{multirow}
\usepackage{subfigure}
\usepackage{amsmath,amsmath,amssymb} 
\usepackage{fancyhdr}
\usepackage{epsfig, amsmath}
\usepackage{algorithm}
\usepackage{algorithmic}

\usepackage[section]{placeins}

% general settings
\hypersetup{
	linktocpage=true,
	colorlinks=true,
	linkcolor=blue,
	citecolor=blue,
}
\definecolor{Hgray}{gray}{0.6}

\newenvironment{definition}[1][Definition]{\begin{trivlist}
\item[\hskip \labelsep {\bfseries #1}]}{\end{trivlist}}

\setlength{\topmargin}{0cm}
\setlength{\textheight}{23cm}
\setlength{\textwidth}{17cm}
\setlength{\oddsidemargin}{0cm}
\setlength{\evensidemargin}{0cm}
\setlength{\headheight}{1cm}

% indica que las 'sub-sub-sections' sean numeradas y aparezcan en el indice
\setcounter{secnumdepth}{3}
\setcounter{tocdepth}{2}

% settings for code
\renewcommand{\algorithmicrequire}{\textbf{Entrada: }}
\renewcommand{\algorithmicensure}{\textbf{Salida: }}

\makeglossaries
\input{closing/glossary.tex}

\begin{document}

% Portada
\input{0_tittle.tex}


%\tableofcontents

\pagenumbering{roman} 
\setcounter{page}{1} 
\pagestyle{plain}

%%%%%%%%%%%%%
%%% FICHA %%%
%%%%%%%%%%%%%
\chapter*{FICHA DEL TRABAJO FINAL}
\begin{table}[ht]
	\centering{}
	\renewcommand{\arraystretch}{2}
	\begin{tabular}{r | l}
		\hline
		Título del trabajo: & Artificial MRI brain images creation with Variational Autoencoders\\
		\hline
        Nombre del autor: & Miguel Tablado León\\
		\hline
        Nombre del Tutor/a de TF: & Baris Kanber\\
		\hline
        Nombre del/de la PRA: & Ferran Prados Carrasco\\
		\hline
        Fecha de entrega: & 02/2023\\
		\hline
        Titulación o programa: & Máster en Ciencia de Datos\\
		\hline
        Área del Trabajo Final: & Medicine Area (TFM-Med)\\
		\hline
        Language: & English\\
		\hline
        Keywords & Deep Learning, Brain MRI, Variational Autoencoder\\
		\hline
	\end{tabular}
\end{table}

\pagestyle{fancy}
\renewcommand{\chaptermark}[1]{ \markboth{#1}{}}
\renewcommand{\sectionmark}[1]{\markright{ \thesection.\ #1}}
\lhead[\fancyplain{}{\bfseries\thepage}]{\fancyplain{}{\bfseries\rightmark}}
\rhead[\fancyplain{}{\bfseries\leftmark}]{\fancyplain{}{\bfseries\thepage}}
\cfoot{}

% indice
\cleardoublepage
\phantomsection
\addcontentsline{toc}{chapter}{Index}
\tableofcontents
% listado de figuras
\cleardoublepage
\phantomsection
\addcontentsline{toc}{chapter}{Figure List}
\listoffigures
% listado de tablas
\cleardoublepage
\phantomsection
\addcontentsline{toc}{chapter}{Table List}
\listoftables

\thispagestyle{empty}

\pagenumbering{arabic}

\pagestyle{fancy}
\renewcommand{\chaptermark}[1]{ \markboth{#1}{}}
\renewcommand{\sectionmark}[1]{\markright{ \thesection.\ #1}}
\lhead[\fancyplain{}{\bfseries\thepage}]{\fancyplain{}{\bfseries\rightmark}}
\rhead[\fancyplain{}{\bfseries\leftmark}]{\fancyplain{}{\bfseries\thepage}}
\cfoot{}

\onehalfspacing

\input{introduction/0_abstract.tex}

\chapter{Introduction}

\input{introduction/1_context.tex}
\input{introduction/2_aim.tex}
\input{introduction/5_plan.tex}

\input{state_of_art/main.tex}
\input{impl/main.tex}
\input{conclusions/main.tex}

\clearpage
\printglossary[type=\acronymtype]
\printglossary

\bibliographystyle{plain} % We choose the "plain" reference style
\bibliography{refs}

\end{document}

\documentclass[11pt,a4paper,openany]{book}

\usepackage{amsmath}
\usepackage{graphicx}
\usepackage{hyperref}
\usepackage[utf8]{inputenc}
\usepackage[acronym, toc]{glossaries}

\usepackage{setspace}	%double spacing for text, single for captions, footnotes, etc.
\usepackage{natbib}		% substituye a 'hypernat' que funciona en Windows.
\usepackage[english]{babel}
\usepackage[utf8]{inputenc}
\usepackage{color}
\usepackage{hhline} 		% extended styles for tables
\usepackage{multirow}
\usepackage{subfigure}
\usepackage{amsmath,amsmath,amssymb} 
\usepackage{fancyhdr}
\usepackage{epsfig, amsmath}
\usepackage{algorithm}
\usepackage{algorithmic}

\usepackage[section]{placeins}

% general settings
\hypersetup{
	linktocpage=true,
	colorlinks=true,
	linkcolor=blue,
	citecolor=blue,
}
\definecolor{Hgray}{gray}{0.6}

\newenvironment{definition}[1][Definition]{\begin{trivlist}
\item[\hskip \labelsep {\bfseries #1}]}{\end{trivlist}}

\setlength{\topmargin}{0cm}
\setlength{\textheight}{23cm}
\setlength{\textwidth}{17cm}
\setlength{\oddsidemargin}{0cm}
\setlength{\evensidemargin}{0cm}
\setlength{\headheight}{1cm}

% indica que las 'sub-sub-sections' sean numeradas y aparezcan en el indice
\setcounter{secnumdepth}{3}
\setcounter{tocdepth}{2}

% settings for code
\renewcommand{\algorithmicrequire}{\textbf{Entrada: }}
\renewcommand{\algorithmicensure}{\textbf{Salida: }}

\makeglossaries
\input{closing/glossary.tex}

\begin{document}

% Portada
\input{0_tittle.tex}


%\tableofcontents

\pagenumbering{roman} 
\setcounter{page}{1} 
\pagestyle{plain}

%%%%%%%%%%%%%
%%% FICHA %%%
%%%%%%%%%%%%%
\chapter*{FICHA DEL TRABAJO FINAL}
\begin{table}[ht]
	\centering{}
	\renewcommand{\arraystretch}{2}
	\begin{tabular}{r | l}
		\hline
		Título del trabajo: & Artificial MRI brain images creation with Variational Autoencoders\\
		\hline
        Nombre del autor: & Miguel Tablado León\\
		\hline
        Nombre del Tutor/a de TF: & Baris Kanber\\
		\hline
        Nombre del/de la PRA: & Ferran Prados Carrasco\\
		\hline
        Fecha de entrega: & 02/2023\\
		\hline
        Titulación o programa: & Máster en Ciencia de Datos\\
		\hline
        Área del Trabajo Final: & Medicine Area (TFM-Med)\\
		\hline
        Language: & English\\
		\hline
        Keywords & Deep Learning, Brain MRI, Variational Autoencoder\\
		\hline
	\end{tabular}
\end{table}

\pagestyle{fancy}
\renewcommand{\chaptermark}[1]{ \markboth{#1}{}}
\renewcommand{\sectionmark}[1]{\markright{ \thesection.\ #1}}
\lhead[\fancyplain{}{\bfseries\thepage}]{\fancyplain{}{\bfseries\rightmark}}
\rhead[\fancyplain{}{\bfseries\leftmark}]{\fancyplain{}{\bfseries\thepage}}
\cfoot{}

% indice
\cleardoublepage
\phantomsection
\addcontentsline{toc}{chapter}{Index}
\tableofcontents
% listado de figuras
\cleardoublepage
\phantomsection
\addcontentsline{toc}{chapter}{Figure List}
\listoffigures
% listado de tablas
\cleardoublepage
\phantomsection
\addcontentsline{toc}{chapter}{Table List}
\listoftables

\thispagestyle{empty}

\pagenumbering{arabic}

\pagestyle{fancy}
\renewcommand{\chaptermark}[1]{ \markboth{#1}{}}
\renewcommand{\sectionmark}[1]{\markright{ \thesection.\ #1}}
\lhead[\fancyplain{}{\bfseries\thepage}]{\fancyplain{}{\bfseries\rightmark}}
\rhead[\fancyplain{}{\bfseries\leftmark}]{\fancyplain{}{\bfseries\thepage}}
\cfoot{}

\onehalfspacing

\input{introduction/0_abstract.tex}

\chapter{Introduction}

\input{introduction/1_context.tex}
\input{introduction/2_aim.tex}
\input{introduction/5_plan.tex}

\input{state_of_art/main.tex}
\input{impl/main.tex}
\input{conclusions/main.tex}

\clearpage
\printglossary[type=\acronymtype]
\printglossary

\bibliographystyle{plain} % We choose the "plain" reference style
\bibliography{refs}

\end{document}

\documentclass[11pt,a4paper,openany]{book}

\usepackage{amsmath}
\usepackage{graphicx}
\usepackage{hyperref}
\usepackage[utf8]{inputenc}
\usepackage[acronym, toc]{glossaries}

\usepackage{setspace}	%double spacing for text, single for captions, footnotes, etc.
\usepackage{natbib}		% substituye a 'hypernat' que funciona en Windows.
\usepackage[english]{babel}
\usepackage[utf8]{inputenc}
\usepackage{color}
\usepackage{hhline} 		% extended styles for tables
\usepackage{multirow}
\usepackage{subfigure}
\usepackage{amsmath,amsmath,amssymb} 
\usepackage{fancyhdr}
\usepackage{epsfig, amsmath}
\usepackage{algorithm}
\usepackage{algorithmic}

\usepackage[section]{placeins}

% general settings
\hypersetup{
	linktocpage=true,
	colorlinks=true,
	linkcolor=blue,
	citecolor=blue,
}
\definecolor{Hgray}{gray}{0.6}

\newenvironment{definition}[1][Definition]{\begin{trivlist}
\item[\hskip \labelsep {\bfseries #1}]}{\end{trivlist}}

\setlength{\topmargin}{0cm}
\setlength{\textheight}{23cm}
\setlength{\textwidth}{17cm}
\setlength{\oddsidemargin}{0cm}
\setlength{\evensidemargin}{0cm}
\setlength{\headheight}{1cm}

% indica que las 'sub-sub-sections' sean numeradas y aparezcan en el indice
\setcounter{secnumdepth}{3}
\setcounter{tocdepth}{2}

% settings for code
\renewcommand{\algorithmicrequire}{\textbf{Entrada: }}
\renewcommand{\algorithmicensure}{\textbf{Salida: }}

\makeglossaries
\input{closing/glossary.tex}

\begin{document}

% Portada
\input{0_tittle.tex}


%\tableofcontents

\pagenumbering{roman} 
\setcounter{page}{1} 
\pagestyle{plain}

%%%%%%%%%%%%%
%%% FICHA %%%
%%%%%%%%%%%%%
\chapter*{FICHA DEL TRABAJO FINAL}
\begin{table}[ht]
	\centering{}
	\renewcommand{\arraystretch}{2}
	\begin{tabular}{r | l}
		\hline
		Título del trabajo: & Artificial MRI brain images creation with Variational Autoencoders\\
		\hline
        Nombre del autor: & Miguel Tablado León\\
		\hline
        Nombre del Tutor/a de TF: & Baris Kanber\\
		\hline
        Nombre del/de la PRA: & Ferran Prados Carrasco\\
		\hline
        Fecha de entrega: & 02/2023\\
		\hline
        Titulación o programa: & Máster en Ciencia de Datos\\
		\hline
        Área del Trabajo Final: & Medicine Area (TFM-Med)\\
		\hline
        Language: & English\\
		\hline
        Keywords & Deep Learning, Brain MRI, Variational Autoencoder\\
		\hline
	\end{tabular}
\end{table}

\pagestyle{fancy}
\renewcommand{\chaptermark}[1]{ \markboth{#1}{}}
\renewcommand{\sectionmark}[1]{\markright{ \thesection.\ #1}}
\lhead[\fancyplain{}{\bfseries\thepage}]{\fancyplain{}{\bfseries\rightmark}}
\rhead[\fancyplain{}{\bfseries\leftmark}]{\fancyplain{}{\bfseries\thepage}}
\cfoot{}

% indice
\cleardoublepage
\phantomsection
\addcontentsline{toc}{chapter}{Index}
\tableofcontents
% listado de figuras
\cleardoublepage
\phantomsection
\addcontentsline{toc}{chapter}{Figure List}
\listoffigures
% listado de tablas
\cleardoublepage
\phantomsection
\addcontentsline{toc}{chapter}{Table List}
\listoftables

\thispagestyle{empty}

\pagenumbering{arabic}

\pagestyle{fancy}
\renewcommand{\chaptermark}[1]{ \markboth{#1}{}}
\renewcommand{\sectionmark}[1]{\markright{ \thesection.\ #1}}
\lhead[\fancyplain{}{\bfseries\thepage}]{\fancyplain{}{\bfseries\rightmark}}
\rhead[\fancyplain{}{\bfseries\leftmark}]{\fancyplain{}{\bfseries\thepage}}
\cfoot{}

\onehalfspacing

\input{introduction/0_abstract.tex}

\chapter{Introduction}

\input{introduction/1_context.tex}
\input{introduction/2_aim.tex}
\input{introduction/5_plan.tex}

\input{state_of_art/main.tex}
\input{impl/main.tex}
\input{conclusions/main.tex}

\clearpage
\printglossary[type=\acronymtype]
\printglossary

\bibliographystyle{plain} % We choose the "plain" reference style
\bibliography{refs}

\end{document}


\clearpage
\printglossary[type=\acronymtype]
\printglossary

\bibliographystyle{plain} % We choose the "plain" reference style
\bibliography{refs}

\end{document}

\documentclass[11pt,a4paper,openany]{book}

\usepackage{amsmath}
\usepackage{graphicx}
\usepackage{hyperref}
\usepackage[utf8]{inputenc}
\usepackage[acronym, toc]{glossaries}

\usepackage{setspace}	%double spacing for text, single for captions, footnotes, etc.
\usepackage{natbib}		% substituye a 'hypernat' que funciona en Windows.
\usepackage[english]{babel}
\usepackage[utf8]{inputenc}
\usepackage{color}
\usepackage{hhline} 		% extended styles for tables
\usepackage{multirow}
\usepackage{subfigure}
\usepackage{amsmath,amsmath,amssymb} 
\usepackage{fancyhdr}
\usepackage{epsfig, amsmath}
\usepackage{algorithm}
\usepackage{algorithmic}

\usepackage[section]{placeins}

% general settings
\hypersetup{
	linktocpage=true,
	colorlinks=true,
	linkcolor=blue,
	citecolor=blue,
}
\definecolor{Hgray}{gray}{0.6}

\newenvironment{definition}[1][Definition]{\begin{trivlist}
\item[\hskip \labelsep {\bfseries #1}]}{\end{trivlist}}

\setlength{\topmargin}{0cm}
\setlength{\textheight}{23cm}
\setlength{\textwidth}{17cm}
\setlength{\oddsidemargin}{0cm}
\setlength{\evensidemargin}{0cm}
\setlength{\headheight}{1cm}

% indica que las 'sub-sub-sections' sean numeradas y aparezcan en el indice
\setcounter{secnumdepth}{3}
\setcounter{tocdepth}{2}

% settings for code
\renewcommand{\algorithmicrequire}{\textbf{Entrada: }}
\renewcommand{\algorithmicensure}{\textbf{Salida: }}

\makeglossaries
\newglossaryentry{kpi}
{
    name=KPI,
    description={Key Performance Indicator}
}

\newglossaryentry{mnist}
{
    name=MNIST,
    description={The MNIST database of handwritten digits used for training models}
}

\newglossaryentry{kdd}
{
    name=KDD,
    description={KDD cup 1999 network intrusion dataset}
}

\newglossaryentry{ixi}
{
    name=IXI,
    description={IXI dataset is a collection of 600 Brain MR Images, in 3-D, collected from 3 different hospitals in London. The images are scanned in 1.5 and 3 Tesla which is the range of the quality that we could usually find in hospitals around the world}
}

\newacronym{pca}{PCA}{Principal Component Analysis}

\newacronym{mri}{MRI}{Magnetic Resonance Image}

\newacronym{vqvae}{VQ-VAE}{Vector-Quantized Variational Autoencoder}

\newacronym{cevae}{ceVAE}{Context-Encoder Variational Autoencoder}

\newacronym{strega}{StRegA}{Segmentation Regularised Anomaly}


\begin{document}

% Portada
\newpage
\thispagestyle{empty}

\baselineskip 2em

%\vspace*{1cm}

\centerline{\includegraphics[width=0.6\textwidth]{images/UOC-logo}}
\begin{center}
\textsc{Universitat Oberta de Catalunya (UOC) \\
 Máster Universitario en Ciencia de Datos (\textit{Data Science})\\}

%\centerline {\pic{UOC}{4cm}}

\vspace*{1.5cm}

\textsc{\Large TRABAJO FINAL DE MÁSTER}

\vspace*{0.5cm}

\textsc{\large Área: YYY}


%\textbf{\Huge VirtualTechLab Model: }

\vspace*{2.0cm}

\title{\Large Artificial MRI brain images creation with Variational Autoencoders}

\vspace{2.5cm}
\baselineskip 1em

\baselineskip 2em
-----------------------------------------------------------------------------\\
Autor:      Miguel Tablado\\
Tutor:      Baris Kanber\\
Profesor:   Nombre del profesor responsable del área de TF\\
-----------------------------------------------------------------------------\\
\vspace*{1.5cm}
Barcelona, \today

\author{Miguel Tablado}

\end{center}



%\tableofcontents

\pagenumbering{roman} 
\setcounter{page}{1} 
\pagestyle{plain}

%%%%%%%%%%%%%
%%% FICHA %%%
%%%%%%%%%%%%%
\chapter*{FICHA DEL TRABAJO FINAL}
\begin{table}[ht]
	\centering{}
	\renewcommand{\arraystretch}{2}
	\begin{tabular}{r | l}
		\hline
		Título del trabajo: & Artificial MRI brain images creation with Variational Autoencoders\\
		\hline
        Nombre del autor: & Miguel Tablado León\\
		\hline
        Nombre del Tutor/a de TF: & Baris Kanber\\
		\hline
        Nombre del/de la PRA: & Ferran Prados Carrasco\\
		\hline
        Fecha de entrega: & 02/2023\\
		\hline
        Titulación o programa: & Máster en Ciencia de Datos\\
		\hline
        Área del Trabajo Final: & Medicine Area (TFM-Med)\\
		\hline
        Language: & English\\
		\hline
        Keywords & Deep Learning, Brain MRI, Variational Autoencoder\\
		\hline
	\end{tabular}
\end{table}

\pagestyle{fancy}
\renewcommand{\chaptermark}[1]{ \markboth{#1}{}}
\renewcommand{\sectionmark}[1]{\markright{ \thesection.\ #1}}
\lhead[\fancyplain{}{\bfseries\thepage}]{\fancyplain{}{\bfseries\rightmark}}
\rhead[\fancyplain{}{\bfseries\leftmark}]{\fancyplain{}{\bfseries\thepage}}
\cfoot{}

% indice
\cleardoublepage
\phantomsection
\addcontentsline{toc}{chapter}{Index}
\tableofcontents
% listado de figuras
\cleardoublepage
\phantomsection
\addcontentsline{toc}{chapter}{Figure List}
\listoffigures
% listado de tablas
\cleardoublepage
\phantomsection
\addcontentsline{toc}{chapter}{Table List}
\listoftables

\thispagestyle{empty}

\pagenumbering{arabic}

\pagestyle{fancy}
\renewcommand{\chaptermark}[1]{ \markboth{#1}{}}
\renewcommand{\sectionmark}[1]{\markright{ \thesection.\ #1}}
\lhead[\fancyplain{}{\bfseries\thepage}]{\fancyplain{}{\bfseries\rightmark}}
\rhead[\fancyplain{}{\bfseries\leftmark}]{\fancyplain{}{\bfseries\thepage}}
\cfoot{}

\onehalfspacing

\input{introduction/0_abstract.tex}

\chapter{Introduction}

\section{Context and project justification}

Artificial Intelligence has arrived to change the world in almost (if not all) any field. Today, we are surrounded by (and we are using many) AI products like smartphones’ face recognition capabilities, home cleaning robots or cars with autopilot options.

From the different AI fields, computer vision is maybe the most popular one and the one which is usually used for explaining AI capabilities to general public. Identifying a cat in a picture could perfectly be the example used in every AI presentation to welcome people to AI.

Image processing is intuitively matched with medical diagnosis by anyone having or not any expertise on the field. Every single citizen will have heard of magnetic resonance imaging, and anyone easily transposes ‘cat detection’ to ‘anomaly detection’, being the anomaly a tumor or anything else.

There are different techniques to scan people and create images for clinical diagnosis like X-Ray or Magnetic Resonance Imaging. In this project, we will work with Magnetic Resonance Images (MRI) which are images created by a machine with a large bore that scans people lying inside it. The MR technique is non-invasive, it produces no radiation, and is used to scan almost any part of the body from which we will focus on brain images.

Combining AI diagnosis capabilities on image processing and MRI images, we can think of helping doctors to identify the presence of anomalies or looking for concrete diagnosis for a specific disease.

Obviously, these projects are not easy at all and they face a lot of challenges. One of the first challenges that such AI project faces is the difficulty to obtain a large set of brain images that are needed to train an accurate AI model. Scanning people is too costly and requires a lot of time. and the challenge of having a large dataset gets harder once we understand that brain images may differ depending on age or gender.

Today, there is a clear limitation on how to reproduce or obtain healthy brain images for AI-based diagnosis projects while the appearance of new AI techniques known as Autoencoders and Variational Autoencoders introduces a new area of investigation to mitigate the gap.

This project aims to be a proof of concept to create AI-created images with new variational autoencoders that would serve to augment any existing MRI dataset and that will help to improve the accuracy of brain anomaly detection projects, including those which could be used for overall anomaly detection to others more specific which would help on concrete disease diagnosis.

Personal motivation comes from various angles:

\begin{itemize}
    \item One is to prove myself that I can work with new AI architectures demonstrating that I have acquired the knowledge needed (deep enough) to be productive and to be able to innovate in the health sector.
    \item Deep Learning has been the subject which I enjoyed the most, hence continuing with Autoencoders seems natural to me as the next step on AI adoption
    \item At no doubts, if I can contribute to help on brain issue detection or diagnosis, I will feel my life been completely fulfilled
\end{itemize}
\section{Aim of the project}

The aim of this project is to serve as a Proof of Concept on how MRI brain images can be artificially generated with Variational Autoencoders which ultimately would serve to enhance existing or new datasets to improve model accuracy by having larger samples of data.

Foreseen projects objectives are:

\begin{itemize}
    \item Obtain basic knowledge about MRI images and the NIFTI file format
    \item Obtain and visualise 2-D images from 3-D images in the dataset
    \item Select what range of slices (2-D images) from brain to be created
    \item Test different existing networks and choose the one to be used
    \item Tune network parameters
    \item Compare generated brain images against real ones, qualitatively
\end{itemize}
\section{Project Plan}

\subsection{Resources}

\begin{enumerate}
    \item 1 Data Scientist: Miguel Tablado will be playing this role and will dedicate 300h
    \item 1 Coach/Tutor: Baris Kanber will be assisting Miguel Tablado during the project
    \item MRI images and demographic information from IXI Dataset
    \item GPU resources are needed to train and test the network
\end{enumerate}

\subsection{High Level Plan}

The plan will be executed in 3 different phases with the listed tasks:
\begin{enumerate}
    \item Phase 1: Analysis
    \begin{enumerate}
        \item Gain knowledge on MRI and NIFTI protocol
        \item Describe images and dataset
        \item Extract 2D images
        \item Pre-processing images
    \end{enumerate}
    \item Phase 2: MR Images creation
    \begin{enumerate}
        \item Test different Network architectures
        \item Tune-up architectural network
    \end{enumerate}
    \item Phase 3: Project documentation
    \begin{enumerate}
        \item Write conclusions
        \item Create project documentation
        \item Create project presentation
    \end{enumerate}
\end{enumerate}

\begin{figure}[ht]
    \hspace*{-1.2in}
    \centering
    \includegraphics[width = 20cm, height = 6cm]{images/project-plan.png}
    \caption[]{Project Plan. source: https://www.onlinegantt.com}
    \label{fig:project-plan}
\end{figure}

\newpage
\subsection{Tasks}
\subsubsection*{Phase 1: Analysis}

During this face the different tasks will be executed to prepare the work and includes:
\begin{enumerate}
    \item Gain knowledge on MRI and the NIFTI file format: This activity consists of reading papers and documents to gain sufficient knowledge to execute the project. There is no need to become an expert on the matter but understanding how those files are and how to process them.
    \item Describe images and dataset: During this activity, a description of the dataset will be generated with a view of the quality of the dataset for the aim of the project and any findings which could result.
    \item Extract 2D images: Load 3D images and extract 2D slices from the original dataset, which will be depicted with code.
    \item Pre-processing images: Decide which transformations on the 2D images would help the project like pixel changes or applying gray-scale transformations.
\end{enumerate}

\subsubsection*{Phase 2: MR Images creation}

\begin{enumerate}
    \item Test different Network architectures: This task will take few existing network architectures and be tested with the dataset so that one of them will be selected to be improved and used as the project architecture.
    \item Tune-up architectural network: Tune the selected architecture with useful techniques like changing network layers
\end{enumerate}

\documentclass[11pt,a4paper,openany]{book}

\usepackage{amsmath}
\usepackage{graphicx}
\usepackage{hyperref}
\usepackage[utf8]{inputenc}
\usepackage[acronym, toc]{glossaries}

\usepackage{setspace}	%double spacing for text, single for captions, footnotes, etc.
\usepackage{natbib}		% substituye a 'hypernat' que funciona en Windows.
\usepackage[english]{babel}
\usepackage[utf8]{inputenc}
\usepackage{color}
\usepackage{hhline} 		% extended styles for tables
\usepackage{multirow}
\usepackage{subfigure}
\usepackage{amsmath,amsmath,amssymb} 
\usepackage{fancyhdr}
\usepackage{epsfig, amsmath}
\usepackage{algorithm}
\usepackage{algorithmic}

\usepackage[section]{placeins}

% general settings
\hypersetup{
	linktocpage=true,
	colorlinks=true,
	linkcolor=blue,
	citecolor=blue,
}
\definecolor{Hgray}{gray}{0.6}

\newenvironment{definition}[1][Definition]{\begin{trivlist}
\item[\hskip \labelsep {\bfseries #1}]}{\end{trivlist}}

\setlength{\topmargin}{0cm}
\setlength{\textheight}{23cm}
\setlength{\textwidth}{17cm}
\setlength{\oddsidemargin}{0cm}
\setlength{\evensidemargin}{0cm}
\setlength{\headheight}{1cm}

% indica que las 'sub-sub-sections' sean numeradas y aparezcan en el indice
\setcounter{secnumdepth}{3}
\setcounter{tocdepth}{2}

% settings for code
\renewcommand{\algorithmicrequire}{\textbf{Entrada: }}
\renewcommand{\algorithmicensure}{\textbf{Salida: }}

\makeglossaries
\input{closing/glossary.tex}

\begin{document}

% Portada
\input{0_tittle.tex}


%\tableofcontents

\pagenumbering{roman} 
\setcounter{page}{1} 
\pagestyle{plain}

%%%%%%%%%%%%%
%%% FICHA %%%
%%%%%%%%%%%%%
\chapter*{FICHA DEL TRABAJO FINAL}
\begin{table}[ht]
	\centering{}
	\renewcommand{\arraystretch}{2}
	\begin{tabular}{r | l}
		\hline
		Título del trabajo: & Artificial MRI brain images creation with Variational Autoencoders\\
		\hline
        Nombre del autor: & Miguel Tablado León\\
		\hline
        Nombre del Tutor/a de TF: & Baris Kanber\\
		\hline
        Nombre del/de la PRA: & Ferran Prados Carrasco\\
		\hline
        Fecha de entrega: & 02/2023\\
		\hline
        Titulación o programa: & Máster en Ciencia de Datos\\
		\hline
        Área del Trabajo Final: & Medicine Area (TFM-Med)\\
		\hline
        Language: & English\\
		\hline
        Keywords & Deep Learning, Brain MRI, Variational Autoencoder\\
		\hline
	\end{tabular}
\end{table}

\pagestyle{fancy}
\renewcommand{\chaptermark}[1]{ \markboth{#1}{}}
\renewcommand{\sectionmark}[1]{\markright{ \thesection.\ #1}}
\lhead[\fancyplain{}{\bfseries\thepage}]{\fancyplain{}{\bfseries\rightmark}}
\rhead[\fancyplain{}{\bfseries\leftmark}]{\fancyplain{}{\bfseries\thepage}}
\cfoot{}

% indice
\cleardoublepage
\phantomsection
\addcontentsline{toc}{chapter}{Index}
\tableofcontents
% listado de figuras
\cleardoublepage
\phantomsection
\addcontentsline{toc}{chapter}{Figure List}
\listoffigures
% listado de tablas
\cleardoublepage
\phantomsection
\addcontentsline{toc}{chapter}{Table List}
\listoftables

\thispagestyle{empty}

\pagenumbering{arabic}

\pagestyle{fancy}
\renewcommand{\chaptermark}[1]{ \markboth{#1}{}}
\renewcommand{\sectionmark}[1]{\markright{ \thesection.\ #1}}
\lhead[\fancyplain{}{\bfseries\thepage}]{\fancyplain{}{\bfseries\rightmark}}
\rhead[\fancyplain{}{\bfseries\leftmark}]{\fancyplain{}{\bfseries\thepage}}
\cfoot{}

\onehalfspacing

\input{introduction/0_abstract.tex}

\chapter{Introduction}

\input{introduction/1_context.tex}
\input{introduction/2_aim.tex}
\input{introduction/5_plan.tex}

\input{state_of_art/main.tex}
\input{impl/main.tex}
\input{conclusions/main.tex}

\clearpage
\printglossary[type=\acronymtype]
\printglossary

\bibliographystyle{plain} % We choose the "plain" reference style
\bibliography{refs}

\end{document}

\documentclass[11pt,a4paper,openany]{book}

\usepackage{amsmath}
\usepackage{graphicx}
\usepackage{hyperref}
\usepackage[utf8]{inputenc}
\usepackage[acronym, toc]{glossaries}

\usepackage{setspace}	%double spacing for text, single for captions, footnotes, etc.
\usepackage{natbib}		% substituye a 'hypernat' que funciona en Windows.
\usepackage[english]{babel}
\usepackage[utf8]{inputenc}
\usepackage{color}
\usepackage{hhline} 		% extended styles for tables
\usepackage{multirow}
\usepackage{subfigure}
\usepackage{amsmath,amsmath,amssymb} 
\usepackage{fancyhdr}
\usepackage{epsfig, amsmath}
\usepackage{algorithm}
\usepackage{algorithmic}

\usepackage[section]{placeins}

% general settings
\hypersetup{
	linktocpage=true,
	colorlinks=true,
	linkcolor=blue,
	citecolor=blue,
}
\definecolor{Hgray}{gray}{0.6}

\newenvironment{definition}[1][Definition]{\begin{trivlist}
\item[\hskip \labelsep {\bfseries #1}]}{\end{trivlist}}

\setlength{\topmargin}{0cm}
\setlength{\textheight}{23cm}
\setlength{\textwidth}{17cm}
\setlength{\oddsidemargin}{0cm}
\setlength{\evensidemargin}{0cm}
\setlength{\headheight}{1cm}

% indica que las 'sub-sub-sections' sean numeradas y aparezcan en el indice
\setcounter{secnumdepth}{3}
\setcounter{tocdepth}{2}

% settings for code
\renewcommand{\algorithmicrequire}{\textbf{Entrada: }}
\renewcommand{\algorithmicensure}{\textbf{Salida: }}

\makeglossaries
\input{closing/glossary.tex}

\begin{document}

% Portada
\input{0_tittle.tex}


%\tableofcontents

\pagenumbering{roman} 
\setcounter{page}{1} 
\pagestyle{plain}

%%%%%%%%%%%%%
%%% FICHA %%%
%%%%%%%%%%%%%
\chapter*{FICHA DEL TRABAJO FINAL}
\begin{table}[ht]
	\centering{}
	\renewcommand{\arraystretch}{2}
	\begin{tabular}{r | l}
		\hline
		Título del trabajo: & Artificial MRI brain images creation with Variational Autoencoders\\
		\hline
        Nombre del autor: & Miguel Tablado León\\
		\hline
        Nombre del Tutor/a de TF: & Baris Kanber\\
		\hline
        Nombre del/de la PRA: & Ferran Prados Carrasco\\
		\hline
        Fecha de entrega: & 02/2023\\
		\hline
        Titulación o programa: & Máster en Ciencia de Datos\\
		\hline
        Área del Trabajo Final: & Medicine Area (TFM-Med)\\
		\hline
        Language: & English\\
		\hline
        Keywords & Deep Learning, Brain MRI, Variational Autoencoder\\
		\hline
	\end{tabular}
\end{table}

\pagestyle{fancy}
\renewcommand{\chaptermark}[1]{ \markboth{#1}{}}
\renewcommand{\sectionmark}[1]{\markright{ \thesection.\ #1}}
\lhead[\fancyplain{}{\bfseries\thepage}]{\fancyplain{}{\bfseries\rightmark}}
\rhead[\fancyplain{}{\bfseries\leftmark}]{\fancyplain{}{\bfseries\thepage}}
\cfoot{}

% indice
\cleardoublepage
\phantomsection
\addcontentsline{toc}{chapter}{Index}
\tableofcontents
% listado de figuras
\cleardoublepage
\phantomsection
\addcontentsline{toc}{chapter}{Figure List}
\listoffigures
% listado de tablas
\cleardoublepage
\phantomsection
\addcontentsline{toc}{chapter}{Table List}
\listoftables

\thispagestyle{empty}

\pagenumbering{arabic}

\pagestyle{fancy}
\renewcommand{\chaptermark}[1]{ \markboth{#1}{}}
\renewcommand{\sectionmark}[1]{\markright{ \thesection.\ #1}}
\lhead[\fancyplain{}{\bfseries\thepage}]{\fancyplain{}{\bfseries\rightmark}}
\rhead[\fancyplain{}{\bfseries\leftmark}]{\fancyplain{}{\bfseries\thepage}}
\cfoot{}

\onehalfspacing

\input{introduction/0_abstract.tex}

\chapter{Introduction}

\input{introduction/1_context.tex}
\input{introduction/2_aim.tex}
\input{introduction/5_plan.tex}

\input{state_of_art/main.tex}
\input{impl/main.tex}
\input{conclusions/main.tex}

\clearpage
\printglossary[type=\acronymtype]
\printglossary

\bibliographystyle{plain} % We choose the "plain" reference style
\bibliography{refs}

\end{document}

\documentclass[11pt,a4paper,openany]{book}

\usepackage{amsmath}
\usepackage{graphicx}
\usepackage{hyperref}
\usepackage[utf8]{inputenc}
\usepackage[acronym, toc]{glossaries}

\usepackage{setspace}	%double spacing for text, single for captions, footnotes, etc.
\usepackage{natbib}		% substituye a 'hypernat' que funciona en Windows.
\usepackage[english]{babel}
\usepackage[utf8]{inputenc}
\usepackage{color}
\usepackage{hhline} 		% extended styles for tables
\usepackage{multirow}
\usepackage{subfigure}
\usepackage{amsmath,amsmath,amssymb} 
\usepackage{fancyhdr}
\usepackage{epsfig, amsmath}
\usepackage{algorithm}
\usepackage{algorithmic}

\usepackage[section]{placeins}

% general settings
\hypersetup{
	linktocpage=true,
	colorlinks=true,
	linkcolor=blue,
	citecolor=blue,
}
\definecolor{Hgray}{gray}{0.6}

\newenvironment{definition}[1][Definition]{\begin{trivlist}
\item[\hskip \labelsep {\bfseries #1}]}{\end{trivlist}}

\setlength{\topmargin}{0cm}
\setlength{\textheight}{23cm}
\setlength{\textwidth}{17cm}
\setlength{\oddsidemargin}{0cm}
\setlength{\evensidemargin}{0cm}
\setlength{\headheight}{1cm}

% indica que las 'sub-sub-sections' sean numeradas y aparezcan en el indice
\setcounter{secnumdepth}{3}
\setcounter{tocdepth}{2}

% settings for code
\renewcommand{\algorithmicrequire}{\textbf{Entrada: }}
\renewcommand{\algorithmicensure}{\textbf{Salida: }}

\makeglossaries
\input{closing/glossary.tex}

\begin{document}

% Portada
\input{0_tittle.tex}


%\tableofcontents

\pagenumbering{roman} 
\setcounter{page}{1} 
\pagestyle{plain}

%%%%%%%%%%%%%
%%% FICHA %%%
%%%%%%%%%%%%%
\chapter*{FICHA DEL TRABAJO FINAL}
\begin{table}[ht]
	\centering{}
	\renewcommand{\arraystretch}{2}
	\begin{tabular}{r | l}
		\hline
		Título del trabajo: & Artificial MRI brain images creation with Variational Autoencoders\\
		\hline
        Nombre del autor: & Miguel Tablado León\\
		\hline
        Nombre del Tutor/a de TF: & Baris Kanber\\
		\hline
        Nombre del/de la PRA: & Ferran Prados Carrasco\\
		\hline
        Fecha de entrega: & 02/2023\\
		\hline
        Titulación o programa: & Máster en Ciencia de Datos\\
		\hline
        Área del Trabajo Final: & Medicine Area (TFM-Med)\\
		\hline
        Language: & English\\
		\hline
        Keywords & Deep Learning, Brain MRI, Variational Autoencoder\\
		\hline
	\end{tabular}
\end{table}

\pagestyle{fancy}
\renewcommand{\chaptermark}[1]{ \markboth{#1}{}}
\renewcommand{\sectionmark}[1]{\markright{ \thesection.\ #1}}
\lhead[\fancyplain{}{\bfseries\thepage}]{\fancyplain{}{\bfseries\rightmark}}
\rhead[\fancyplain{}{\bfseries\leftmark}]{\fancyplain{}{\bfseries\thepage}}
\cfoot{}

% indice
\cleardoublepage
\phantomsection
\addcontentsline{toc}{chapter}{Index}
\tableofcontents
% listado de figuras
\cleardoublepage
\phantomsection
\addcontentsline{toc}{chapter}{Figure List}
\listoffigures
% listado de tablas
\cleardoublepage
\phantomsection
\addcontentsline{toc}{chapter}{Table List}
\listoftables

\thispagestyle{empty}

\pagenumbering{arabic}

\pagestyle{fancy}
\renewcommand{\chaptermark}[1]{ \markboth{#1}{}}
\renewcommand{\sectionmark}[1]{\markright{ \thesection.\ #1}}
\lhead[\fancyplain{}{\bfseries\thepage}]{\fancyplain{}{\bfseries\rightmark}}
\rhead[\fancyplain{}{\bfseries\leftmark}]{\fancyplain{}{\bfseries\thepage}}
\cfoot{}

\onehalfspacing

\input{introduction/0_abstract.tex}

\chapter{Introduction}

\input{introduction/1_context.tex}
\input{introduction/2_aim.tex}
\input{introduction/5_plan.tex}

\input{state_of_art/main.tex}
\input{impl/main.tex}
\input{conclusions/main.tex}

\clearpage
\printglossary[type=\acronymtype]
\printglossary

\bibliographystyle{plain} % We choose the "plain" reference style
\bibliography{refs}

\end{document}


\clearpage
\printglossary[type=\acronymtype]
\printglossary

\bibliographystyle{plain} % We choose the "plain" reference style
\bibliography{refs}

\end{document}


\clearpage
\printglossary[type=\acronymtype]
\printglossary

\bibliographystyle{plain} % We choose the "plain" reference style
\bibliography{refs}

\end{document}

\documentclass[11pt,a4paper,openany]{book}

\usepackage{amsmath}
\usepackage{graphicx}
\usepackage{hyperref}
\usepackage[utf8]{inputenc}
\usepackage[acronym, toc]{glossaries}

\usepackage{setspace}	%double spacing for text, single for captions, footnotes, etc.
\usepackage{natbib}		% substituye a 'hypernat' que funciona en Windows.
\usepackage[english]{babel}
\usepackage[utf8]{inputenc}
\usepackage{color}
\usepackage{hhline} 		% extended styles for tables
\usepackage{multirow}
\usepackage{subfigure}
\usepackage{amsmath,amsmath,amssymb} 
\usepackage{fancyhdr}
\usepackage{epsfig, amsmath}
\usepackage{algorithm}
\usepackage{algorithmic}

\usepackage[section]{placeins}

% general settings
\hypersetup{
	linktocpage=true,
	colorlinks=true,
	linkcolor=blue,
	citecolor=blue,
}
\definecolor{Hgray}{gray}{0.6}

\newenvironment{definition}[1][Definition]{\begin{trivlist}
\item[\hskip \labelsep {\bfseries #1}]}{\end{trivlist}}

\setlength{\topmargin}{0cm}
\setlength{\textheight}{23cm}
\setlength{\textwidth}{17cm}
\setlength{\oddsidemargin}{0cm}
\setlength{\evensidemargin}{0cm}
\setlength{\headheight}{1cm}

% indica que las 'sub-sub-sections' sean numeradas y aparezcan en el indice
\setcounter{secnumdepth}{3}
\setcounter{tocdepth}{2}

% settings for code
\renewcommand{\algorithmicrequire}{\textbf{Entrada: }}
\renewcommand{\algorithmicensure}{\textbf{Salida: }}

\makeglossaries
\newglossaryentry{kpi}
{
    name=KPI,
    description={Key Performance Indicator}
}

\newglossaryentry{mnist}
{
    name=MNIST,
    description={The MNIST database of handwritten digits used for training models}
}

\newglossaryentry{kdd}
{
    name=KDD,
    description={KDD cup 1999 network intrusion dataset}
}

\newglossaryentry{ixi}
{
    name=IXI,
    description={IXI dataset is a collection of 600 Brain MR Images, in 3-D, collected from 3 different hospitals in London. The images are scanned in 1.5 and 3 Tesla which is the range of the quality that we could usually find in hospitals around the world}
}

\newacronym{pca}{PCA}{Principal Component Analysis}

\newacronym{mri}{MRI}{Magnetic Resonance Image}

\newacronym{vqvae}{VQ-VAE}{Vector-Quantized Variational Autoencoder}

\newacronym{cevae}{ceVAE}{Context-Encoder Variational Autoencoder}

\newacronym{strega}{StRegA}{Segmentation Regularised Anomaly}


\begin{document}

% Portada
\newpage
\thispagestyle{empty}

\baselineskip 2em

%\vspace*{1cm}

\centerline{\includegraphics[width=0.6\textwidth]{images/UOC-logo}}
\begin{center}
\textsc{Universitat Oberta de Catalunya (UOC) \\
 Máster Universitario en Ciencia de Datos (\textit{Data Science})\\}

%\centerline {\pic{UOC}{4cm}}

\vspace*{1.5cm}

\textsc{\Large TRABAJO FINAL DE MÁSTER}

\vspace*{0.5cm}

\textsc{\large Área: YYY}


%\textbf{\Huge VirtualTechLab Model: }

\vspace*{2.0cm}

\title{\Large Artificial MRI brain images creation with Variational Autoencoders}

\vspace{2.5cm}
\baselineskip 1em

\baselineskip 2em
-----------------------------------------------------------------------------\\
Autor:      Miguel Tablado\\
Tutor:      Baris Kanber\\
Profesor:   Nombre del profesor responsable del área de TF\\
-----------------------------------------------------------------------------\\
\vspace*{1.5cm}
Barcelona, \today

\author{Miguel Tablado}

\end{center}



%\tableofcontents

\pagenumbering{roman} 
\setcounter{page}{1} 
\pagestyle{plain}

%%%%%%%%%%%%%
%%% FICHA %%%
%%%%%%%%%%%%%
\chapter*{FICHA DEL TRABAJO FINAL}
\begin{table}[ht]
	\centering{}
	\renewcommand{\arraystretch}{2}
	\begin{tabular}{r | l}
		\hline
		Título del trabajo: & Artificial MRI brain images creation with Variational Autoencoders\\
		\hline
        Nombre del autor: & Miguel Tablado León\\
		\hline
        Nombre del Tutor/a de TF: & Baris Kanber\\
		\hline
        Nombre del/de la PRA: & Ferran Prados Carrasco\\
		\hline
        Fecha de entrega: & 02/2023\\
		\hline
        Titulación o programa: & Máster en Ciencia de Datos\\
		\hline
        Área del Trabajo Final: & Medicine Area (TFM-Med)\\
		\hline
        Language: & English\\
		\hline
        Keywords & Deep Learning, Brain MRI, Variational Autoencoder\\
		\hline
	\end{tabular}
\end{table}

\pagestyle{fancy}
\renewcommand{\chaptermark}[1]{ \markboth{#1}{}}
\renewcommand{\sectionmark}[1]{\markright{ \thesection.\ #1}}
\lhead[\fancyplain{}{\bfseries\thepage}]{\fancyplain{}{\bfseries\rightmark}}
\rhead[\fancyplain{}{\bfseries\leftmark}]{\fancyplain{}{\bfseries\thepage}}
\cfoot{}

% indice
\cleardoublepage
\phantomsection
\addcontentsline{toc}{chapter}{Index}
\tableofcontents
% listado de figuras
\cleardoublepage
\phantomsection
\addcontentsline{toc}{chapter}{Figure List}
\listoffigures
% listado de tablas
\cleardoublepage
\phantomsection
\addcontentsline{toc}{chapter}{Table List}
\listoftables

\thispagestyle{empty}

\pagenumbering{arabic}

\pagestyle{fancy}
\renewcommand{\chaptermark}[1]{ \markboth{#1}{}}
\renewcommand{\sectionmark}[1]{\markright{ \thesection.\ #1}}
\lhead[\fancyplain{}{\bfseries\thepage}]{\fancyplain{}{\bfseries\rightmark}}
\rhead[\fancyplain{}{\bfseries\leftmark}]{\fancyplain{}{\bfseries\thepage}}
\cfoot{}

\onehalfspacing

\input{introduction/0_abstract.tex}

\chapter{Introduction}

\section{Context and project justification}

Artificial Intelligence has arrived to change the world in almost (if not all) any field. Today, we are surrounded by (and we are using many) AI products like smartphones’ face recognition capabilities, home cleaning robots or cars with autopilot options.

From the different AI fields, computer vision is maybe the most popular one and the one which is usually used for explaining AI capabilities to general public. Identifying a cat in a picture could perfectly be the example used in every AI presentation to welcome people to AI.

Image processing is intuitively matched with medical diagnosis by anyone having or not any expertise on the field. Every single citizen will have heard of magnetic resonance imaging, and anyone easily transposes ‘cat detection’ to ‘anomaly detection’, being the anomaly a tumor or anything else.

There are different techniques to scan people and create images for clinical diagnosis like X-Ray or Magnetic Resonance Imaging. In this project, we will work with Magnetic Resonance Images (MRI) which are images created by a machine with a large bore that scans people lying inside it. The MR technique is non-invasive, it produces no radiation, and is used to scan almost any part of the body from which we will focus on brain images.

Combining AI diagnosis capabilities on image processing and MRI images, we can think of helping doctors to identify the presence of anomalies or looking for concrete diagnosis for a specific disease.

Obviously, these projects are not easy at all and they face a lot of challenges. One of the first challenges that such AI project faces is the difficulty to obtain a large set of brain images that are needed to train an accurate AI model. Scanning people is too costly and requires a lot of time. and the challenge of having a large dataset gets harder once we understand that brain images may differ depending on age or gender.

Today, there is a clear limitation on how to reproduce or obtain healthy brain images for AI-based diagnosis projects while the appearance of new AI techniques known as Autoencoders and Variational Autoencoders introduces a new area of investigation to mitigate the gap.

This project aims to be a proof of concept to create AI-created images with new variational autoencoders that would serve to augment any existing MRI dataset and that will help to improve the accuracy of brain anomaly detection projects, including those which could be used for overall anomaly detection to others more specific which would help on concrete disease diagnosis.

Personal motivation comes from various angles:

\begin{itemize}
    \item One is to prove myself that I can work with new AI architectures demonstrating that I have acquired the knowledge needed (deep enough) to be productive and to be able to innovate in the health sector.
    \item Deep Learning has been the subject which I enjoyed the most, hence continuing with Autoencoders seems natural to me as the next step on AI adoption
    \item At no doubts, if I can contribute to help on brain issue detection or diagnosis, I will feel my life been completely fulfilled
\end{itemize}
\section{Aim of the project}

The aim of this project is to serve as a Proof of Concept on how MRI brain images can be artificially generated with Variational Autoencoders which ultimately would serve to enhance existing or new datasets to improve model accuracy by having larger samples of data.

Foreseen projects objectives are:

\begin{itemize}
    \item Obtain basic knowledge about MRI images and the NIFTI file format
    \item Obtain and visualise 2-D images from 3-D images in the dataset
    \item Select what range of slices (2-D images) from brain to be created
    \item Test different existing networks and choose the one to be used
    \item Tune network parameters
    \item Compare generated brain images against real ones, qualitatively
\end{itemize}
\section{Project Plan}

\subsection{Resources}

\begin{enumerate}
    \item 1 Data Scientist: Miguel Tablado will be playing this role and will dedicate 300h
    \item 1 Coach/Tutor: Baris Kanber will be assisting Miguel Tablado during the project
    \item MRI images and demographic information from IXI Dataset
    \item GPU resources are needed to train and test the network
\end{enumerate}

\subsection{High Level Plan}

The plan will be executed in 3 different phases with the listed tasks:
\begin{enumerate}
    \item Phase 1: Analysis
    \begin{enumerate}
        \item Gain knowledge on MRI and NIFTI protocol
        \item Describe images and dataset
        \item Extract 2D images
        \item Pre-processing images
    \end{enumerate}
    \item Phase 2: MR Images creation
    \begin{enumerate}
        \item Test different Network architectures
        \item Tune-up architectural network
    \end{enumerate}
    \item Phase 3: Project documentation
    \begin{enumerate}
        \item Write conclusions
        \item Create project documentation
        \item Create project presentation
    \end{enumerate}
\end{enumerate}

\begin{figure}[ht]
    \hspace*{-1.2in}
    \centering
    \includegraphics[width = 20cm, height = 6cm]{images/project-plan.png}
    \caption[]{Project Plan. source: https://www.onlinegantt.com}
    \label{fig:project-plan}
\end{figure}

\newpage
\subsection{Tasks}
\subsubsection*{Phase 1: Analysis}

During this face the different tasks will be executed to prepare the work and includes:
\begin{enumerate}
    \item Gain knowledge on MRI and the NIFTI file format: This activity consists of reading papers and documents to gain sufficient knowledge to execute the project. There is no need to become an expert on the matter but understanding how those files are and how to process them.
    \item Describe images and dataset: During this activity, a description of the dataset will be generated with a view of the quality of the dataset for the aim of the project and any findings which could result.
    \item Extract 2D images: Load 3D images and extract 2D slices from the original dataset, which will be depicted with code.
    \item Pre-processing images: Decide which transformations on the 2D images would help the project like pixel changes or applying gray-scale transformations.
\end{enumerate}

\subsubsection*{Phase 2: MR Images creation}

\begin{enumerate}
    \item Test different Network architectures: This task will take few existing network architectures and be tested with the dataset so that one of them will be selected to be improved and used as the project architecture.
    \item Tune-up architectural network: Tune the selected architecture with useful techniques like changing network layers
\end{enumerate}

\documentclass[11pt,a4paper,openany]{book}

\usepackage{amsmath}
\usepackage{graphicx}
\usepackage{hyperref}
\usepackage[utf8]{inputenc}
\usepackage[acronym, toc]{glossaries}

\usepackage{setspace}	%double spacing for text, single for captions, footnotes, etc.
\usepackage{natbib}		% substituye a 'hypernat' que funciona en Windows.
\usepackage[english]{babel}
\usepackage[utf8]{inputenc}
\usepackage{color}
\usepackage{hhline} 		% extended styles for tables
\usepackage{multirow}
\usepackage{subfigure}
\usepackage{amsmath,amsmath,amssymb} 
\usepackage{fancyhdr}
\usepackage{epsfig, amsmath}
\usepackage{algorithm}
\usepackage{algorithmic}

\usepackage[section]{placeins}

% general settings
\hypersetup{
	linktocpage=true,
	colorlinks=true,
	linkcolor=blue,
	citecolor=blue,
}
\definecolor{Hgray}{gray}{0.6}

\newenvironment{definition}[1][Definition]{\begin{trivlist}
\item[\hskip \labelsep {\bfseries #1}]}{\end{trivlist}}

\setlength{\topmargin}{0cm}
\setlength{\textheight}{23cm}
\setlength{\textwidth}{17cm}
\setlength{\oddsidemargin}{0cm}
\setlength{\evensidemargin}{0cm}
\setlength{\headheight}{1cm}

% indica que las 'sub-sub-sections' sean numeradas y aparezcan en el indice
\setcounter{secnumdepth}{3}
\setcounter{tocdepth}{2}

% settings for code
\renewcommand{\algorithmicrequire}{\textbf{Entrada: }}
\renewcommand{\algorithmicensure}{\textbf{Salida: }}

\makeglossaries
\newglossaryentry{kpi}
{
    name=KPI,
    description={Key Performance Indicator}
}

\newglossaryentry{mnist}
{
    name=MNIST,
    description={The MNIST database of handwritten digits used for training models}
}

\newglossaryentry{kdd}
{
    name=KDD,
    description={KDD cup 1999 network intrusion dataset}
}

\newglossaryentry{ixi}
{
    name=IXI,
    description={IXI dataset is a collection of 600 Brain MR Images, in 3-D, collected from 3 different hospitals in London. The images are scanned in 1.5 and 3 Tesla which is the range of the quality that we could usually find in hospitals around the world}
}

\newacronym{pca}{PCA}{Principal Component Analysis}

\newacronym{mri}{MRI}{Magnetic Resonance Image}

\newacronym{vqvae}{VQ-VAE}{Vector-Quantized Variational Autoencoder}

\newacronym{cevae}{ceVAE}{Context-Encoder Variational Autoencoder}

\newacronym{strega}{StRegA}{Segmentation Regularised Anomaly}


\begin{document}

% Portada
\newpage
\thispagestyle{empty}

\baselineskip 2em

%\vspace*{1cm}

\centerline{\includegraphics[width=0.6\textwidth]{images/UOC-logo}}
\begin{center}
\textsc{Universitat Oberta de Catalunya (UOC) \\
 Máster Universitario en Ciencia de Datos (\textit{Data Science})\\}

%\centerline {\pic{UOC}{4cm}}

\vspace*{1.5cm}

\textsc{\Large TRABAJO FINAL DE MÁSTER}

\vspace*{0.5cm}

\textsc{\large Área: YYY}


%\textbf{\Huge VirtualTechLab Model: }

\vspace*{2.0cm}

\title{\Large Artificial MRI brain images creation with Variational Autoencoders}

\vspace{2.5cm}
\baselineskip 1em

\baselineskip 2em
-----------------------------------------------------------------------------\\
Autor:      Miguel Tablado\\
Tutor:      Baris Kanber\\
Profesor:   Nombre del profesor responsable del área de TF\\
-----------------------------------------------------------------------------\\
\vspace*{1.5cm}
Barcelona, \today

\author{Miguel Tablado}

\end{center}



%\tableofcontents

\pagenumbering{roman} 
\setcounter{page}{1} 
\pagestyle{plain}

%%%%%%%%%%%%%
%%% FICHA %%%
%%%%%%%%%%%%%
\chapter*{FICHA DEL TRABAJO FINAL}
\begin{table}[ht]
	\centering{}
	\renewcommand{\arraystretch}{2}
	\begin{tabular}{r | l}
		\hline
		Título del trabajo: & Artificial MRI brain images creation with Variational Autoencoders\\
		\hline
        Nombre del autor: & Miguel Tablado León\\
		\hline
        Nombre del Tutor/a de TF: & Baris Kanber\\
		\hline
        Nombre del/de la PRA: & Ferran Prados Carrasco\\
		\hline
        Fecha de entrega: & 02/2023\\
		\hline
        Titulación o programa: & Máster en Ciencia de Datos\\
		\hline
        Área del Trabajo Final: & Medicine Area (TFM-Med)\\
		\hline
        Language: & English\\
		\hline
        Keywords & Deep Learning, Brain MRI, Variational Autoencoder\\
		\hline
	\end{tabular}
\end{table}

\pagestyle{fancy}
\renewcommand{\chaptermark}[1]{ \markboth{#1}{}}
\renewcommand{\sectionmark}[1]{\markright{ \thesection.\ #1}}
\lhead[\fancyplain{}{\bfseries\thepage}]{\fancyplain{}{\bfseries\rightmark}}
\rhead[\fancyplain{}{\bfseries\leftmark}]{\fancyplain{}{\bfseries\thepage}}
\cfoot{}

% indice
\cleardoublepage
\phantomsection
\addcontentsline{toc}{chapter}{Index}
\tableofcontents
% listado de figuras
\cleardoublepage
\phantomsection
\addcontentsline{toc}{chapter}{Figure List}
\listoffigures
% listado de tablas
\cleardoublepage
\phantomsection
\addcontentsline{toc}{chapter}{Table List}
\listoftables

\thispagestyle{empty}

\pagenumbering{arabic}

\pagestyle{fancy}
\renewcommand{\chaptermark}[1]{ \markboth{#1}{}}
\renewcommand{\sectionmark}[1]{\markright{ \thesection.\ #1}}
\lhead[\fancyplain{}{\bfseries\thepage}]{\fancyplain{}{\bfseries\rightmark}}
\rhead[\fancyplain{}{\bfseries\leftmark}]{\fancyplain{}{\bfseries\thepage}}
\cfoot{}

\onehalfspacing

\input{introduction/0_abstract.tex}

\chapter{Introduction}

\section{Context and project justification}

Artificial Intelligence has arrived to change the world in almost (if not all) any field. Today, we are surrounded by (and we are using many) AI products like smartphones’ face recognition capabilities, home cleaning robots or cars with autopilot options.

From the different AI fields, computer vision is maybe the most popular one and the one which is usually used for explaining AI capabilities to general public. Identifying a cat in a picture could perfectly be the example used in every AI presentation to welcome people to AI.

Image processing is intuitively matched with medical diagnosis by anyone having or not any expertise on the field. Every single citizen will have heard of magnetic resonance imaging, and anyone easily transposes ‘cat detection’ to ‘anomaly detection’, being the anomaly a tumor or anything else.

There are different techniques to scan people and create images for clinical diagnosis like X-Ray or Magnetic Resonance Imaging. In this project, we will work with Magnetic Resonance Images (MRI) which are images created by a machine with a large bore that scans people lying inside it. The MR technique is non-invasive, it produces no radiation, and is used to scan almost any part of the body from which we will focus on brain images.

Combining AI diagnosis capabilities on image processing and MRI images, we can think of helping doctors to identify the presence of anomalies or looking for concrete diagnosis for a specific disease.

Obviously, these projects are not easy at all and they face a lot of challenges. One of the first challenges that such AI project faces is the difficulty to obtain a large set of brain images that are needed to train an accurate AI model. Scanning people is too costly and requires a lot of time. and the challenge of having a large dataset gets harder once we understand that brain images may differ depending on age or gender.

Today, there is a clear limitation on how to reproduce or obtain healthy brain images for AI-based diagnosis projects while the appearance of new AI techniques known as Autoencoders and Variational Autoencoders introduces a new area of investigation to mitigate the gap.

This project aims to be a proof of concept to create AI-created images with new variational autoencoders that would serve to augment any existing MRI dataset and that will help to improve the accuracy of brain anomaly detection projects, including those which could be used for overall anomaly detection to others more specific which would help on concrete disease diagnosis.

Personal motivation comes from various angles:

\begin{itemize}
    \item One is to prove myself that I can work with new AI architectures demonstrating that I have acquired the knowledge needed (deep enough) to be productive and to be able to innovate in the health sector.
    \item Deep Learning has been the subject which I enjoyed the most, hence continuing with Autoencoders seems natural to me as the next step on AI adoption
    \item At no doubts, if I can contribute to help on brain issue detection or diagnosis, I will feel my life been completely fulfilled
\end{itemize}
\section{Aim of the project}

The aim of this project is to serve as a Proof of Concept on how MRI brain images can be artificially generated with Variational Autoencoders which ultimately would serve to enhance existing or new datasets to improve model accuracy by having larger samples of data.

Foreseen projects objectives are:

\begin{itemize}
    \item Obtain basic knowledge about MRI images and the NIFTI file format
    \item Obtain and visualise 2-D images from 3-D images in the dataset
    \item Select what range of slices (2-D images) from brain to be created
    \item Test different existing networks and choose the one to be used
    \item Tune network parameters
    \item Compare generated brain images against real ones, qualitatively
\end{itemize}
\section{Project Plan}

\subsection{Resources}

\begin{enumerate}
    \item 1 Data Scientist: Miguel Tablado will be playing this role and will dedicate 300h
    \item 1 Coach/Tutor: Baris Kanber will be assisting Miguel Tablado during the project
    \item MRI images and demographic information from IXI Dataset
    \item GPU resources are needed to train and test the network
\end{enumerate}

\subsection{High Level Plan}

The plan will be executed in 3 different phases with the listed tasks:
\begin{enumerate}
    \item Phase 1: Analysis
    \begin{enumerate}
        \item Gain knowledge on MRI and NIFTI protocol
        \item Describe images and dataset
        \item Extract 2D images
        \item Pre-processing images
    \end{enumerate}
    \item Phase 2: MR Images creation
    \begin{enumerate}
        \item Test different Network architectures
        \item Tune-up architectural network
    \end{enumerate}
    \item Phase 3: Project documentation
    \begin{enumerate}
        \item Write conclusions
        \item Create project documentation
        \item Create project presentation
    \end{enumerate}
\end{enumerate}

\begin{figure}[ht]
    \hspace*{-1.2in}
    \centering
    \includegraphics[width = 20cm, height = 6cm]{images/project-plan.png}
    \caption[]{Project Plan. source: https://www.onlinegantt.com}
    \label{fig:project-plan}
\end{figure}

\newpage
\subsection{Tasks}
\subsubsection*{Phase 1: Analysis}

During this face the different tasks will be executed to prepare the work and includes:
\begin{enumerate}
    \item Gain knowledge on MRI and the NIFTI file format: This activity consists of reading papers and documents to gain sufficient knowledge to execute the project. There is no need to become an expert on the matter but understanding how those files are and how to process them.
    \item Describe images and dataset: During this activity, a description of the dataset will be generated with a view of the quality of the dataset for the aim of the project and any findings which could result.
    \item Extract 2D images: Load 3D images and extract 2D slices from the original dataset, which will be depicted with code.
    \item Pre-processing images: Decide which transformations on the 2D images would help the project like pixel changes or applying gray-scale transformations.
\end{enumerate}

\subsubsection*{Phase 2: MR Images creation}

\begin{enumerate}
    \item Test different Network architectures: This task will take few existing network architectures and be tested with the dataset so that one of them will be selected to be improved and used as the project architecture.
    \item Tune-up architectural network: Tune the selected architecture with useful techniques like changing network layers
\end{enumerate}

\documentclass[11pt,a4paper,openany]{book}

\usepackage{amsmath}
\usepackage{graphicx}
\usepackage{hyperref}
\usepackage[utf8]{inputenc}
\usepackage[acronym, toc]{glossaries}

\usepackage{setspace}	%double spacing for text, single for captions, footnotes, etc.
\usepackage{natbib}		% substituye a 'hypernat' que funciona en Windows.
\usepackage[english]{babel}
\usepackage[utf8]{inputenc}
\usepackage{color}
\usepackage{hhline} 		% extended styles for tables
\usepackage{multirow}
\usepackage{subfigure}
\usepackage{amsmath,amsmath,amssymb} 
\usepackage{fancyhdr}
\usepackage{epsfig, amsmath}
\usepackage{algorithm}
\usepackage{algorithmic}

\usepackage[section]{placeins}

% general settings
\hypersetup{
	linktocpage=true,
	colorlinks=true,
	linkcolor=blue,
	citecolor=blue,
}
\definecolor{Hgray}{gray}{0.6}

\newenvironment{definition}[1][Definition]{\begin{trivlist}
\item[\hskip \labelsep {\bfseries #1}]}{\end{trivlist}}

\setlength{\topmargin}{0cm}
\setlength{\textheight}{23cm}
\setlength{\textwidth}{17cm}
\setlength{\oddsidemargin}{0cm}
\setlength{\evensidemargin}{0cm}
\setlength{\headheight}{1cm}

% indica que las 'sub-sub-sections' sean numeradas y aparezcan en el indice
\setcounter{secnumdepth}{3}
\setcounter{tocdepth}{2}

% settings for code
\renewcommand{\algorithmicrequire}{\textbf{Entrada: }}
\renewcommand{\algorithmicensure}{\textbf{Salida: }}

\makeglossaries
\input{closing/glossary.tex}

\begin{document}

% Portada
\input{0_tittle.tex}


%\tableofcontents

\pagenumbering{roman} 
\setcounter{page}{1} 
\pagestyle{plain}

%%%%%%%%%%%%%
%%% FICHA %%%
%%%%%%%%%%%%%
\chapter*{FICHA DEL TRABAJO FINAL}
\begin{table}[ht]
	\centering{}
	\renewcommand{\arraystretch}{2}
	\begin{tabular}{r | l}
		\hline
		Título del trabajo: & Artificial MRI brain images creation with Variational Autoencoders\\
		\hline
        Nombre del autor: & Miguel Tablado León\\
		\hline
        Nombre del Tutor/a de TF: & Baris Kanber\\
		\hline
        Nombre del/de la PRA: & Ferran Prados Carrasco\\
		\hline
        Fecha de entrega: & 02/2023\\
		\hline
        Titulación o programa: & Máster en Ciencia de Datos\\
		\hline
        Área del Trabajo Final: & Medicine Area (TFM-Med)\\
		\hline
        Language: & English\\
		\hline
        Keywords & Deep Learning, Brain MRI, Variational Autoencoder\\
		\hline
	\end{tabular}
\end{table}

\pagestyle{fancy}
\renewcommand{\chaptermark}[1]{ \markboth{#1}{}}
\renewcommand{\sectionmark}[1]{\markright{ \thesection.\ #1}}
\lhead[\fancyplain{}{\bfseries\thepage}]{\fancyplain{}{\bfseries\rightmark}}
\rhead[\fancyplain{}{\bfseries\leftmark}]{\fancyplain{}{\bfseries\thepage}}
\cfoot{}

% indice
\cleardoublepage
\phantomsection
\addcontentsline{toc}{chapter}{Index}
\tableofcontents
% listado de figuras
\cleardoublepage
\phantomsection
\addcontentsline{toc}{chapter}{Figure List}
\listoffigures
% listado de tablas
\cleardoublepage
\phantomsection
\addcontentsline{toc}{chapter}{Table List}
\listoftables

\thispagestyle{empty}

\pagenumbering{arabic}

\pagestyle{fancy}
\renewcommand{\chaptermark}[1]{ \markboth{#1}{}}
\renewcommand{\sectionmark}[1]{\markright{ \thesection.\ #1}}
\lhead[\fancyplain{}{\bfseries\thepage}]{\fancyplain{}{\bfseries\rightmark}}
\rhead[\fancyplain{}{\bfseries\leftmark}]{\fancyplain{}{\bfseries\thepage}}
\cfoot{}

\onehalfspacing

\input{introduction/0_abstract.tex}

\chapter{Introduction}

\input{introduction/1_context.tex}
\input{introduction/2_aim.tex}
\input{introduction/5_plan.tex}

\input{state_of_art/main.tex}
\input{impl/main.tex}
\input{conclusions/main.tex}

\clearpage
\printglossary[type=\acronymtype]
\printglossary

\bibliographystyle{plain} % We choose the "plain" reference style
\bibliography{refs}

\end{document}

\documentclass[11pt,a4paper,openany]{book}

\usepackage{amsmath}
\usepackage{graphicx}
\usepackage{hyperref}
\usepackage[utf8]{inputenc}
\usepackage[acronym, toc]{glossaries}

\usepackage{setspace}	%double spacing for text, single for captions, footnotes, etc.
\usepackage{natbib}		% substituye a 'hypernat' que funciona en Windows.
\usepackage[english]{babel}
\usepackage[utf8]{inputenc}
\usepackage{color}
\usepackage{hhline} 		% extended styles for tables
\usepackage{multirow}
\usepackage{subfigure}
\usepackage{amsmath,amsmath,amssymb} 
\usepackage{fancyhdr}
\usepackage{epsfig, amsmath}
\usepackage{algorithm}
\usepackage{algorithmic}

\usepackage[section]{placeins}

% general settings
\hypersetup{
	linktocpage=true,
	colorlinks=true,
	linkcolor=blue,
	citecolor=blue,
}
\definecolor{Hgray}{gray}{0.6}

\newenvironment{definition}[1][Definition]{\begin{trivlist}
\item[\hskip \labelsep {\bfseries #1}]}{\end{trivlist}}

\setlength{\topmargin}{0cm}
\setlength{\textheight}{23cm}
\setlength{\textwidth}{17cm}
\setlength{\oddsidemargin}{0cm}
\setlength{\evensidemargin}{0cm}
\setlength{\headheight}{1cm}

% indica que las 'sub-sub-sections' sean numeradas y aparezcan en el indice
\setcounter{secnumdepth}{3}
\setcounter{tocdepth}{2}

% settings for code
\renewcommand{\algorithmicrequire}{\textbf{Entrada: }}
\renewcommand{\algorithmicensure}{\textbf{Salida: }}

\makeglossaries
\input{closing/glossary.tex}

\begin{document}

% Portada
\input{0_tittle.tex}


%\tableofcontents

\pagenumbering{roman} 
\setcounter{page}{1} 
\pagestyle{plain}

%%%%%%%%%%%%%
%%% FICHA %%%
%%%%%%%%%%%%%
\chapter*{FICHA DEL TRABAJO FINAL}
\begin{table}[ht]
	\centering{}
	\renewcommand{\arraystretch}{2}
	\begin{tabular}{r | l}
		\hline
		Título del trabajo: & Artificial MRI brain images creation with Variational Autoencoders\\
		\hline
        Nombre del autor: & Miguel Tablado León\\
		\hline
        Nombre del Tutor/a de TF: & Baris Kanber\\
		\hline
        Nombre del/de la PRA: & Ferran Prados Carrasco\\
		\hline
        Fecha de entrega: & 02/2023\\
		\hline
        Titulación o programa: & Máster en Ciencia de Datos\\
		\hline
        Área del Trabajo Final: & Medicine Area (TFM-Med)\\
		\hline
        Language: & English\\
		\hline
        Keywords & Deep Learning, Brain MRI, Variational Autoencoder\\
		\hline
	\end{tabular}
\end{table}

\pagestyle{fancy}
\renewcommand{\chaptermark}[1]{ \markboth{#1}{}}
\renewcommand{\sectionmark}[1]{\markright{ \thesection.\ #1}}
\lhead[\fancyplain{}{\bfseries\thepage}]{\fancyplain{}{\bfseries\rightmark}}
\rhead[\fancyplain{}{\bfseries\leftmark}]{\fancyplain{}{\bfseries\thepage}}
\cfoot{}

% indice
\cleardoublepage
\phantomsection
\addcontentsline{toc}{chapter}{Index}
\tableofcontents
% listado de figuras
\cleardoublepage
\phantomsection
\addcontentsline{toc}{chapter}{Figure List}
\listoffigures
% listado de tablas
\cleardoublepage
\phantomsection
\addcontentsline{toc}{chapter}{Table List}
\listoftables

\thispagestyle{empty}

\pagenumbering{arabic}

\pagestyle{fancy}
\renewcommand{\chaptermark}[1]{ \markboth{#1}{}}
\renewcommand{\sectionmark}[1]{\markright{ \thesection.\ #1}}
\lhead[\fancyplain{}{\bfseries\thepage}]{\fancyplain{}{\bfseries\rightmark}}
\rhead[\fancyplain{}{\bfseries\leftmark}]{\fancyplain{}{\bfseries\thepage}}
\cfoot{}

\onehalfspacing

\input{introduction/0_abstract.tex}

\chapter{Introduction}

\input{introduction/1_context.tex}
\input{introduction/2_aim.tex}
\input{introduction/5_plan.tex}

\input{state_of_art/main.tex}
\input{impl/main.tex}
\input{conclusions/main.tex}

\clearpage
\printglossary[type=\acronymtype]
\printglossary

\bibliographystyle{plain} % We choose the "plain" reference style
\bibliography{refs}

\end{document}

\documentclass[11pt,a4paper,openany]{book}

\usepackage{amsmath}
\usepackage{graphicx}
\usepackage{hyperref}
\usepackage[utf8]{inputenc}
\usepackage[acronym, toc]{glossaries}

\usepackage{setspace}	%double spacing for text, single for captions, footnotes, etc.
\usepackage{natbib}		% substituye a 'hypernat' que funciona en Windows.
\usepackage[english]{babel}
\usepackage[utf8]{inputenc}
\usepackage{color}
\usepackage{hhline} 		% extended styles for tables
\usepackage{multirow}
\usepackage{subfigure}
\usepackage{amsmath,amsmath,amssymb} 
\usepackage{fancyhdr}
\usepackage{epsfig, amsmath}
\usepackage{algorithm}
\usepackage{algorithmic}

\usepackage[section]{placeins}

% general settings
\hypersetup{
	linktocpage=true,
	colorlinks=true,
	linkcolor=blue,
	citecolor=blue,
}
\definecolor{Hgray}{gray}{0.6}

\newenvironment{definition}[1][Definition]{\begin{trivlist}
\item[\hskip \labelsep {\bfseries #1}]}{\end{trivlist}}

\setlength{\topmargin}{0cm}
\setlength{\textheight}{23cm}
\setlength{\textwidth}{17cm}
\setlength{\oddsidemargin}{0cm}
\setlength{\evensidemargin}{0cm}
\setlength{\headheight}{1cm}

% indica que las 'sub-sub-sections' sean numeradas y aparezcan en el indice
\setcounter{secnumdepth}{3}
\setcounter{tocdepth}{2}

% settings for code
\renewcommand{\algorithmicrequire}{\textbf{Entrada: }}
\renewcommand{\algorithmicensure}{\textbf{Salida: }}

\makeglossaries
\input{closing/glossary.tex}

\begin{document}

% Portada
\input{0_tittle.tex}


%\tableofcontents

\pagenumbering{roman} 
\setcounter{page}{1} 
\pagestyle{plain}

%%%%%%%%%%%%%
%%% FICHA %%%
%%%%%%%%%%%%%
\chapter*{FICHA DEL TRABAJO FINAL}
\begin{table}[ht]
	\centering{}
	\renewcommand{\arraystretch}{2}
	\begin{tabular}{r | l}
		\hline
		Título del trabajo: & Artificial MRI brain images creation with Variational Autoencoders\\
		\hline
        Nombre del autor: & Miguel Tablado León\\
		\hline
        Nombre del Tutor/a de TF: & Baris Kanber\\
		\hline
        Nombre del/de la PRA: & Ferran Prados Carrasco\\
		\hline
        Fecha de entrega: & 02/2023\\
		\hline
        Titulación o programa: & Máster en Ciencia de Datos\\
		\hline
        Área del Trabajo Final: & Medicine Area (TFM-Med)\\
		\hline
        Language: & English\\
		\hline
        Keywords & Deep Learning, Brain MRI, Variational Autoencoder\\
		\hline
	\end{tabular}
\end{table}

\pagestyle{fancy}
\renewcommand{\chaptermark}[1]{ \markboth{#1}{}}
\renewcommand{\sectionmark}[1]{\markright{ \thesection.\ #1}}
\lhead[\fancyplain{}{\bfseries\thepage}]{\fancyplain{}{\bfseries\rightmark}}
\rhead[\fancyplain{}{\bfseries\leftmark}]{\fancyplain{}{\bfseries\thepage}}
\cfoot{}

% indice
\cleardoublepage
\phantomsection
\addcontentsline{toc}{chapter}{Index}
\tableofcontents
% listado de figuras
\cleardoublepage
\phantomsection
\addcontentsline{toc}{chapter}{Figure List}
\listoffigures
% listado de tablas
\cleardoublepage
\phantomsection
\addcontentsline{toc}{chapter}{Table List}
\listoftables

\thispagestyle{empty}

\pagenumbering{arabic}

\pagestyle{fancy}
\renewcommand{\chaptermark}[1]{ \markboth{#1}{}}
\renewcommand{\sectionmark}[1]{\markright{ \thesection.\ #1}}
\lhead[\fancyplain{}{\bfseries\thepage}]{\fancyplain{}{\bfseries\rightmark}}
\rhead[\fancyplain{}{\bfseries\leftmark}]{\fancyplain{}{\bfseries\thepage}}
\cfoot{}

\onehalfspacing

\input{introduction/0_abstract.tex}

\chapter{Introduction}

\input{introduction/1_context.tex}
\input{introduction/2_aim.tex}
\input{introduction/5_plan.tex}

\input{state_of_art/main.tex}
\input{impl/main.tex}
\input{conclusions/main.tex}

\clearpage
\printglossary[type=\acronymtype]
\printglossary

\bibliographystyle{plain} % We choose the "plain" reference style
\bibliography{refs}

\end{document}


\clearpage
\printglossary[type=\acronymtype]
\printglossary

\bibliographystyle{plain} % We choose the "plain" reference style
\bibliography{refs}

\end{document}

\documentclass[11pt,a4paper,openany]{book}

\usepackage{amsmath}
\usepackage{graphicx}
\usepackage{hyperref}
\usepackage[utf8]{inputenc}
\usepackage[acronym, toc]{glossaries}

\usepackage{setspace}	%double spacing for text, single for captions, footnotes, etc.
\usepackage{natbib}		% substituye a 'hypernat' que funciona en Windows.
\usepackage[english]{babel}
\usepackage[utf8]{inputenc}
\usepackage{color}
\usepackage{hhline} 		% extended styles for tables
\usepackage{multirow}
\usepackage{subfigure}
\usepackage{amsmath,amsmath,amssymb} 
\usepackage{fancyhdr}
\usepackage{epsfig, amsmath}
\usepackage{algorithm}
\usepackage{algorithmic}

\usepackage[section]{placeins}

% general settings
\hypersetup{
	linktocpage=true,
	colorlinks=true,
	linkcolor=blue,
	citecolor=blue,
}
\definecolor{Hgray}{gray}{0.6}

\newenvironment{definition}[1][Definition]{\begin{trivlist}
\item[\hskip \labelsep {\bfseries #1}]}{\end{trivlist}}

\setlength{\topmargin}{0cm}
\setlength{\textheight}{23cm}
\setlength{\textwidth}{17cm}
\setlength{\oddsidemargin}{0cm}
\setlength{\evensidemargin}{0cm}
\setlength{\headheight}{1cm}

% indica que las 'sub-sub-sections' sean numeradas y aparezcan en el indice
\setcounter{secnumdepth}{3}
\setcounter{tocdepth}{2}

% settings for code
\renewcommand{\algorithmicrequire}{\textbf{Entrada: }}
\renewcommand{\algorithmicensure}{\textbf{Salida: }}

\makeglossaries
\newglossaryentry{kpi}
{
    name=KPI,
    description={Key Performance Indicator}
}

\newglossaryentry{mnist}
{
    name=MNIST,
    description={The MNIST database of handwritten digits used for training models}
}

\newglossaryentry{kdd}
{
    name=KDD,
    description={KDD cup 1999 network intrusion dataset}
}

\newglossaryentry{ixi}
{
    name=IXI,
    description={IXI dataset is a collection of 600 Brain MR Images, in 3-D, collected from 3 different hospitals in London. The images are scanned in 1.5 and 3 Tesla which is the range of the quality that we could usually find in hospitals around the world}
}

\newacronym{pca}{PCA}{Principal Component Analysis}

\newacronym{mri}{MRI}{Magnetic Resonance Image}

\newacronym{vqvae}{VQ-VAE}{Vector-Quantized Variational Autoencoder}

\newacronym{cevae}{ceVAE}{Context-Encoder Variational Autoencoder}

\newacronym{strega}{StRegA}{Segmentation Regularised Anomaly}


\begin{document}

% Portada
\newpage
\thispagestyle{empty}

\baselineskip 2em

%\vspace*{1cm}

\centerline{\includegraphics[width=0.6\textwidth]{images/UOC-logo}}
\begin{center}
\textsc{Universitat Oberta de Catalunya (UOC) \\
 Máster Universitario en Ciencia de Datos (\textit{Data Science})\\}

%\centerline {\pic{UOC}{4cm}}

\vspace*{1.5cm}

\textsc{\Large TRABAJO FINAL DE MÁSTER}

\vspace*{0.5cm}

\textsc{\large Área: YYY}


%\textbf{\Huge VirtualTechLab Model: }

\vspace*{2.0cm}

\title{\Large Artificial MRI brain images creation with Variational Autoencoders}

\vspace{2.5cm}
\baselineskip 1em

\baselineskip 2em
-----------------------------------------------------------------------------\\
Autor:      Miguel Tablado\\
Tutor:      Baris Kanber\\
Profesor:   Nombre del profesor responsable del área de TF\\
-----------------------------------------------------------------------------\\
\vspace*{1.5cm}
Barcelona, \today

\author{Miguel Tablado}

\end{center}



%\tableofcontents

\pagenumbering{roman} 
\setcounter{page}{1} 
\pagestyle{plain}

%%%%%%%%%%%%%
%%% FICHA %%%
%%%%%%%%%%%%%
\chapter*{FICHA DEL TRABAJO FINAL}
\begin{table}[ht]
	\centering{}
	\renewcommand{\arraystretch}{2}
	\begin{tabular}{r | l}
		\hline
		Título del trabajo: & Artificial MRI brain images creation with Variational Autoencoders\\
		\hline
        Nombre del autor: & Miguel Tablado León\\
		\hline
        Nombre del Tutor/a de TF: & Baris Kanber\\
		\hline
        Nombre del/de la PRA: & Ferran Prados Carrasco\\
		\hline
        Fecha de entrega: & 02/2023\\
		\hline
        Titulación o programa: & Máster en Ciencia de Datos\\
		\hline
        Área del Trabajo Final: & Medicine Area (TFM-Med)\\
		\hline
        Language: & English\\
		\hline
        Keywords & Deep Learning, Brain MRI, Variational Autoencoder\\
		\hline
	\end{tabular}
\end{table}

\pagestyle{fancy}
\renewcommand{\chaptermark}[1]{ \markboth{#1}{}}
\renewcommand{\sectionmark}[1]{\markright{ \thesection.\ #1}}
\lhead[\fancyplain{}{\bfseries\thepage}]{\fancyplain{}{\bfseries\rightmark}}
\rhead[\fancyplain{}{\bfseries\leftmark}]{\fancyplain{}{\bfseries\thepage}}
\cfoot{}

% indice
\cleardoublepage
\phantomsection
\addcontentsline{toc}{chapter}{Index}
\tableofcontents
% listado de figuras
\cleardoublepage
\phantomsection
\addcontentsline{toc}{chapter}{Figure List}
\listoffigures
% listado de tablas
\cleardoublepage
\phantomsection
\addcontentsline{toc}{chapter}{Table List}
\listoftables

\thispagestyle{empty}

\pagenumbering{arabic}

\pagestyle{fancy}
\renewcommand{\chaptermark}[1]{ \markboth{#1}{}}
\renewcommand{\sectionmark}[1]{\markright{ \thesection.\ #1}}
\lhead[\fancyplain{}{\bfseries\thepage}]{\fancyplain{}{\bfseries\rightmark}}
\rhead[\fancyplain{}{\bfseries\leftmark}]{\fancyplain{}{\bfseries\thepage}}
\cfoot{}

\onehalfspacing

\input{introduction/0_abstract.tex}

\chapter{Introduction}

\section{Context and project justification}

Artificial Intelligence has arrived to change the world in almost (if not all) any field. Today, we are surrounded by (and we are using many) AI products like smartphones’ face recognition capabilities, home cleaning robots or cars with autopilot options.

From the different AI fields, computer vision is maybe the most popular one and the one which is usually used for explaining AI capabilities to general public. Identifying a cat in a picture could perfectly be the example used in every AI presentation to welcome people to AI.

Image processing is intuitively matched with medical diagnosis by anyone having or not any expertise on the field. Every single citizen will have heard of magnetic resonance imaging, and anyone easily transposes ‘cat detection’ to ‘anomaly detection’, being the anomaly a tumor or anything else.

There are different techniques to scan people and create images for clinical diagnosis like X-Ray or Magnetic Resonance Imaging. In this project, we will work with Magnetic Resonance Images (MRI) which are images created by a machine with a large bore that scans people lying inside it. The MR technique is non-invasive, it produces no radiation, and is used to scan almost any part of the body from which we will focus on brain images.

Combining AI diagnosis capabilities on image processing and MRI images, we can think of helping doctors to identify the presence of anomalies or looking for concrete diagnosis for a specific disease.

Obviously, these projects are not easy at all and they face a lot of challenges. One of the first challenges that such AI project faces is the difficulty to obtain a large set of brain images that are needed to train an accurate AI model. Scanning people is too costly and requires a lot of time. and the challenge of having a large dataset gets harder once we understand that brain images may differ depending on age or gender.

Today, there is a clear limitation on how to reproduce or obtain healthy brain images for AI-based diagnosis projects while the appearance of new AI techniques known as Autoencoders and Variational Autoencoders introduces a new area of investigation to mitigate the gap.

This project aims to be a proof of concept to create AI-created images with new variational autoencoders that would serve to augment any existing MRI dataset and that will help to improve the accuracy of brain anomaly detection projects, including those which could be used for overall anomaly detection to others more specific which would help on concrete disease diagnosis.

Personal motivation comes from various angles:

\begin{itemize}
    \item One is to prove myself that I can work with new AI architectures demonstrating that I have acquired the knowledge needed (deep enough) to be productive and to be able to innovate in the health sector.
    \item Deep Learning has been the subject which I enjoyed the most, hence continuing with Autoencoders seems natural to me as the next step on AI adoption
    \item At no doubts, if I can contribute to help on brain issue detection or diagnosis, I will feel my life been completely fulfilled
\end{itemize}
\section{Aim of the project}

The aim of this project is to serve as a Proof of Concept on how MRI brain images can be artificially generated with Variational Autoencoders which ultimately would serve to enhance existing or new datasets to improve model accuracy by having larger samples of data.

Foreseen projects objectives are:

\begin{itemize}
    \item Obtain basic knowledge about MRI images and the NIFTI file format
    \item Obtain and visualise 2-D images from 3-D images in the dataset
    \item Select what range of slices (2-D images) from brain to be created
    \item Test different existing networks and choose the one to be used
    \item Tune network parameters
    \item Compare generated brain images against real ones, qualitatively
\end{itemize}
\section{Project Plan}

\subsection{Resources}

\begin{enumerate}
    \item 1 Data Scientist: Miguel Tablado will be playing this role and will dedicate 300h
    \item 1 Coach/Tutor: Baris Kanber will be assisting Miguel Tablado during the project
    \item MRI images and demographic information from IXI Dataset
    \item GPU resources are needed to train and test the network
\end{enumerate}

\subsection{High Level Plan}

The plan will be executed in 3 different phases with the listed tasks:
\begin{enumerate}
    \item Phase 1: Analysis
    \begin{enumerate}
        \item Gain knowledge on MRI and NIFTI protocol
        \item Describe images and dataset
        \item Extract 2D images
        \item Pre-processing images
    \end{enumerate}
    \item Phase 2: MR Images creation
    \begin{enumerate}
        \item Test different Network architectures
        \item Tune-up architectural network
    \end{enumerate}
    \item Phase 3: Project documentation
    \begin{enumerate}
        \item Write conclusions
        \item Create project documentation
        \item Create project presentation
    \end{enumerate}
\end{enumerate}

\begin{figure}[ht]
    \hspace*{-1.2in}
    \centering
    \includegraphics[width = 20cm, height = 6cm]{images/project-plan.png}
    \caption[]{Project Plan. source: https://www.onlinegantt.com}
    \label{fig:project-plan}
\end{figure}

\newpage
\subsection{Tasks}
\subsubsection*{Phase 1: Analysis}

During this face the different tasks will be executed to prepare the work and includes:
\begin{enumerate}
    \item Gain knowledge on MRI and the NIFTI file format: This activity consists of reading papers and documents to gain sufficient knowledge to execute the project. There is no need to become an expert on the matter but understanding how those files are and how to process them.
    \item Describe images and dataset: During this activity, a description of the dataset will be generated with a view of the quality of the dataset for the aim of the project and any findings which could result.
    \item Extract 2D images: Load 3D images and extract 2D slices from the original dataset, which will be depicted with code.
    \item Pre-processing images: Decide which transformations on the 2D images would help the project like pixel changes or applying gray-scale transformations.
\end{enumerate}

\subsubsection*{Phase 2: MR Images creation}

\begin{enumerate}
    \item Test different Network architectures: This task will take few existing network architectures and be tested with the dataset so that one of them will be selected to be improved and used as the project architecture.
    \item Tune-up architectural network: Tune the selected architecture with useful techniques like changing network layers
\end{enumerate}

\documentclass[11pt,a4paper,openany]{book}

\usepackage{amsmath}
\usepackage{graphicx}
\usepackage{hyperref}
\usepackage[utf8]{inputenc}
\usepackage[acronym, toc]{glossaries}

\usepackage{setspace}	%double spacing for text, single for captions, footnotes, etc.
\usepackage{natbib}		% substituye a 'hypernat' que funciona en Windows.
\usepackage[english]{babel}
\usepackage[utf8]{inputenc}
\usepackage{color}
\usepackage{hhline} 		% extended styles for tables
\usepackage{multirow}
\usepackage{subfigure}
\usepackage{amsmath,amsmath,amssymb} 
\usepackage{fancyhdr}
\usepackage{epsfig, amsmath}
\usepackage{algorithm}
\usepackage{algorithmic}

\usepackage[section]{placeins}

% general settings
\hypersetup{
	linktocpage=true,
	colorlinks=true,
	linkcolor=blue,
	citecolor=blue,
}
\definecolor{Hgray}{gray}{0.6}

\newenvironment{definition}[1][Definition]{\begin{trivlist}
\item[\hskip \labelsep {\bfseries #1}]}{\end{trivlist}}

\setlength{\topmargin}{0cm}
\setlength{\textheight}{23cm}
\setlength{\textwidth}{17cm}
\setlength{\oddsidemargin}{0cm}
\setlength{\evensidemargin}{0cm}
\setlength{\headheight}{1cm}

% indica que las 'sub-sub-sections' sean numeradas y aparezcan en el indice
\setcounter{secnumdepth}{3}
\setcounter{tocdepth}{2}

% settings for code
\renewcommand{\algorithmicrequire}{\textbf{Entrada: }}
\renewcommand{\algorithmicensure}{\textbf{Salida: }}

\makeglossaries
\input{closing/glossary.tex}

\begin{document}

% Portada
\input{0_tittle.tex}


%\tableofcontents

\pagenumbering{roman} 
\setcounter{page}{1} 
\pagestyle{plain}

%%%%%%%%%%%%%
%%% FICHA %%%
%%%%%%%%%%%%%
\chapter*{FICHA DEL TRABAJO FINAL}
\begin{table}[ht]
	\centering{}
	\renewcommand{\arraystretch}{2}
	\begin{tabular}{r | l}
		\hline
		Título del trabajo: & Artificial MRI brain images creation with Variational Autoencoders\\
		\hline
        Nombre del autor: & Miguel Tablado León\\
		\hline
        Nombre del Tutor/a de TF: & Baris Kanber\\
		\hline
        Nombre del/de la PRA: & Ferran Prados Carrasco\\
		\hline
        Fecha de entrega: & 02/2023\\
		\hline
        Titulación o programa: & Máster en Ciencia de Datos\\
		\hline
        Área del Trabajo Final: & Medicine Area (TFM-Med)\\
		\hline
        Language: & English\\
		\hline
        Keywords & Deep Learning, Brain MRI, Variational Autoencoder\\
		\hline
	\end{tabular}
\end{table}

\pagestyle{fancy}
\renewcommand{\chaptermark}[1]{ \markboth{#1}{}}
\renewcommand{\sectionmark}[1]{\markright{ \thesection.\ #1}}
\lhead[\fancyplain{}{\bfseries\thepage}]{\fancyplain{}{\bfseries\rightmark}}
\rhead[\fancyplain{}{\bfseries\leftmark}]{\fancyplain{}{\bfseries\thepage}}
\cfoot{}

% indice
\cleardoublepage
\phantomsection
\addcontentsline{toc}{chapter}{Index}
\tableofcontents
% listado de figuras
\cleardoublepage
\phantomsection
\addcontentsline{toc}{chapter}{Figure List}
\listoffigures
% listado de tablas
\cleardoublepage
\phantomsection
\addcontentsline{toc}{chapter}{Table List}
\listoftables

\thispagestyle{empty}

\pagenumbering{arabic}

\pagestyle{fancy}
\renewcommand{\chaptermark}[1]{ \markboth{#1}{}}
\renewcommand{\sectionmark}[1]{\markright{ \thesection.\ #1}}
\lhead[\fancyplain{}{\bfseries\thepage}]{\fancyplain{}{\bfseries\rightmark}}
\rhead[\fancyplain{}{\bfseries\leftmark}]{\fancyplain{}{\bfseries\thepage}}
\cfoot{}

\onehalfspacing

\input{introduction/0_abstract.tex}

\chapter{Introduction}

\input{introduction/1_context.tex}
\input{introduction/2_aim.tex}
\input{introduction/5_plan.tex}

\input{state_of_art/main.tex}
\input{impl/main.tex}
\input{conclusions/main.tex}

\clearpage
\printglossary[type=\acronymtype]
\printglossary

\bibliographystyle{plain} % We choose the "plain" reference style
\bibliography{refs}

\end{document}

\documentclass[11pt,a4paper,openany]{book}

\usepackage{amsmath}
\usepackage{graphicx}
\usepackage{hyperref}
\usepackage[utf8]{inputenc}
\usepackage[acronym, toc]{glossaries}

\usepackage{setspace}	%double spacing for text, single for captions, footnotes, etc.
\usepackage{natbib}		% substituye a 'hypernat' que funciona en Windows.
\usepackage[english]{babel}
\usepackage[utf8]{inputenc}
\usepackage{color}
\usepackage{hhline} 		% extended styles for tables
\usepackage{multirow}
\usepackage{subfigure}
\usepackage{amsmath,amsmath,amssymb} 
\usepackage{fancyhdr}
\usepackage{epsfig, amsmath}
\usepackage{algorithm}
\usepackage{algorithmic}

\usepackage[section]{placeins}

% general settings
\hypersetup{
	linktocpage=true,
	colorlinks=true,
	linkcolor=blue,
	citecolor=blue,
}
\definecolor{Hgray}{gray}{0.6}

\newenvironment{definition}[1][Definition]{\begin{trivlist}
\item[\hskip \labelsep {\bfseries #1}]}{\end{trivlist}}

\setlength{\topmargin}{0cm}
\setlength{\textheight}{23cm}
\setlength{\textwidth}{17cm}
\setlength{\oddsidemargin}{0cm}
\setlength{\evensidemargin}{0cm}
\setlength{\headheight}{1cm}

% indica que las 'sub-sub-sections' sean numeradas y aparezcan en el indice
\setcounter{secnumdepth}{3}
\setcounter{tocdepth}{2}

% settings for code
\renewcommand{\algorithmicrequire}{\textbf{Entrada: }}
\renewcommand{\algorithmicensure}{\textbf{Salida: }}

\makeglossaries
\input{closing/glossary.tex}

\begin{document}

% Portada
\input{0_tittle.tex}


%\tableofcontents

\pagenumbering{roman} 
\setcounter{page}{1} 
\pagestyle{plain}

%%%%%%%%%%%%%
%%% FICHA %%%
%%%%%%%%%%%%%
\chapter*{FICHA DEL TRABAJO FINAL}
\begin{table}[ht]
	\centering{}
	\renewcommand{\arraystretch}{2}
	\begin{tabular}{r | l}
		\hline
		Título del trabajo: & Artificial MRI brain images creation with Variational Autoencoders\\
		\hline
        Nombre del autor: & Miguel Tablado León\\
		\hline
        Nombre del Tutor/a de TF: & Baris Kanber\\
		\hline
        Nombre del/de la PRA: & Ferran Prados Carrasco\\
		\hline
        Fecha de entrega: & 02/2023\\
		\hline
        Titulación o programa: & Máster en Ciencia de Datos\\
		\hline
        Área del Trabajo Final: & Medicine Area (TFM-Med)\\
		\hline
        Language: & English\\
		\hline
        Keywords & Deep Learning, Brain MRI, Variational Autoencoder\\
		\hline
	\end{tabular}
\end{table}

\pagestyle{fancy}
\renewcommand{\chaptermark}[1]{ \markboth{#1}{}}
\renewcommand{\sectionmark}[1]{\markright{ \thesection.\ #1}}
\lhead[\fancyplain{}{\bfseries\thepage}]{\fancyplain{}{\bfseries\rightmark}}
\rhead[\fancyplain{}{\bfseries\leftmark}]{\fancyplain{}{\bfseries\thepage}}
\cfoot{}

% indice
\cleardoublepage
\phantomsection
\addcontentsline{toc}{chapter}{Index}
\tableofcontents
% listado de figuras
\cleardoublepage
\phantomsection
\addcontentsline{toc}{chapter}{Figure List}
\listoffigures
% listado de tablas
\cleardoublepage
\phantomsection
\addcontentsline{toc}{chapter}{Table List}
\listoftables

\thispagestyle{empty}

\pagenumbering{arabic}

\pagestyle{fancy}
\renewcommand{\chaptermark}[1]{ \markboth{#1}{}}
\renewcommand{\sectionmark}[1]{\markright{ \thesection.\ #1}}
\lhead[\fancyplain{}{\bfseries\thepage}]{\fancyplain{}{\bfseries\rightmark}}
\rhead[\fancyplain{}{\bfseries\leftmark}]{\fancyplain{}{\bfseries\thepage}}
\cfoot{}

\onehalfspacing

\input{introduction/0_abstract.tex}

\chapter{Introduction}

\input{introduction/1_context.tex}
\input{introduction/2_aim.tex}
\input{introduction/5_plan.tex}

\input{state_of_art/main.tex}
\input{impl/main.tex}
\input{conclusions/main.tex}

\clearpage
\printglossary[type=\acronymtype]
\printglossary

\bibliographystyle{plain} % We choose the "plain" reference style
\bibliography{refs}

\end{document}

\documentclass[11pt,a4paper,openany]{book}

\usepackage{amsmath}
\usepackage{graphicx}
\usepackage{hyperref}
\usepackage[utf8]{inputenc}
\usepackage[acronym, toc]{glossaries}

\usepackage{setspace}	%double spacing for text, single for captions, footnotes, etc.
\usepackage{natbib}		% substituye a 'hypernat' que funciona en Windows.
\usepackage[english]{babel}
\usepackage[utf8]{inputenc}
\usepackage{color}
\usepackage{hhline} 		% extended styles for tables
\usepackage{multirow}
\usepackage{subfigure}
\usepackage{amsmath,amsmath,amssymb} 
\usepackage{fancyhdr}
\usepackage{epsfig, amsmath}
\usepackage{algorithm}
\usepackage{algorithmic}

\usepackage[section]{placeins}

% general settings
\hypersetup{
	linktocpage=true,
	colorlinks=true,
	linkcolor=blue,
	citecolor=blue,
}
\definecolor{Hgray}{gray}{0.6}

\newenvironment{definition}[1][Definition]{\begin{trivlist}
\item[\hskip \labelsep {\bfseries #1}]}{\end{trivlist}}

\setlength{\topmargin}{0cm}
\setlength{\textheight}{23cm}
\setlength{\textwidth}{17cm}
\setlength{\oddsidemargin}{0cm}
\setlength{\evensidemargin}{0cm}
\setlength{\headheight}{1cm}

% indica que las 'sub-sub-sections' sean numeradas y aparezcan en el indice
\setcounter{secnumdepth}{3}
\setcounter{tocdepth}{2}

% settings for code
\renewcommand{\algorithmicrequire}{\textbf{Entrada: }}
\renewcommand{\algorithmicensure}{\textbf{Salida: }}

\makeglossaries
\input{closing/glossary.tex}

\begin{document}

% Portada
\input{0_tittle.tex}


%\tableofcontents

\pagenumbering{roman} 
\setcounter{page}{1} 
\pagestyle{plain}

%%%%%%%%%%%%%
%%% FICHA %%%
%%%%%%%%%%%%%
\chapter*{FICHA DEL TRABAJO FINAL}
\begin{table}[ht]
	\centering{}
	\renewcommand{\arraystretch}{2}
	\begin{tabular}{r | l}
		\hline
		Título del trabajo: & Artificial MRI brain images creation with Variational Autoencoders\\
		\hline
        Nombre del autor: & Miguel Tablado León\\
		\hline
        Nombre del Tutor/a de TF: & Baris Kanber\\
		\hline
        Nombre del/de la PRA: & Ferran Prados Carrasco\\
		\hline
        Fecha de entrega: & 02/2023\\
		\hline
        Titulación o programa: & Máster en Ciencia de Datos\\
		\hline
        Área del Trabajo Final: & Medicine Area (TFM-Med)\\
		\hline
        Language: & English\\
		\hline
        Keywords & Deep Learning, Brain MRI, Variational Autoencoder\\
		\hline
	\end{tabular}
\end{table}

\pagestyle{fancy}
\renewcommand{\chaptermark}[1]{ \markboth{#1}{}}
\renewcommand{\sectionmark}[1]{\markright{ \thesection.\ #1}}
\lhead[\fancyplain{}{\bfseries\thepage}]{\fancyplain{}{\bfseries\rightmark}}
\rhead[\fancyplain{}{\bfseries\leftmark}]{\fancyplain{}{\bfseries\thepage}}
\cfoot{}

% indice
\cleardoublepage
\phantomsection
\addcontentsline{toc}{chapter}{Index}
\tableofcontents
% listado de figuras
\cleardoublepage
\phantomsection
\addcontentsline{toc}{chapter}{Figure List}
\listoffigures
% listado de tablas
\cleardoublepage
\phantomsection
\addcontentsline{toc}{chapter}{Table List}
\listoftables

\thispagestyle{empty}

\pagenumbering{arabic}

\pagestyle{fancy}
\renewcommand{\chaptermark}[1]{ \markboth{#1}{}}
\renewcommand{\sectionmark}[1]{\markright{ \thesection.\ #1}}
\lhead[\fancyplain{}{\bfseries\thepage}]{\fancyplain{}{\bfseries\rightmark}}
\rhead[\fancyplain{}{\bfseries\leftmark}]{\fancyplain{}{\bfseries\thepage}}
\cfoot{}

\onehalfspacing

\input{introduction/0_abstract.tex}

\chapter{Introduction}

\input{introduction/1_context.tex}
\input{introduction/2_aim.tex}
\input{introduction/5_plan.tex}

\input{state_of_art/main.tex}
\input{impl/main.tex}
\input{conclusions/main.tex}

\clearpage
\printglossary[type=\acronymtype]
\printglossary

\bibliographystyle{plain} % We choose the "plain" reference style
\bibliography{refs}

\end{document}


\clearpage
\printglossary[type=\acronymtype]
\printglossary

\bibliographystyle{plain} % We choose the "plain" reference style
\bibliography{refs}

\end{document}

\documentclass[11pt,a4paper,openany]{book}

\usepackage{amsmath}
\usepackage{graphicx}
\usepackage{hyperref}
\usepackage[utf8]{inputenc}
\usepackage[acronym, toc]{glossaries}

\usepackage{setspace}	%double spacing for text, single for captions, footnotes, etc.
\usepackage{natbib}		% substituye a 'hypernat' que funciona en Windows.
\usepackage[english]{babel}
\usepackage[utf8]{inputenc}
\usepackage{color}
\usepackage{hhline} 		% extended styles for tables
\usepackage{multirow}
\usepackage{subfigure}
\usepackage{amsmath,amsmath,amssymb} 
\usepackage{fancyhdr}
\usepackage{epsfig, amsmath}
\usepackage{algorithm}
\usepackage{algorithmic}

\usepackage[section]{placeins}

% general settings
\hypersetup{
	linktocpage=true,
	colorlinks=true,
	linkcolor=blue,
	citecolor=blue,
}
\definecolor{Hgray}{gray}{0.6}

\newenvironment{definition}[1][Definition]{\begin{trivlist}
\item[\hskip \labelsep {\bfseries #1}]}{\end{trivlist}}

\setlength{\topmargin}{0cm}
\setlength{\textheight}{23cm}
\setlength{\textwidth}{17cm}
\setlength{\oddsidemargin}{0cm}
\setlength{\evensidemargin}{0cm}
\setlength{\headheight}{1cm}

% indica que las 'sub-sub-sections' sean numeradas y aparezcan en el indice
\setcounter{secnumdepth}{3}
\setcounter{tocdepth}{2}

% settings for code
\renewcommand{\algorithmicrequire}{\textbf{Entrada: }}
\renewcommand{\algorithmicensure}{\textbf{Salida: }}

\makeglossaries
\newglossaryentry{kpi}
{
    name=KPI,
    description={Key Performance Indicator}
}

\newglossaryentry{mnist}
{
    name=MNIST,
    description={The MNIST database of handwritten digits used for training models}
}

\newglossaryentry{kdd}
{
    name=KDD,
    description={KDD cup 1999 network intrusion dataset}
}

\newglossaryentry{ixi}
{
    name=IXI,
    description={IXI dataset is a collection of 600 Brain MR Images, in 3-D, collected from 3 different hospitals in London. The images are scanned in 1.5 and 3 Tesla which is the range of the quality that we could usually find in hospitals around the world}
}

\newacronym{pca}{PCA}{Principal Component Analysis}

\newacronym{mri}{MRI}{Magnetic Resonance Image}

\newacronym{vqvae}{VQ-VAE}{Vector-Quantized Variational Autoencoder}

\newacronym{cevae}{ceVAE}{Context-Encoder Variational Autoencoder}

\newacronym{strega}{StRegA}{Segmentation Regularised Anomaly}


\begin{document}

% Portada
\newpage
\thispagestyle{empty}

\baselineskip 2em

%\vspace*{1cm}

\centerline{\includegraphics[width=0.6\textwidth]{images/UOC-logo}}
\begin{center}
\textsc{Universitat Oberta de Catalunya (UOC) \\
 Máster Universitario en Ciencia de Datos (\textit{Data Science})\\}

%\centerline {\pic{UOC}{4cm}}

\vspace*{1.5cm}

\textsc{\Large TRABAJO FINAL DE MÁSTER}

\vspace*{0.5cm}

\textsc{\large Área: YYY}


%\textbf{\Huge VirtualTechLab Model: }

\vspace*{2.0cm}

\title{\Large Artificial MRI brain images creation with Variational Autoencoders}

\vspace{2.5cm}
\baselineskip 1em

\baselineskip 2em
-----------------------------------------------------------------------------\\
Autor:      Miguel Tablado\\
Tutor:      Baris Kanber\\
Profesor:   Nombre del profesor responsable del área de TF\\
-----------------------------------------------------------------------------\\
\vspace*{1.5cm}
Barcelona, \today

\author{Miguel Tablado}

\end{center}



%\tableofcontents

\pagenumbering{roman} 
\setcounter{page}{1} 
\pagestyle{plain}

%%%%%%%%%%%%%
%%% FICHA %%%
%%%%%%%%%%%%%
\chapter*{FICHA DEL TRABAJO FINAL}
\begin{table}[ht]
	\centering{}
	\renewcommand{\arraystretch}{2}
	\begin{tabular}{r | l}
		\hline
		Título del trabajo: & Artificial MRI brain images creation with Variational Autoencoders\\
		\hline
        Nombre del autor: & Miguel Tablado León\\
		\hline
        Nombre del Tutor/a de TF: & Baris Kanber\\
		\hline
        Nombre del/de la PRA: & Ferran Prados Carrasco\\
		\hline
        Fecha de entrega: & 02/2023\\
		\hline
        Titulación o programa: & Máster en Ciencia de Datos\\
		\hline
        Área del Trabajo Final: & Medicine Area (TFM-Med)\\
		\hline
        Language: & English\\
		\hline
        Keywords & Deep Learning, Brain MRI, Variational Autoencoder\\
		\hline
	\end{tabular}
\end{table}

\pagestyle{fancy}
\renewcommand{\chaptermark}[1]{ \markboth{#1}{}}
\renewcommand{\sectionmark}[1]{\markright{ \thesection.\ #1}}
\lhead[\fancyplain{}{\bfseries\thepage}]{\fancyplain{}{\bfseries\rightmark}}
\rhead[\fancyplain{}{\bfseries\leftmark}]{\fancyplain{}{\bfseries\thepage}}
\cfoot{}

% indice
\cleardoublepage
\phantomsection
\addcontentsline{toc}{chapter}{Index}
\tableofcontents
% listado de figuras
\cleardoublepage
\phantomsection
\addcontentsline{toc}{chapter}{Figure List}
\listoffigures
% listado de tablas
\cleardoublepage
\phantomsection
\addcontentsline{toc}{chapter}{Table List}
\listoftables

\thispagestyle{empty}

\pagenumbering{arabic}

\pagestyle{fancy}
\renewcommand{\chaptermark}[1]{ \markboth{#1}{}}
\renewcommand{\sectionmark}[1]{\markright{ \thesection.\ #1}}
\lhead[\fancyplain{}{\bfseries\thepage}]{\fancyplain{}{\bfseries\rightmark}}
\rhead[\fancyplain{}{\bfseries\leftmark}]{\fancyplain{}{\bfseries\thepage}}
\cfoot{}

\onehalfspacing

\input{introduction/0_abstract.tex}

\chapter{Introduction}

\section{Context and project justification}

Artificial Intelligence has arrived to change the world in almost (if not all) any field. Today, we are surrounded by (and we are using many) AI products like smartphones’ face recognition capabilities, home cleaning robots or cars with autopilot options.

From the different AI fields, computer vision is maybe the most popular one and the one which is usually used for explaining AI capabilities to general public. Identifying a cat in a picture could perfectly be the example used in every AI presentation to welcome people to AI.

Image processing is intuitively matched with medical diagnosis by anyone having or not any expertise on the field. Every single citizen will have heard of magnetic resonance imaging, and anyone easily transposes ‘cat detection’ to ‘anomaly detection’, being the anomaly a tumor or anything else.

There are different techniques to scan people and create images for clinical diagnosis like X-Ray or Magnetic Resonance Imaging. In this project, we will work with Magnetic Resonance Images (MRI) which are images created by a machine with a large bore that scans people lying inside it. The MR technique is non-invasive, it produces no radiation, and is used to scan almost any part of the body from which we will focus on brain images.

Combining AI diagnosis capabilities on image processing and MRI images, we can think of helping doctors to identify the presence of anomalies or looking for concrete diagnosis for a specific disease.

Obviously, these projects are not easy at all and they face a lot of challenges. One of the first challenges that such AI project faces is the difficulty to obtain a large set of brain images that are needed to train an accurate AI model. Scanning people is too costly and requires a lot of time. and the challenge of having a large dataset gets harder once we understand that brain images may differ depending on age or gender.

Today, there is a clear limitation on how to reproduce or obtain healthy brain images for AI-based diagnosis projects while the appearance of new AI techniques known as Autoencoders and Variational Autoencoders introduces a new area of investigation to mitigate the gap.

This project aims to be a proof of concept to create AI-created images with new variational autoencoders that would serve to augment any existing MRI dataset and that will help to improve the accuracy of brain anomaly detection projects, including those which could be used for overall anomaly detection to others more specific which would help on concrete disease diagnosis.

Personal motivation comes from various angles:

\begin{itemize}
    \item One is to prove myself that I can work with new AI architectures demonstrating that I have acquired the knowledge needed (deep enough) to be productive and to be able to innovate in the health sector.
    \item Deep Learning has been the subject which I enjoyed the most, hence continuing with Autoencoders seems natural to me as the next step on AI adoption
    \item At no doubts, if I can contribute to help on brain issue detection or diagnosis, I will feel my life been completely fulfilled
\end{itemize}
\section{Aim of the project}

The aim of this project is to serve as a Proof of Concept on how MRI brain images can be artificially generated with Variational Autoencoders which ultimately would serve to enhance existing or new datasets to improve model accuracy by having larger samples of data.

Foreseen projects objectives are:

\begin{itemize}
    \item Obtain basic knowledge about MRI images and the NIFTI file format
    \item Obtain and visualise 2-D images from 3-D images in the dataset
    \item Select what range of slices (2-D images) from brain to be created
    \item Test different existing networks and choose the one to be used
    \item Tune network parameters
    \item Compare generated brain images against real ones, qualitatively
\end{itemize}
\section{Project Plan}

\subsection{Resources}

\begin{enumerate}
    \item 1 Data Scientist: Miguel Tablado will be playing this role and will dedicate 300h
    \item 1 Coach/Tutor: Baris Kanber will be assisting Miguel Tablado during the project
    \item MRI images and demographic information from IXI Dataset
    \item GPU resources are needed to train and test the network
\end{enumerate}

\subsection{High Level Plan}

The plan will be executed in 3 different phases with the listed tasks:
\begin{enumerate}
    \item Phase 1: Analysis
    \begin{enumerate}
        \item Gain knowledge on MRI and NIFTI protocol
        \item Describe images and dataset
        \item Extract 2D images
        \item Pre-processing images
    \end{enumerate}
    \item Phase 2: MR Images creation
    \begin{enumerate}
        \item Test different Network architectures
        \item Tune-up architectural network
    \end{enumerate}
    \item Phase 3: Project documentation
    \begin{enumerate}
        \item Write conclusions
        \item Create project documentation
        \item Create project presentation
    \end{enumerate}
\end{enumerate}

\begin{figure}[ht]
    \hspace*{-1.2in}
    \centering
    \includegraphics[width = 20cm, height = 6cm]{images/project-plan.png}
    \caption[]{Project Plan. source: https://www.onlinegantt.com}
    \label{fig:project-plan}
\end{figure}

\newpage
\subsection{Tasks}
\subsubsection*{Phase 1: Analysis}

During this face the different tasks will be executed to prepare the work and includes:
\begin{enumerate}
    \item Gain knowledge on MRI and the NIFTI file format: This activity consists of reading papers and documents to gain sufficient knowledge to execute the project. There is no need to become an expert on the matter but understanding how those files are and how to process them.
    \item Describe images and dataset: During this activity, a description of the dataset will be generated with a view of the quality of the dataset for the aim of the project and any findings which could result.
    \item Extract 2D images: Load 3D images and extract 2D slices from the original dataset, which will be depicted with code.
    \item Pre-processing images: Decide which transformations on the 2D images would help the project like pixel changes or applying gray-scale transformations.
\end{enumerate}

\subsubsection*{Phase 2: MR Images creation}

\begin{enumerate}
    \item Test different Network architectures: This task will take few existing network architectures and be tested with the dataset so that one of them will be selected to be improved and used as the project architecture.
    \item Tune-up architectural network: Tune the selected architecture with useful techniques like changing network layers
\end{enumerate}

\documentclass[11pt,a4paper,openany]{book}

\usepackage{amsmath}
\usepackage{graphicx}
\usepackage{hyperref}
\usepackage[utf8]{inputenc}
\usepackage[acronym, toc]{glossaries}

\usepackage{setspace}	%double spacing for text, single for captions, footnotes, etc.
\usepackage{natbib}		% substituye a 'hypernat' que funciona en Windows.
\usepackage[english]{babel}
\usepackage[utf8]{inputenc}
\usepackage{color}
\usepackage{hhline} 		% extended styles for tables
\usepackage{multirow}
\usepackage{subfigure}
\usepackage{amsmath,amsmath,amssymb} 
\usepackage{fancyhdr}
\usepackage{epsfig, amsmath}
\usepackage{algorithm}
\usepackage{algorithmic}

\usepackage[section]{placeins}

% general settings
\hypersetup{
	linktocpage=true,
	colorlinks=true,
	linkcolor=blue,
	citecolor=blue,
}
\definecolor{Hgray}{gray}{0.6}

\newenvironment{definition}[1][Definition]{\begin{trivlist}
\item[\hskip \labelsep {\bfseries #1}]}{\end{trivlist}}

\setlength{\topmargin}{0cm}
\setlength{\textheight}{23cm}
\setlength{\textwidth}{17cm}
\setlength{\oddsidemargin}{0cm}
\setlength{\evensidemargin}{0cm}
\setlength{\headheight}{1cm}

% indica que las 'sub-sub-sections' sean numeradas y aparezcan en el indice
\setcounter{secnumdepth}{3}
\setcounter{tocdepth}{2}

% settings for code
\renewcommand{\algorithmicrequire}{\textbf{Entrada: }}
\renewcommand{\algorithmicensure}{\textbf{Salida: }}

\makeglossaries
\input{closing/glossary.tex}

\begin{document}

% Portada
\input{0_tittle.tex}


%\tableofcontents

\pagenumbering{roman} 
\setcounter{page}{1} 
\pagestyle{plain}

%%%%%%%%%%%%%
%%% FICHA %%%
%%%%%%%%%%%%%
\chapter*{FICHA DEL TRABAJO FINAL}
\begin{table}[ht]
	\centering{}
	\renewcommand{\arraystretch}{2}
	\begin{tabular}{r | l}
		\hline
		Título del trabajo: & Artificial MRI brain images creation with Variational Autoencoders\\
		\hline
        Nombre del autor: & Miguel Tablado León\\
		\hline
        Nombre del Tutor/a de TF: & Baris Kanber\\
		\hline
        Nombre del/de la PRA: & Ferran Prados Carrasco\\
		\hline
        Fecha de entrega: & 02/2023\\
		\hline
        Titulación o programa: & Máster en Ciencia de Datos\\
		\hline
        Área del Trabajo Final: & Medicine Area (TFM-Med)\\
		\hline
        Language: & English\\
		\hline
        Keywords & Deep Learning, Brain MRI, Variational Autoencoder\\
		\hline
	\end{tabular}
\end{table}

\pagestyle{fancy}
\renewcommand{\chaptermark}[1]{ \markboth{#1}{}}
\renewcommand{\sectionmark}[1]{\markright{ \thesection.\ #1}}
\lhead[\fancyplain{}{\bfseries\thepage}]{\fancyplain{}{\bfseries\rightmark}}
\rhead[\fancyplain{}{\bfseries\leftmark}]{\fancyplain{}{\bfseries\thepage}}
\cfoot{}

% indice
\cleardoublepage
\phantomsection
\addcontentsline{toc}{chapter}{Index}
\tableofcontents
% listado de figuras
\cleardoublepage
\phantomsection
\addcontentsline{toc}{chapter}{Figure List}
\listoffigures
% listado de tablas
\cleardoublepage
\phantomsection
\addcontentsline{toc}{chapter}{Table List}
\listoftables

\thispagestyle{empty}

\pagenumbering{arabic}

\pagestyle{fancy}
\renewcommand{\chaptermark}[1]{ \markboth{#1}{}}
\renewcommand{\sectionmark}[1]{\markright{ \thesection.\ #1}}
\lhead[\fancyplain{}{\bfseries\thepage}]{\fancyplain{}{\bfseries\rightmark}}
\rhead[\fancyplain{}{\bfseries\leftmark}]{\fancyplain{}{\bfseries\thepage}}
\cfoot{}

\onehalfspacing

\input{introduction/0_abstract.tex}

\chapter{Introduction}

\input{introduction/1_context.tex}
\input{introduction/2_aim.tex}
\input{introduction/5_plan.tex}

\input{state_of_art/main.tex}
\input{impl/main.tex}
\input{conclusions/main.tex}

\clearpage
\printglossary[type=\acronymtype]
\printglossary

\bibliographystyle{plain} % We choose the "plain" reference style
\bibliography{refs}

\end{document}

\documentclass[11pt,a4paper,openany]{book}

\usepackage{amsmath}
\usepackage{graphicx}
\usepackage{hyperref}
\usepackage[utf8]{inputenc}
\usepackage[acronym, toc]{glossaries}

\usepackage{setspace}	%double spacing for text, single for captions, footnotes, etc.
\usepackage{natbib}		% substituye a 'hypernat' que funciona en Windows.
\usepackage[english]{babel}
\usepackage[utf8]{inputenc}
\usepackage{color}
\usepackage{hhline} 		% extended styles for tables
\usepackage{multirow}
\usepackage{subfigure}
\usepackage{amsmath,amsmath,amssymb} 
\usepackage{fancyhdr}
\usepackage{epsfig, amsmath}
\usepackage{algorithm}
\usepackage{algorithmic}

\usepackage[section]{placeins}

% general settings
\hypersetup{
	linktocpage=true,
	colorlinks=true,
	linkcolor=blue,
	citecolor=blue,
}
\definecolor{Hgray}{gray}{0.6}

\newenvironment{definition}[1][Definition]{\begin{trivlist}
\item[\hskip \labelsep {\bfseries #1}]}{\end{trivlist}}

\setlength{\topmargin}{0cm}
\setlength{\textheight}{23cm}
\setlength{\textwidth}{17cm}
\setlength{\oddsidemargin}{0cm}
\setlength{\evensidemargin}{0cm}
\setlength{\headheight}{1cm}

% indica que las 'sub-sub-sections' sean numeradas y aparezcan en el indice
\setcounter{secnumdepth}{3}
\setcounter{tocdepth}{2}

% settings for code
\renewcommand{\algorithmicrequire}{\textbf{Entrada: }}
\renewcommand{\algorithmicensure}{\textbf{Salida: }}

\makeglossaries
\input{closing/glossary.tex}

\begin{document}

% Portada
\input{0_tittle.tex}


%\tableofcontents

\pagenumbering{roman} 
\setcounter{page}{1} 
\pagestyle{plain}

%%%%%%%%%%%%%
%%% FICHA %%%
%%%%%%%%%%%%%
\chapter*{FICHA DEL TRABAJO FINAL}
\begin{table}[ht]
	\centering{}
	\renewcommand{\arraystretch}{2}
	\begin{tabular}{r | l}
		\hline
		Título del trabajo: & Artificial MRI brain images creation with Variational Autoencoders\\
		\hline
        Nombre del autor: & Miguel Tablado León\\
		\hline
        Nombre del Tutor/a de TF: & Baris Kanber\\
		\hline
        Nombre del/de la PRA: & Ferran Prados Carrasco\\
		\hline
        Fecha de entrega: & 02/2023\\
		\hline
        Titulación o programa: & Máster en Ciencia de Datos\\
		\hline
        Área del Trabajo Final: & Medicine Area (TFM-Med)\\
		\hline
        Language: & English\\
		\hline
        Keywords & Deep Learning, Brain MRI, Variational Autoencoder\\
		\hline
	\end{tabular}
\end{table}

\pagestyle{fancy}
\renewcommand{\chaptermark}[1]{ \markboth{#1}{}}
\renewcommand{\sectionmark}[1]{\markright{ \thesection.\ #1}}
\lhead[\fancyplain{}{\bfseries\thepage}]{\fancyplain{}{\bfseries\rightmark}}
\rhead[\fancyplain{}{\bfseries\leftmark}]{\fancyplain{}{\bfseries\thepage}}
\cfoot{}

% indice
\cleardoublepage
\phantomsection
\addcontentsline{toc}{chapter}{Index}
\tableofcontents
% listado de figuras
\cleardoublepage
\phantomsection
\addcontentsline{toc}{chapter}{Figure List}
\listoffigures
% listado de tablas
\cleardoublepage
\phantomsection
\addcontentsline{toc}{chapter}{Table List}
\listoftables

\thispagestyle{empty}

\pagenumbering{arabic}

\pagestyle{fancy}
\renewcommand{\chaptermark}[1]{ \markboth{#1}{}}
\renewcommand{\sectionmark}[1]{\markright{ \thesection.\ #1}}
\lhead[\fancyplain{}{\bfseries\thepage}]{\fancyplain{}{\bfseries\rightmark}}
\rhead[\fancyplain{}{\bfseries\leftmark}]{\fancyplain{}{\bfseries\thepage}}
\cfoot{}

\onehalfspacing

\input{introduction/0_abstract.tex}

\chapter{Introduction}

\input{introduction/1_context.tex}
\input{introduction/2_aim.tex}
\input{introduction/5_plan.tex}

\input{state_of_art/main.tex}
\input{impl/main.tex}
\input{conclusions/main.tex}

\clearpage
\printglossary[type=\acronymtype]
\printglossary

\bibliographystyle{plain} % We choose the "plain" reference style
\bibliography{refs}

\end{document}

\documentclass[11pt,a4paper,openany]{book}

\usepackage{amsmath}
\usepackage{graphicx}
\usepackage{hyperref}
\usepackage[utf8]{inputenc}
\usepackage[acronym, toc]{glossaries}

\usepackage{setspace}	%double spacing for text, single for captions, footnotes, etc.
\usepackage{natbib}		% substituye a 'hypernat' que funciona en Windows.
\usepackage[english]{babel}
\usepackage[utf8]{inputenc}
\usepackage{color}
\usepackage{hhline} 		% extended styles for tables
\usepackage{multirow}
\usepackage{subfigure}
\usepackage{amsmath,amsmath,amssymb} 
\usepackage{fancyhdr}
\usepackage{epsfig, amsmath}
\usepackage{algorithm}
\usepackage{algorithmic}

\usepackage[section]{placeins}

% general settings
\hypersetup{
	linktocpage=true,
	colorlinks=true,
	linkcolor=blue,
	citecolor=blue,
}
\definecolor{Hgray}{gray}{0.6}

\newenvironment{definition}[1][Definition]{\begin{trivlist}
\item[\hskip \labelsep {\bfseries #1}]}{\end{trivlist}}

\setlength{\topmargin}{0cm}
\setlength{\textheight}{23cm}
\setlength{\textwidth}{17cm}
\setlength{\oddsidemargin}{0cm}
\setlength{\evensidemargin}{0cm}
\setlength{\headheight}{1cm}

% indica que las 'sub-sub-sections' sean numeradas y aparezcan en el indice
\setcounter{secnumdepth}{3}
\setcounter{tocdepth}{2}

% settings for code
\renewcommand{\algorithmicrequire}{\textbf{Entrada: }}
\renewcommand{\algorithmicensure}{\textbf{Salida: }}

\makeglossaries
\input{closing/glossary.tex}

\begin{document}

% Portada
\input{0_tittle.tex}


%\tableofcontents

\pagenumbering{roman} 
\setcounter{page}{1} 
\pagestyle{plain}

%%%%%%%%%%%%%
%%% FICHA %%%
%%%%%%%%%%%%%
\chapter*{FICHA DEL TRABAJO FINAL}
\begin{table}[ht]
	\centering{}
	\renewcommand{\arraystretch}{2}
	\begin{tabular}{r | l}
		\hline
		Título del trabajo: & Artificial MRI brain images creation with Variational Autoencoders\\
		\hline
        Nombre del autor: & Miguel Tablado León\\
		\hline
        Nombre del Tutor/a de TF: & Baris Kanber\\
		\hline
        Nombre del/de la PRA: & Ferran Prados Carrasco\\
		\hline
        Fecha de entrega: & 02/2023\\
		\hline
        Titulación o programa: & Máster en Ciencia de Datos\\
		\hline
        Área del Trabajo Final: & Medicine Area (TFM-Med)\\
		\hline
        Language: & English\\
		\hline
        Keywords & Deep Learning, Brain MRI, Variational Autoencoder\\
		\hline
	\end{tabular}
\end{table}

\pagestyle{fancy}
\renewcommand{\chaptermark}[1]{ \markboth{#1}{}}
\renewcommand{\sectionmark}[1]{\markright{ \thesection.\ #1}}
\lhead[\fancyplain{}{\bfseries\thepage}]{\fancyplain{}{\bfseries\rightmark}}
\rhead[\fancyplain{}{\bfseries\leftmark}]{\fancyplain{}{\bfseries\thepage}}
\cfoot{}

% indice
\cleardoublepage
\phantomsection
\addcontentsline{toc}{chapter}{Index}
\tableofcontents
% listado de figuras
\cleardoublepage
\phantomsection
\addcontentsline{toc}{chapter}{Figure List}
\listoffigures
% listado de tablas
\cleardoublepage
\phantomsection
\addcontentsline{toc}{chapter}{Table List}
\listoftables

\thispagestyle{empty}

\pagenumbering{arabic}

\pagestyle{fancy}
\renewcommand{\chaptermark}[1]{ \markboth{#1}{}}
\renewcommand{\sectionmark}[1]{\markright{ \thesection.\ #1}}
\lhead[\fancyplain{}{\bfseries\thepage}]{\fancyplain{}{\bfseries\rightmark}}
\rhead[\fancyplain{}{\bfseries\leftmark}]{\fancyplain{}{\bfseries\thepage}}
\cfoot{}

\onehalfspacing

\input{introduction/0_abstract.tex}

\chapter{Introduction}

\input{introduction/1_context.tex}
\input{introduction/2_aim.tex}
\input{introduction/5_plan.tex}

\input{state_of_art/main.tex}
\input{impl/main.tex}
\input{conclusions/main.tex}

\clearpage
\printglossary[type=\acronymtype]
\printglossary

\bibliographystyle{plain} % We choose the "plain" reference style
\bibliography{refs}

\end{document}


\clearpage
\printglossary[type=\acronymtype]
\printglossary

\bibliographystyle{plain} % We choose the "plain" reference style
\bibliography{refs}

\end{document}


\clearpage
\printglossary[type=\acronymtype]
\printglossary

\bibliographystyle{plain} % We choose the "plain" reference style
\bibliography{refs}

\end{document}

\documentclass[11pt,a4paper,openany]{book}

\usepackage{amsmath}
\usepackage{graphicx}
\usepackage{hyperref}
\usepackage[utf8]{inputenc}
\usepackage[acronym, toc]{glossaries}

\usepackage{setspace}	%double spacing for text, single for captions, footnotes, etc.
\usepackage{natbib}		% substituye a 'hypernat' que funciona en Windows.
\usepackage[english]{babel}
\usepackage[utf8]{inputenc}
\usepackage{color}
\usepackage{hhline} 		% extended styles for tables
\usepackage{multirow}
\usepackage{subfigure}
\usepackage{amsmath,amsmath,amssymb} 
\usepackage{fancyhdr}
\usepackage{epsfig, amsmath}
\usepackage{algorithm}
\usepackage{algorithmic}

\usepackage[section]{placeins}

% general settings
\hypersetup{
	linktocpage=true,
	colorlinks=true,
	linkcolor=blue,
	citecolor=blue,
}
\definecolor{Hgray}{gray}{0.6}

\newenvironment{definition}[1][Definition]{\begin{trivlist}
\item[\hskip \labelsep {\bfseries #1}]}{\end{trivlist}}

\setlength{\topmargin}{0cm}
\setlength{\textheight}{23cm}
\setlength{\textwidth}{17cm}
\setlength{\oddsidemargin}{0cm}
\setlength{\evensidemargin}{0cm}
\setlength{\headheight}{1cm}

% indica que las 'sub-sub-sections' sean numeradas y aparezcan en el indice
\setcounter{secnumdepth}{3}
\setcounter{tocdepth}{2}

% settings for code
\renewcommand{\algorithmicrequire}{\textbf{Entrada: }}
\renewcommand{\algorithmicensure}{\textbf{Salida: }}

\makeglossaries
\newglossaryentry{kpi}
{
    name=KPI,
    description={Key Performance Indicator}
}

\newglossaryentry{mnist}
{
    name=MNIST,
    description={The MNIST database of handwritten digits used for training models}
}

\newglossaryentry{kdd}
{
    name=KDD,
    description={KDD cup 1999 network intrusion dataset}
}

\newglossaryentry{ixi}
{
    name=IXI,
    description={IXI dataset is a collection of 600 Brain MR Images, in 3-D, collected from 3 different hospitals in London. The images are scanned in 1.5 and 3 Tesla which is the range of the quality that we could usually find in hospitals around the world}
}

\newacronym{pca}{PCA}{Principal Component Analysis}

\newacronym{mri}{MRI}{Magnetic Resonance Image}

\newacronym{vqvae}{VQ-VAE}{Vector-Quantized Variational Autoencoder}

\newacronym{cevae}{ceVAE}{Context-Encoder Variational Autoencoder}

\newacronym{strega}{StRegA}{Segmentation Regularised Anomaly}


\begin{document}

% Portada
\newpage
\thispagestyle{empty}

\baselineskip 2em

%\vspace*{1cm}

\centerline{\includegraphics[width=0.6\textwidth]{images/UOC-logo}}
\begin{center}
\textsc{Universitat Oberta de Catalunya (UOC) \\
 Máster Universitario en Ciencia de Datos (\textit{Data Science})\\}

%\centerline {\pic{UOC}{4cm}}

\vspace*{1.5cm}

\textsc{\Large TRABAJO FINAL DE MÁSTER}

\vspace*{0.5cm}

\textsc{\large Área: YYY}


%\textbf{\Huge VirtualTechLab Model: }

\vspace*{2.0cm}

\title{\Large Artificial MRI brain images creation with Variational Autoencoders}

\vspace{2.5cm}
\baselineskip 1em

\baselineskip 2em
-----------------------------------------------------------------------------\\
Autor:      Miguel Tablado\\
Tutor:      Baris Kanber\\
Profesor:   Nombre del profesor responsable del área de TF\\
-----------------------------------------------------------------------------\\
\vspace*{1.5cm}
Barcelona, \today

\author{Miguel Tablado}

\end{center}



%\tableofcontents

\pagenumbering{roman} 
\setcounter{page}{1} 
\pagestyle{plain}

%%%%%%%%%%%%%
%%% FICHA %%%
%%%%%%%%%%%%%
\chapter*{FICHA DEL TRABAJO FINAL}
\begin{table}[ht]
	\centering{}
	\renewcommand{\arraystretch}{2}
	\begin{tabular}{r | l}
		\hline
		Título del trabajo: & Artificial MRI brain images creation with Variational Autoencoders\\
		\hline
        Nombre del autor: & Miguel Tablado León\\
		\hline
        Nombre del Tutor/a de TF: & Baris Kanber\\
		\hline
        Nombre del/de la PRA: & Ferran Prados Carrasco\\
		\hline
        Fecha de entrega: & 02/2023\\
		\hline
        Titulación o programa: & Máster en Ciencia de Datos\\
		\hline
        Área del Trabajo Final: & Medicine Area (TFM-Med)\\
		\hline
        Language: & English\\
		\hline
        Keywords & Deep Learning, Brain MRI, Variational Autoencoder\\
		\hline
	\end{tabular}
\end{table}

\pagestyle{fancy}
\renewcommand{\chaptermark}[1]{ \markboth{#1}{}}
\renewcommand{\sectionmark}[1]{\markright{ \thesection.\ #1}}
\lhead[\fancyplain{}{\bfseries\thepage}]{\fancyplain{}{\bfseries\rightmark}}
\rhead[\fancyplain{}{\bfseries\leftmark}]{\fancyplain{}{\bfseries\thepage}}
\cfoot{}

% indice
\cleardoublepage
\phantomsection
\addcontentsline{toc}{chapter}{Index}
\tableofcontents
% listado de figuras
\cleardoublepage
\phantomsection
\addcontentsline{toc}{chapter}{Figure List}
\listoffigures
% listado de tablas
\cleardoublepage
\phantomsection
\addcontentsline{toc}{chapter}{Table List}
\listoftables

\thispagestyle{empty}

\pagenumbering{arabic}

\pagestyle{fancy}
\renewcommand{\chaptermark}[1]{ \markboth{#1}{}}
\renewcommand{\sectionmark}[1]{\markright{ \thesection.\ #1}}
\lhead[\fancyplain{}{\bfseries\thepage}]{\fancyplain{}{\bfseries\rightmark}}
\rhead[\fancyplain{}{\bfseries\leftmark}]{\fancyplain{}{\bfseries\thepage}}
\cfoot{}

\onehalfspacing

\input{introduction/0_abstract.tex}

\chapter{Introduction}

\section{Context and project justification}

Artificial Intelligence has arrived to change the world in almost (if not all) any field. Today, we are surrounded by (and we are using many) AI products like smartphones’ face recognition capabilities, home cleaning robots or cars with autopilot options.

From the different AI fields, computer vision is maybe the most popular one and the one which is usually used for explaining AI capabilities to general public. Identifying a cat in a picture could perfectly be the example used in every AI presentation to welcome people to AI.

Image processing is intuitively matched with medical diagnosis by anyone having or not any expertise on the field. Every single citizen will have heard of magnetic resonance imaging, and anyone easily transposes ‘cat detection’ to ‘anomaly detection’, being the anomaly a tumor or anything else.

There are different techniques to scan people and create images for clinical diagnosis like X-Ray or Magnetic Resonance Imaging. In this project, we will work with Magnetic Resonance Images (MRI) which are images created by a machine with a large bore that scans people lying inside it. The MR technique is non-invasive, it produces no radiation, and is used to scan almost any part of the body from which we will focus on brain images.

Combining AI diagnosis capabilities on image processing and MRI images, we can think of helping doctors to identify the presence of anomalies or looking for concrete diagnosis for a specific disease.

Obviously, these projects are not easy at all and they face a lot of challenges. One of the first challenges that such AI project faces is the difficulty to obtain a large set of brain images that are needed to train an accurate AI model. Scanning people is too costly and requires a lot of time. and the challenge of having a large dataset gets harder once we understand that brain images may differ depending on age or gender.

Today, there is a clear limitation on how to reproduce or obtain healthy brain images for AI-based diagnosis projects while the appearance of new AI techniques known as Autoencoders and Variational Autoencoders introduces a new area of investigation to mitigate the gap.

This project aims to be a proof of concept to create AI-created images with new variational autoencoders that would serve to augment any existing MRI dataset and that will help to improve the accuracy of brain anomaly detection projects, including those which could be used for overall anomaly detection to others more specific which would help on concrete disease diagnosis.

Personal motivation comes from various angles:

\begin{itemize}
    \item One is to prove myself that I can work with new AI architectures demonstrating that I have acquired the knowledge needed (deep enough) to be productive and to be able to innovate in the health sector.
    \item Deep Learning has been the subject which I enjoyed the most, hence continuing with Autoencoders seems natural to me as the next step on AI adoption
    \item At no doubts, if I can contribute to help on brain issue detection or diagnosis, I will feel my life been completely fulfilled
\end{itemize}
\section{Aim of the project}

The aim of this project is to serve as a Proof of Concept on how MRI brain images can be artificially generated with Variational Autoencoders which ultimately would serve to enhance existing or new datasets to improve model accuracy by having larger samples of data.

Foreseen projects objectives are:

\begin{itemize}
    \item Obtain basic knowledge about MRI images and the NIFTI file format
    \item Obtain and visualise 2-D images from 3-D images in the dataset
    \item Select what range of slices (2-D images) from brain to be created
    \item Test different existing networks and choose the one to be used
    \item Tune network parameters
    \item Compare generated brain images against real ones, qualitatively
\end{itemize}
\section{Project Plan}

\subsection{Resources}

\begin{enumerate}
    \item 1 Data Scientist: Miguel Tablado will be playing this role and will dedicate 300h
    \item 1 Coach/Tutor: Baris Kanber will be assisting Miguel Tablado during the project
    \item MRI images and demographic information from IXI Dataset
    \item GPU resources are needed to train and test the network
\end{enumerate}

\subsection{High Level Plan}

The plan will be executed in 3 different phases with the listed tasks:
\begin{enumerate}
    \item Phase 1: Analysis
    \begin{enumerate}
        \item Gain knowledge on MRI and NIFTI protocol
        \item Describe images and dataset
        \item Extract 2D images
        \item Pre-processing images
    \end{enumerate}
    \item Phase 2: MR Images creation
    \begin{enumerate}
        \item Test different Network architectures
        \item Tune-up architectural network
    \end{enumerate}
    \item Phase 3: Project documentation
    \begin{enumerate}
        \item Write conclusions
        \item Create project documentation
        \item Create project presentation
    \end{enumerate}
\end{enumerate}

\begin{figure}[ht]
    \hspace*{-1.2in}
    \centering
    \includegraphics[width = 20cm, height = 6cm]{images/project-plan.png}
    \caption[]{Project Plan. source: https://www.onlinegantt.com}
    \label{fig:project-plan}
\end{figure}

\newpage
\subsection{Tasks}
\subsubsection*{Phase 1: Analysis}

During this face the different tasks will be executed to prepare the work and includes:
\begin{enumerate}
    \item Gain knowledge on MRI and the NIFTI file format: This activity consists of reading papers and documents to gain sufficient knowledge to execute the project. There is no need to become an expert on the matter but understanding how those files are and how to process them.
    \item Describe images and dataset: During this activity, a description of the dataset will be generated with a view of the quality of the dataset for the aim of the project and any findings which could result.
    \item Extract 2D images: Load 3D images and extract 2D slices from the original dataset, which will be depicted with code.
    \item Pre-processing images: Decide which transformations on the 2D images would help the project like pixel changes or applying gray-scale transformations.
\end{enumerate}

\subsubsection*{Phase 2: MR Images creation}

\begin{enumerate}
    \item Test different Network architectures: This task will take few existing network architectures and be tested with the dataset so that one of them will be selected to be improved and used as the project architecture.
    \item Tune-up architectural network: Tune the selected architecture with useful techniques like changing network layers
\end{enumerate}

\documentclass[11pt,a4paper,openany]{book}

\usepackage{amsmath}
\usepackage{graphicx}
\usepackage{hyperref}
\usepackage[utf8]{inputenc}
\usepackage[acronym, toc]{glossaries}

\usepackage{setspace}	%double spacing for text, single for captions, footnotes, etc.
\usepackage{natbib}		% substituye a 'hypernat' que funciona en Windows.
\usepackage[english]{babel}
\usepackage[utf8]{inputenc}
\usepackage{color}
\usepackage{hhline} 		% extended styles for tables
\usepackage{multirow}
\usepackage{subfigure}
\usepackage{amsmath,amsmath,amssymb} 
\usepackage{fancyhdr}
\usepackage{epsfig, amsmath}
\usepackage{algorithm}
\usepackage{algorithmic}

\usepackage[section]{placeins}

% general settings
\hypersetup{
	linktocpage=true,
	colorlinks=true,
	linkcolor=blue,
	citecolor=blue,
}
\definecolor{Hgray}{gray}{0.6}

\newenvironment{definition}[1][Definition]{\begin{trivlist}
\item[\hskip \labelsep {\bfseries #1}]}{\end{trivlist}}

\setlength{\topmargin}{0cm}
\setlength{\textheight}{23cm}
\setlength{\textwidth}{17cm}
\setlength{\oddsidemargin}{0cm}
\setlength{\evensidemargin}{0cm}
\setlength{\headheight}{1cm}

% indica que las 'sub-sub-sections' sean numeradas y aparezcan en el indice
\setcounter{secnumdepth}{3}
\setcounter{tocdepth}{2}

% settings for code
\renewcommand{\algorithmicrequire}{\textbf{Entrada: }}
\renewcommand{\algorithmicensure}{\textbf{Salida: }}

\makeglossaries
\newglossaryentry{kpi}
{
    name=KPI,
    description={Key Performance Indicator}
}

\newglossaryentry{mnist}
{
    name=MNIST,
    description={The MNIST database of handwritten digits used for training models}
}

\newglossaryentry{kdd}
{
    name=KDD,
    description={KDD cup 1999 network intrusion dataset}
}

\newglossaryentry{ixi}
{
    name=IXI,
    description={IXI dataset is a collection of 600 Brain MR Images, in 3-D, collected from 3 different hospitals in London. The images are scanned in 1.5 and 3 Tesla which is the range of the quality that we could usually find in hospitals around the world}
}

\newacronym{pca}{PCA}{Principal Component Analysis}

\newacronym{mri}{MRI}{Magnetic Resonance Image}

\newacronym{vqvae}{VQ-VAE}{Vector-Quantized Variational Autoencoder}

\newacronym{cevae}{ceVAE}{Context-Encoder Variational Autoencoder}

\newacronym{strega}{StRegA}{Segmentation Regularised Anomaly}


\begin{document}

% Portada
\newpage
\thispagestyle{empty}

\baselineskip 2em

%\vspace*{1cm}

\centerline{\includegraphics[width=0.6\textwidth]{images/UOC-logo}}
\begin{center}
\textsc{Universitat Oberta de Catalunya (UOC) \\
 Máster Universitario en Ciencia de Datos (\textit{Data Science})\\}

%\centerline {\pic{UOC}{4cm}}

\vspace*{1.5cm}

\textsc{\Large TRABAJO FINAL DE MÁSTER}

\vspace*{0.5cm}

\textsc{\large Área: YYY}


%\textbf{\Huge VirtualTechLab Model: }

\vspace*{2.0cm}

\title{\Large Artificial MRI brain images creation with Variational Autoencoders}

\vspace{2.5cm}
\baselineskip 1em

\baselineskip 2em
-----------------------------------------------------------------------------\\
Autor:      Miguel Tablado\\
Tutor:      Baris Kanber\\
Profesor:   Nombre del profesor responsable del área de TF\\
-----------------------------------------------------------------------------\\
\vspace*{1.5cm}
Barcelona, \today

\author{Miguel Tablado}

\end{center}



%\tableofcontents

\pagenumbering{roman} 
\setcounter{page}{1} 
\pagestyle{plain}

%%%%%%%%%%%%%
%%% FICHA %%%
%%%%%%%%%%%%%
\chapter*{FICHA DEL TRABAJO FINAL}
\begin{table}[ht]
	\centering{}
	\renewcommand{\arraystretch}{2}
	\begin{tabular}{r | l}
		\hline
		Título del trabajo: & Artificial MRI brain images creation with Variational Autoencoders\\
		\hline
        Nombre del autor: & Miguel Tablado León\\
		\hline
        Nombre del Tutor/a de TF: & Baris Kanber\\
		\hline
        Nombre del/de la PRA: & Ferran Prados Carrasco\\
		\hline
        Fecha de entrega: & 02/2023\\
		\hline
        Titulación o programa: & Máster en Ciencia de Datos\\
		\hline
        Área del Trabajo Final: & Medicine Area (TFM-Med)\\
		\hline
        Language: & English\\
		\hline
        Keywords & Deep Learning, Brain MRI, Variational Autoencoder\\
		\hline
	\end{tabular}
\end{table}

\pagestyle{fancy}
\renewcommand{\chaptermark}[1]{ \markboth{#1}{}}
\renewcommand{\sectionmark}[1]{\markright{ \thesection.\ #1}}
\lhead[\fancyplain{}{\bfseries\thepage}]{\fancyplain{}{\bfseries\rightmark}}
\rhead[\fancyplain{}{\bfseries\leftmark}]{\fancyplain{}{\bfseries\thepage}}
\cfoot{}

% indice
\cleardoublepage
\phantomsection
\addcontentsline{toc}{chapter}{Index}
\tableofcontents
% listado de figuras
\cleardoublepage
\phantomsection
\addcontentsline{toc}{chapter}{Figure List}
\listoffigures
% listado de tablas
\cleardoublepage
\phantomsection
\addcontentsline{toc}{chapter}{Table List}
\listoftables

\thispagestyle{empty}

\pagenumbering{arabic}

\pagestyle{fancy}
\renewcommand{\chaptermark}[1]{ \markboth{#1}{}}
\renewcommand{\sectionmark}[1]{\markright{ \thesection.\ #1}}
\lhead[\fancyplain{}{\bfseries\thepage}]{\fancyplain{}{\bfseries\rightmark}}
\rhead[\fancyplain{}{\bfseries\leftmark}]{\fancyplain{}{\bfseries\thepage}}
\cfoot{}

\onehalfspacing

\input{introduction/0_abstract.tex}

\chapter{Introduction}

\section{Context and project justification}

Artificial Intelligence has arrived to change the world in almost (if not all) any field. Today, we are surrounded by (and we are using many) AI products like smartphones’ face recognition capabilities, home cleaning robots or cars with autopilot options.

From the different AI fields, computer vision is maybe the most popular one and the one which is usually used for explaining AI capabilities to general public. Identifying a cat in a picture could perfectly be the example used in every AI presentation to welcome people to AI.

Image processing is intuitively matched with medical diagnosis by anyone having or not any expertise on the field. Every single citizen will have heard of magnetic resonance imaging, and anyone easily transposes ‘cat detection’ to ‘anomaly detection’, being the anomaly a tumor or anything else.

There are different techniques to scan people and create images for clinical diagnosis like X-Ray or Magnetic Resonance Imaging. In this project, we will work with Magnetic Resonance Images (MRI) which are images created by a machine with a large bore that scans people lying inside it. The MR technique is non-invasive, it produces no radiation, and is used to scan almost any part of the body from which we will focus on brain images.

Combining AI diagnosis capabilities on image processing and MRI images, we can think of helping doctors to identify the presence of anomalies or looking for concrete diagnosis for a specific disease.

Obviously, these projects are not easy at all and they face a lot of challenges. One of the first challenges that such AI project faces is the difficulty to obtain a large set of brain images that are needed to train an accurate AI model. Scanning people is too costly and requires a lot of time. and the challenge of having a large dataset gets harder once we understand that brain images may differ depending on age or gender.

Today, there is a clear limitation on how to reproduce or obtain healthy brain images for AI-based diagnosis projects while the appearance of new AI techniques known as Autoencoders and Variational Autoencoders introduces a new area of investigation to mitigate the gap.

This project aims to be a proof of concept to create AI-created images with new variational autoencoders that would serve to augment any existing MRI dataset and that will help to improve the accuracy of brain anomaly detection projects, including those which could be used for overall anomaly detection to others more specific which would help on concrete disease diagnosis.

Personal motivation comes from various angles:

\begin{itemize}
    \item One is to prove myself that I can work with new AI architectures demonstrating that I have acquired the knowledge needed (deep enough) to be productive and to be able to innovate in the health sector.
    \item Deep Learning has been the subject which I enjoyed the most, hence continuing with Autoencoders seems natural to me as the next step on AI adoption
    \item At no doubts, if I can contribute to help on brain issue detection or diagnosis, I will feel my life been completely fulfilled
\end{itemize}
\section{Aim of the project}

The aim of this project is to serve as a Proof of Concept on how MRI brain images can be artificially generated with Variational Autoencoders which ultimately would serve to enhance existing or new datasets to improve model accuracy by having larger samples of data.

Foreseen projects objectives are:

\begin{itemize}
    \item Obtain basic knowledge about MRI images and the NIFTI file format
    \item Obtain and visualise 2-D images from 3-D images in the dataset
    \item Select what range of slices (2-D images) from brain to be created
    \item Test different existing networks and choose the one to be used
    \item Tune network parameters
    \item Compare generated brain images against real ones, qualitatively
\end{itemize}
\section{Project Plan}

\subsection{Resources}

\begin{enumerate}
    \item 1 Data Scientist: Miguel Tablado will be playing this role and will dedicate 300h
    \item 1 Coach/Tutor: Baris Kanber will be assisting Miguel Tablado during the project
    \item MRI images and demographic information from IXI Dataset
    \item GPU resources are needed to train and test the network
\end{enumerate}

\subsection{High Level Plan}

The plan will be executed in 3 different phases with the listed tasks:
\begin{enumerate}
    \item Phase 1: Analysis
    \begin{enumerate}
        \item Gain knowledge on MRI and NIFTI protocol
        \item Describe images and dataset
        \item Extract 2D images
        \item Pre-processing images
    \end{enumerate}
    \item Phase 2: MR Images creation
    \begin{enumerate}
        \item Test different Network architectures
        \item Tune-up architectural network
    \end{enumerate}
    \item Phase 3: Project documentation
    \begin{enumerate}
        \item Write conclusions
        \item Create project documentation
        \item Create project presentation
    \end{enumerate}
\end{enumerate}

\begin{figure}[ht]
    \hspace*{-1.2in}
    \centering
    \includegraphics[width = 20cm, height = 6cm]{images/project-plan.png}
    \caption[]{Project Plan. source: https://www.onlinegantt.com}
    \label{fig:project-plan}
\end{figure}

\newpage
\subsection{Tasks}
\subsubsection*{Phase 1: Analysis}

During this face the different tasks will be executed to prepare the work and includes:
\begin{enumerate}
    \item Gain knowledge on MRI and the NIFTI file format: This activity consists of reading papers and documents to gain sufficient knowledge to execute the project. There is no need to become an expert on the matter but understanding how those files are and how to process them.
    \item Describe images and dataset: During this activity, a description of the dataset will be generated with a view of the quality of the dataset for the aim of the project and any findings which could result.
    \item Extract 2D images: Load 3D images and extract 2D slices from the original dataset, which will be depicted with code.
    \item Pre-processing images: Decide which transformations on the 2D images would help the project like pixel changes or applying gray-scale transformations.
\end{enumerate}

\subsubsection*{Phase 2: MR Images creation}

\begin{enumerate}
    \item Test different Network architectures: This task will take few existing network architectures and be tested with the dataset so that one of them will be selected to be improved and used as the project architecture.
    \item Tune-up architectural network: Tune the selected architecture with useful techniques like changing network layers
\end{enumerate}

\documentclass[11pt,a4paper,openany]{book}

\usepackage{amsmath}
\usepackage{graphicx}
\usepackage{hyperref}
\usepackage[utf8]{inputenc}
\usepackage[acronym, toc]{glossaries}

\usepackage{setspace}	%double spacing for text, single for captions, footnotes, etc.
\usepackage{natbib}		% substituye a 'hypernat' que funciona en Windows.
\usepackage[english]{babel}
\usepackage[utf8]{inputenc}
\usepackage{color}
\usepackage{hhline} 		% extended styles for tables
\usepackage{multirow}
\usepackage{subfigure}
\usepackage{amsmath,amsmath,amssymb} 
\usepackage{fancyhdr}
\usepackage{epsfig, amsmath}
\usepackage{algorithm}
\usepackage{algorithmic}

\usepackage[section]{placeins}

% general settings
\hypersetup{
	linktocpage=true,
	colorlinks=true,
	linkcolor=blue,
	citecolor=blue,
}
\definecolor{Hgray}{gray}{0.6}

\newenvironment{definition}[1][Definition]{\begin{trivlist}
\item[\hskip \labelsep {\bfseries #1}]}{\end{trivlist}}

\setlength{\topmargin}{0cm}
\setlength{\textheight}{23cm}
\setlength{\textwidth}{17cm}
\setlength{\oddsidemargin}{0cm}
\setlength{\evensidemargin}{0cm}
\setlength{\headheight}{1cm}

% indica que las 'sub-sub-sections' sean numeradas y aparezcan en el indice
\setcounter{secnumdepth}{3}
\setcounter{tocdepth}{2}

% settings for code
\renewcommand{\algorithmicrequire}{\textbf{Entrada: }}
\renewcommand{\algorithmicensure}{\textbf{Salida: }}

\makeglossaries
\input{closing/glossary.tex}

\begin{document}

% Portada
\input{0_tittle.tex}


%\tableofcontents

\pagenumbering{roman} 
\setcounter{page}{1} 
\pagestyle{plain}

%%%%%%%%%%%%%
%%% FICHA %%%
%%%%%%%%%%%%%
\chapter*{FICHA DEL TRABAJO FINAL}
\begin{table}[ht]
	\centering{}
	\renewcommand{\arraystretch}{2}
	\begin{tabular}{r | l}
		\hline
		Título del trabajo: & Artificial MRI brain images creation with Variational Autoencoders\\
		\hline
        Nombre del autor: & Miguel Tablado León\\
		\hline
        Nombre del Tutor/a de TF: & Baris Kanber\\
		\hline
        Nombre del/de la PRA: & Ferran Prados Carrasco\\
		\hline
        Fecha de entrega: & 02/2023\\
		\hline
        Titulación o programa: & Máster en Ciencia de Datos\\
		\hline
        Área del Trabajo Final: & Medicine Area (TFM-Med)\\
		\hline
        Language: & English\\
		\hline
        Keywords & Deep Learning, Brain MRI, Variational Autoencoder\\
		\hline
	\end{tabular}
\end{table}

\pagestyle{fancy}
\renewcommand{\chaptermark}[1]{ \markboth{#1}{}}
\renewcommand{\sectionmark}[1]{\markright{ \thesection.\ #1}}
\lhead[\fancyplain{}{\bfseries\thepage}]{\fancyplain{}{\bfseries\rightmark}}
\rhead[\fancyplain{}{\bfseries\leftmark}]{\fancyplain{}{\bfseries\thepage}}
\cfoot{}

% indice
\cleardoublepage
\phantomsection
\addcontentsline{toc}{chapter}{Index}
\tableofcontents
% listado de figuras
\cleardoublepage
\phantomsection
\addcontentsline{toc}{chapter}{Figure List}
\listoffigures
% listado de tablas
\cleardoublepage
\phantomsection
\addcontentsline{toc}{chapter}{Table List}
\listoftables

\thispagestyle{empty}

\pagenumbering{arabic}

\pagestyle{fancy}
\renewcommand{\chaptermark}[1]{ \markboth{#1}{}}
\renewcommand{\sectionmark}[1]{\markright{ \thesection.\ #1}}
\lhead[\fancyplain{}{\bfseries\thepage}]{\fancyplain{}{\bfseries\rightmark}}
\rhead[\fancyplain{}{\bfseries\leftmark}]{\fancyplain{}{\bfseries\thepage}}
\cfoot{}

\onehalfspacing

\input{introduction/0_abstract.tex}

\chapter{Introduction}

\input{introduction/1_context.tex}
\input{introduction/2_aim.tex}
\input{introduction/5_plan.tex}

\input{state_of_art/main.tex}
\input{impl/main.tex}
\input{conclusions/main.tex}

\clearpage
\printglossary[type=\acronymtype]
\printglossary

\bibliographystyle{plain} % We choose the "plain" reference style
\bibliography{refs}

\end{document}

\documentclass[11pt,a4paper,openany]{book}

\usepackage{amsmath}
\usepackage{graphicx}
\usepackage{hyperref}
\usepackage[utf8]{inputenc}
\usepackage[acronym, toc]{glossaries}

\usepackage{setspace}	%double spacing for text, single for captions, footnotes, etc.
\usepackage{natbib}		% substituye a 'hypernat' que funciona en Windows.
\usepackage[english]{babel}
\usepackage[utf8]{inputenc}
\usepackage{color}
\usepackage{hhline} 		% extended styles for tables
\usepackage{multirow}
\usepackage{subfigure}
\usepackage{amsmath,amsmath,amssymb} 
\usepackage{fancyhdr}
\usepackage{epsfig, amsmath}
\usepackage{algorithm}
\usepackage{algorithmic}

\usepackage[section]{placeins}

% general settings
\hypersetup{
	linktocpage=true,
	colorlinks=true,
	linkcolor=blue,
	citecolor=blue,
}
\definecolor{Hgray}{gray}{0.6}

\newenvironment{definition}[1][Definition]{\begin{trivlist}
\item[\hskip \labelsep {\bfseries #1}]}{\end{trivlist}}

\setlength{\topmargin}{0cm}
\setlength{\textheight}{23cm}
\setlength{\textwidth}{17cm}
\setlength{\oddsidemargin}{0cm}
\setlength{\evensidemargin}{0cm}
\setlength{\headheight}{1cm}

% indica que las 'sub-sub-sections' sean numeradas y aparezcan en el indice
\setcounter{secnumdepth}{3}
\setcounter{tocdepth}{2}

% settings for code
\renewcommand{\algorithmicrequire}{\textbf{Entrada: }}
\renewcommand{\algorithmicensure}{\textbf{Salida: }}

\makeglossaries
\input{closing/glossary.tex}

\begin{document}

% Portada
\input{0_tittle.tex}


%\tableofcontents

\pagenumbering{roman} 
\setcounter{page}{1} 
\pagestyle{plain}

%%%%%%%%%%%%%
%%% FICHA %%%
%%%%%%%%%%%%%
\chapter*{FICHA DEL TRABAJO FINAL}
\begin{table}[ht]
	\centering{}
	\renewcommand{\arraystretch}{2}
	\begin{tabular}{r | l}
		\hline
		Título del trabajo: & Artificial MRI brain images creation with Variational Autoencoders\\
		\hline
        Nombre del autor: & Miguel Tablado León\\
		\hline
        Nombre del Tutor/a de TF: & Baris Kanber\\
		\hline
        Nombre del/de la PRA: & Ferran Prados Carrasco\\
		\hline
        Fecha de entrega: & 02/2023\\
		\hline
        Titulación o programa: & Máster en Ciencia de Datos\\
		\hline
        Área del Trabajo Final: & Medicine Area (TFM-Med)\\
		\hline
        Language: & English\\
		\hline
        Keywords & Deep Learning, Brain MRI, Variational Autoencoder\\
		\hline
	\end{tabular}
\end{table}

\pagestyle{fancy}
\renewcommand{\chaptermark}[1]{ \markboth{#1}{}}
\renewcommand{\sectionmark}[1]{\markright{ \thesection.\ #1}}
\lhead[\fancyplain{}{\bfseries\thepage}]{\fancyplain{}{\bfseries\rightmark}}
\rhead[\fancyplain{}{\bfseries\leftmark}]{\fancyplain{}{\bfseries\thepage}}
\cfoot{}

% indice
\cleardoublepage
\phantomsection
\addcontentsline{toc}{chapter}{Index}
\tableofcontents
% listado de figuras
\cleardoublepage
\phantomsection
\addcontentsline{toc}{chapter}{Figure List}
\listoffigures
% listado de tablas
\cleardoublepage
\phantomsection
\addcontentsline{toc}{chapter}{Table List}
\listoftables

\thispagestyle{empty}

\pagenumbering{arabic}

\pagestyle{fancy}
\renewcommand{\chaptermark}[1]{ \markboth{#1}{}}
\renewcommand{\sectionmark}[1]{\markright{ \thesection.\ #1}}
\lhead[\fancyplain{}{\bfseries\thepage}]{\fancyplain{}{\bfseries\rightmark}}
\rhead[\fancyplain{}{\bfseries\leftmark}]{\fancyplain{}{\bfseries\thepage}}
\cfoot{}

\onehalfspacing

\input{introduction/0_abstract.tex}

\chapter{Introduction}

\input{introduction/1_context.tex}
\input{introduction/2_aim.tex}
\input{introduction/5_plan.tex}

\input{state_of_art/main.tex}
\input{impl/main.tex}
\input{conclusions/main.tex}

\clearpage
\printglossary[type=\acronymtype]
\printglossary

\bibliographystyle{plain} % We choose the "plain" reference style
\bibliography{refs}

\end{document}

\documentclass[11pt,a4paper,openany]{book}

\usepackage{amsmath}
\usepackage{graphicx}
\usepackage{hyperref}
\usepackage[utf8]{inputenc}
\usepackage[acronym, toc]{glossaries}

\usepackage{setspace}	%double spacing for text, single for captions, footnotes, etc.
\usepackage{natbib}		% substituye a 'hypernat' que funciona en Windows.
\usepackage[english]{babel}
\usepackage[utf8]{inputenc}
\usepackage{color}
\usepackage{hhline} 		% extended styles for tables
\usepackage{multirow}
\usepackage{subfigure}
\usepackage{amsmath,amsmath,amssymb} 
\usepackage{fancyhdr}
\usepackage{epsfig, amsmath}
\usepackage{algorithm}
\usepackage{algorithmic}

\usepackage[section]{placeins}

% general settings
\hypersetup{
	linktocpage=true,
	colorlinks=true,
	linkcolor=blue,
	citecolor=blue,
}
\definecolor{Hgray}{gray}{0.6}

\newenvironment{definition}[1][Definition]{\begin{trivlist}
\item[\hskip \labelsep {\bfseries #1}]}{\end{trivlist}}

\setlength{\topmargin}{0cm}
\setlength{\textheight}{23cm}
\setlength{\textwidth}{17cm}
\setlength{\oddsidemargin}{0cm}
\setlength{\evensidemargin}{0cm}
\setlength{\headheight}{1cm}

% indica que las 'sub-sub-sections' sean numeradas y aparezcan en el indice
\setcounter{secnumdepth}{3}
\setcounter{tocdepth}{2}

% settings for code
\renewcommand{\algorithmicrequire}{\textbf{Entrada: }}
\renewcommand{\algorithmicensure}{\textbf{Salida: }}

\makeglossaries
\input{closing/glossary.tex}

\begin{document}

% Portada
\input{0_tittle.tex}


%\tableofcontents

\pagenumbering{roman} 
\setcounter{page}{1} 
\pagestyle{plain}

%%%%%%%%%%%%%
%%% FICHA %%%
%%%%%%%%%%%%%
\chapter*{FICHA DEL TRABAJO FINAL}
\begin{table}[ht]
	\centering{}
	\renewcommand{\arraystretch}{2}
	\begin{tabular}{r | l}
		\hline
		Título del trabajo: & Artificial MRI brain images creation with Variational Autoencoders\\
		\hline
        Nombre del autor: & Miguel Tablado León\\
		\hline
        Nombre del Tutor/a de TF: & Baris Kanber\\
		\hline
        Nombre del/de la PRA: & Ferran Prados Carrasco\\
		\hline
        Fecha de entrega: & 02/2023\\
		\hline
        Titulación o programa: & Máster en Ciencia de Datos\\
		\hline
        Área del Trabajo Final: & Medicine Area (TFM-Med)\\
		\hline
        Language: & English\\
		\hline
        Keywords & Deep Learning, Brain MRI, Variational Autoencoder\\
		\hline
	\end{tabular}
\end{table}

\pagestyle{fancy}
\renewcommand{\chaptermark}[1]{ \markboth{#1}{}}
\renewcommand{\sectionmark}[1]{\markright{ \thesection.\ #1}}
\lhead[\fancyplain{}{\bfseries\thepage}]{\fancyplain{}{\bfseries\rightmark}}
\rhead[\fancyplain{}{\bfseries\leftmark}]{\fancyplain{}{\bfseries\thepage}}
\cfoot{}

% indice
\cleardoublepage
\phantomsection
\addcontentsline{toc}{chapter}{Index}
\tableofcontents
% listado de figuras
\cleardoublepage
\phantomsection
\addcontentsline{toc}{chapter}{Figure List}
\listoffigures
% listado de tablas
\cleardoublepage
\phantomsection
\addcontentsline{toc}{chapter}{Table List}
\listoftables

\thispagestyle{empty}

\pagenumbering{arabic}

\pagestyle{fancy}
\renewcommand{\chaptermark}[1]{ \markboth{#1}{}}
\renewcommand{\sectionmark}[1]{\markright{ \thesection.\ #1}}
\lhead[\fancyplain{}{\bfseries\thepage}]{\fancyplain{}{\bfseries\rightmark}}
\rhead[\fancyplain{}{\bfseries\leftmark}]{\fancyplain{}{\bfseries\thepage}}
\cfoot{}

\onehalfspacing

\input{introduction/0_abstract.tex}

\chapter{Introduction}

\input{introduction/1_context.tex}
\input{introduction/2_aim.tex}
\input{introduction/5_plan.tex}

\input{state_of_art/main.tex}
\input{impl/main.tex}
\input{conclusions/main.tex}

\clearpage
\printglossary[type=\acronymtype]
\printglossary

\bibliographystyle{plain} % We choose the "plain" reference style
\bibliography{refs}

\end{document}


\clearpage
\printglossary[type=\acronymtype]
\printglossary

\bibliographystyle{plain} % We choose the "plain" reference style
\bibliography{refs}

\end{document}

\documentclass[11pt,a4paper,openany]{book}

\usepackage{amsmath}
\usepackage{graphicx}
\usepackage{hyperref}
\usepackage[utf8]{inputenc}
\usepackage[acronym, toc]{glossaries}

\usepackage{setspace}	%double spacing for text, single for captions, footnotes, etc.
\usepackage{natbib}		% substituye a 'hypernat' que funciona en Windows.
\usepackage[english]{babel}
\usepackage[utf8]{inputenc}
\usepackage{color}
\usepackage{hhline} 		% extended styles for tables
\usepackage{multirow}
\usepackage{subfigure}
\usepackage{amsmath,amsmath,amssymb} 
\usepackage{fancyhdr}
\usepackage{epsfig, amsmath}
\usepackage{algorithm}
\usepackage{algorithmic}

\usepackage[section]{placeins}

% general settings
\hypersetup{
	linktocpage=true,
	colorlinks=true,
	linkcolor=blue,
	citecolor=blue,
}
\definecolor{Hgray}{gray}{0.6}

\newenvironment{definition}[1][Definition]{\begin{trivlist}
\item[\hskip \labelsep {\bfseries #1}]}{\end{trivlist}}

\setlength{\topmargin}{0cm}
\setlength{\textheight}{23cm}
\setlength{\textwidth}{17cm}
\setlength{\oddsidemargin}{0cm}
\setlength{\evensidemargin}{0cm}
\setlength{\headheight}{1cm}

% indica que las 'sub-sub-sections' sean numeradas y aparezcan en el indice
\setcounter{secnumdepth}{3}
\setcounter{tocdepth}{2}

% settings for code
\renewcommand{\algorithmicrequire}{\textbf{Entrada: }}
\renewcommand{\algorithmicensure}{\textbf{Salida: }}

\makeglossaries
\newglossaryentry{kpi}
{
    name=KPI,
    description={Key Performance Indicator}
}

\newglossaryentry{mnist}
{
    name=MNIST,
    description={The MNIST database of handwritten digits used for training models}
}

\newglossaryentry{kdd}
{
    name=KDD,
    description={KDD cup 1999 network intrusion dataset}
}

\newglossaryentry{ixi}
{
    name=IXI,
    description={IXI dataset is a collection of 600 Brain MR Images, in 3-D, collected from 3 different hospitals in London. The images are scanned in 1.5 and 3 Tesla which is the range of the quality that we could usually find in hospitals around the world}
}

\newacronym{pca}{PCA}{Principal Component Analysis}

\newacronym{mri}{MRI}{Magnetic Resonance Image}

\newacronym{vqvae}{VQ-VAE}{Vector-Quantized Variational Autoencoder}

\newacronym{cevae}{ceVAE}{Context-Encoder Variational Autoencoder}

\newacronym{strega}{StRegA}{Segmentation Regularised Anomaly}


\begin{document}

% Portada
\newpage
\thispagestyle{empty}

\baselineskip 2em

%\vspace*{1cm}

\centerline{\includegraphics[width=0.6\textwidth]{images/UOC-logo}}
\begin{center}
\textsc{Universitat Oberta de Catalunya (UOC) \\
 Máster Universitario en Ciencia de Datos (\textit{Data Science})\\}

%\centerline {\pic{UOC}{4cm}}

\vspace*{1.5cm}

\textsc{\Large TRABAJO FINAL DE MÁSTER}

\vspace*{0.5cm}

\textsc{\large Área: YYY}


%\textbf{\Huge VirtualTechLab Model: }

\vspace*{2.0cm}

\title{\Large Artificial MRI brain images creation with Variational Autoencoders}

\vspace{2.5cm}
\baselineskip 1em

\baselineskip 2em
-----------------------------------------------------------------------------\\
Autor:      Miguel Tablado\\
Tutor:      Baris Kanber\\
Profesor:   Nombre del profesor responsable del área de TF\\
-----------------------------------------------------------------------------\\
\vspace*{1.5cm}
Barcelona, \today

\author{Miguel Tablado}

\end{center}



%\tableofcontents

\pagenumbering{roman} 
\setcounter{page}{1} 
\pagestyle{plain}

%%%%%%%%%%%%%
%%% FICHA %%%
%%%%%%%%%%%%%
\chapter*{FICHA DEL TRABAJO FINAL}
\begin{table}[ht]
	\centering{}
	\renewcommand{\arraystretch}{2}
	\begin{tabular}{r | l}
		\hline
		Título del trabajo: & Artificial MRI brain images creation with Variational Autoencoders\\
		\hline
        Nombre del autor: & Miguel Tablado León\\
		\hline
        Nombre del Tutor/a de TF: & Baris Kanber\\
		\hline
        Nombre del/de la PRA: & Ferran Prados Carrasco\\
		\hline
        Fecha de entrega: & 02/2023\\
		\hline
        Titulación o programa: & Máster en Ciencia de Datos\\
		\hline
        Área del Trabajo Final: & Medicine Area (TFM-Med)\\
		\hline
        Language: & English\\
		\hline
        Keywords & Deep Learning, Brain MRI, Variational Autoencoder\\
		\hline
	\end{tabular}
\end{table}

\pagestyle{fancy}
\renewcommand{\chaptermark}[1]{ \markboth{#1}{}}
\renewcommand{\sectionmark}[1]{\markright{ \thesection.\ #1}}
\lhead[\fancyplain{}{\bfseries\thepage}]{\fancyplain{}{\bfseries\rightmark}}
\rhead[\fancyplain{}{\bfseries\leftmark}]{\fancyplain{}{\bfseries\thepage}}
\cfoot{}

% indice
\cleardoublepage
\phantomsection
\addcontentsline{toc}{chapter}{Index}
\tableofcontents
% listado de figuras
\cleardoublepage
\phantomsection
\addcontentsline{toc}{chapter}{Figure List}
\listoffigures
% listado de tablas
\cleardoublepage
\phantomsection
\addcontentsline{toc}{chapter}{Table List}
\listoftables

\thispagestyle{empty}

\pagenumbering{arabic}

\pagestyle{fancy}
\renewcommand{\chaptermark}[1]{ \markboth{#1}{}}
\renewcommand{\sectionmark}[1]{\markright{ \thesection.\ #1}}
\lhead[\fancyplain{}{\bfseries\thepage}]{\fancyplain{}{\bfseries\rightmark}}
\rhead[\fancyplain{}{\bfseries\leftmark}]{\fancyplain{}{\bfseries\thepage}}
\cfoot{}

\onehalfspacing

\input{introduction/0_abstract.tex}

\chapter{Introduction}

\section{Context and project justification}

Artificial Intelligence has arrived to change the world in almost (if not all) any field. Today, we are surrounded by (and we are using many) AI products like smartphones’ face recognition capabilities, home cleaning robots or cars with autopilot options.

From the different AI fields, computer vision is maybe the most popular one and the one which is usually used for explaining AI capabilities to general public. Identifying a cat in a picture could perfectly be the example used in every AI presentation to welcome people to AI.

Image processing is intuitively matched with medical diagnosis by anyone having or not any expertise on the field. Every single citizen will have heard of magnetic resonance imaging, and anyone easily transposes ‘cat detection’ to ‘anomaly detection’, being the anomaly a tumor or anything else.

There are different techniques to scan people and create images for clinical diagnosis like X-Ray or Magnetic Resonance Imaging. In this project, we will work with Magnetic Resonance Images (MRI) which are images created by a machine with a large bore that scans people lying inside it. The MR technique is non-invasive, it produces no radiation, and is used to scan almost any part of the body from which we will focus on brain images.

Combining AI diagnosis capabilities on image processing and MRI images, we can think of helping doctors to identify the presence of anomalies or looking for concrete diagnosis for a specific disease.

Obviously, these projects are not easy at all and they face a lot of challenges. One of the first challenges that such AI project faces is the difficulty to obtain a large set of brain images that are needed to train an accurate AI model. Scanning people is too costly and requires a lot of time. and the challenge of having a large dataset gets harder once we understand that brain images may differ depending on age or gender.

Today, there is a clear limitation on how to reproduce or obtain healthy brain images for AI-based diagnosis projects while the appearance of new AI techniques known as Autoencoders and Variational Autoencoders introduces a new area of investigation to mitigate the gap.

This project aims to be a proof of concept to create AI-created images with new variational autoencoders that would serve to augment any existing MRI dataset and that will help to improve the accuracy of brain anomaly detection projects, including those which could be used for overall anomaly detection to others more specific which would help on concrete disease diagnosis.

Personal motivation comes from various angles:

\begin{itemize}
    \item One is to prove myself that I can work with new AI architectures demonstrating that I have acquired the knowledge needed (deep enough) to be productive and to be able to innovate in the health sector.
    \item Deep Learning has been the subject which I enjoyed the most, hence continuing with Autoencoders seems natural to me as the next step on AI adoption
    \item At no doubts, if I can contribute to help on brain issue detection or diagnosis, I will feel my life been completely fulfilled
\end{itemize}
\section{Aim of the project}

The aim of this project is to serve as a Proof of Concept on how MRI brain images can be artificially generated with Variational Autoencoders which ultimately would serve to enhance existing or new datasets to improve model accuracy by having larger samples of data.

Foreseen projects objectives are:

\begin{itemize}
    \item Obtain basic knowledge about MRI images and the NIFTI file format
    \item Obtain and visualise 2-D images from 3-D images in the dataset
    \item Select what range of slices (2-D images) from brain to be created
    \item Test different existing networks and choose the one to be used
    \item Tune network parameters
    \item Compare generated brain images against real ones, qualitatively
\end{itemize}
\section{Project Plan}

\subsection{Resources}

\begin{enumerate}
    \item 1 Data Scientist: Miguel Tablado will be playing this role and will dedicate 300h
    \item 1 Coach/Tutor: Baris Kanber will be assisting Miguel Tablado during the project
    \item MRI images and demographic information from IXI Dataset
    \item GPU resources are needed to train and test the network
\end{enumerate}

\subsection{High Level Plan}

The plan will be executed in 3 different phases with the listed tasks:
\begin{enumerate}
    \item Phase 1: Analysis
    \begin{enumerate}
        \item Gain knowledge on MRI and NIFTI protocol
        \item Describe images and dataset
        \item Extract 2D images
        \item Pre-processing images
    \end{enumerate}
    \item Phase 2: MR Images creation
    \begin{enumerate}
        \item Test different Network architectures
        \item Tune-up architectural network
    \end{enumerate}
    \item Phase 3: Project documentation
    \begin{enumerate}
        \item Write conclusions
        \item Create project documentation
        \item Create project presentation
    \end{enumerate}
\end{enumerate}

\begin{figure}[ht]
    \hspace*{-1.2in}
    \centering
    \includegraphics[width = 20cm, height = 6cm]{images/project-plan.png}
    \caption[]{Project Plan. source: https://www.onlinegantt.com}
    \label{fig:project-plan}
\end{figure}

\newpage
\subsection{Tasks}
\subsubsection*{Phase 1: Analysis}

During this face the different tasks will be executed to prepare the work and includes:
\begin{enumerate}
    \item Gain knowledge on MRI and the NIFTI file format: This activity consists of reading papers and documents to gain sufficient knowledge to execute the project. There is no need to become an expert on the matter but understanding how those files are and how to process them.
    \item Describe images and dataset: During this activity, a description of the dataset will be generated with a view of the quality of the dataset for the aim of the project and any findings which could result.
    \item Extract 2D images: Load 3D images and extract 2D slices from the original dataset, which will be depicted with code.
    \item Pre-processing images: Decide which transformations on the 2D images would help the project like pixel changes or applying gray-scale transformations.
\end{enumerate}

\subsubsection*{Phase 2: MR Images creation}

\begin{enumerate}
    \item Test different Network architectures: This task will take few existing network architectures and be tested with the dataset so that one of them will be selected to be improved and used as the project architecture.
    \item Tune-up architectural network: Tune the selected architecture with useful techniques like changing network layers
\end{enumerate}

\documentclass[11pt,a4paper,openany]{book}

\usepackage{amsmath}
\usepackage{graphicx}
\usepackage{hyperref}
\usepackage[utf8]{inputenc}
\usepackage[acronym, toc]{glossaries}

\usepackage{setspace}	%double spacing for text, single for captions, footnotes, etc.
\usepackage{natbib}		% substituye a 'hypernat' que funciona en Windows.
\usepackage[english]{babel}
\usepackage[utf8]{inputenc}
\usepackage{color}
\usepackage{hhline} 		% extended styles for tables
\usepackage{multirow}
\usepackage{subfigure}
\usepackage{amsmath,amsmath,amssymb} 
\usepackage{fancyhdr}
\usepackage{epsfig, amsmath}
\usepackage{algorithm}
\usepackage{algorithmic}

\usepackage[section]{placeins}

% general settings
\hypersetup{
	linktocpage=true,
	colorlinks=true,
	linkcolor=blue,
	citecolor=blue,
}
\definecolor{Hgray}{gray}{0.6}

\newenvironment{definition}[1][Definition]{\begin{trivlist}
\item[\hskip \labelsep {\bfseries #1}]}{\end{trivlist}}

\setlength{\topmargin}{0cm}
\setlength{\textheight}{23cm}
\setlength{\textwidth}{17cm}
\setlength{\oddsidemargin}{0cm}
\setlength{\evensidemargin}{0cm}
\setlength{\headheight}{1cm}

% indica que las 'sub-sub-sections' sean numeradas y aparezcan en el indice
\setcounter{secnumdepth}{3}
\setcounter{tocdepth}{2}

% settings for code
\renewcommand{\algorithmicrequire}{\textbf{Entrada: }}
\renewcommand{\algorithmicensure}{\textbf{Salida: }}

\makeglossaries
\input{closing/glossary.tex}

\begin{document}

% Portada
\input{0_tittle.tex}


%\tableofcontents

\pagenumbering{roman} 
\setcounter{page}{1} 
\pagestyle{plain}

%%%%%%%%%%%%%
%%% FICHA %%%
%%%%%%%%%%%%%
\chapter*{FICHA DEL TRABAJO FINAL}
\begin{table}[ht]
	\centering{}
	\renewcommand{\arraystretch}{2}
	\begin{tabular}{r | l}
		\hline
		Título del trabajo: & Artificial MRI brain images creation with Variational Autoencoders\\
		\hline
        Nombre del autor: & Miguel Tablado León\\
		\hline
        Nombre del Tutor/a de TF: & Baris Kanber\\
		\hline
        Nombre del/de la PRA: & Ferran Prados Carrasco\\
		\hline
        Fecha de entrega: & 02/2023\\
		\hline
        Titulación o programa: & Máster en Ciencia de Datos\\
		\hline
        Área del Trabajo Final: & Medicine Area (TFM-Med)\\
		\hline
        Language: & English\\
		\hline
        Keywords & Deep Learning, Brain MRI, Variational Autoencoder\\
		\hline
	\end{tabular}
\end{table}

\pagestyle{fancy}
\renewcommand{\chaptermark}[1]{ \markboth{#1}{}}
\renewcommand{\sectionmark}[1]{\markright{ \thesection.\ #1}}
\lhead[\fancyplain{}{\bfseries\thepage}]{\fancyplain{}{\bfseries\rightmark}}
\rhead[\fancyplain{}{\bfseries\leftmark}]{\fancyplain{}{\bfseries\thepage}}
\cfoot{}

% indice
\cleardoublepage
\phantomsection
\addcontentsline{toc}{chapter}{Index}
\tableofcontents
% listado de figuras
\cleardoublepage
\phantomsection
\addcontentsline{toc}{chapter}{Figure List}
\listoffigures
% listado de tablas
\cleardoublepage
\phantomsection
\addcontentsline{toc}{chapter}{Table List}
\listoftables

\thispagestyle{empty}

\pagenumbering{arabic}

\pagestyle{fancy}
\renewcommand{\chaptermark}[1]{ \markboth{#1}{}}
\renewcommand{\sectionmark}[1]{\markright{ \thesection.\ #1}}
\lhead[\fancyplain{}{\bfseries\thepage}]{\fancyplain{}{\bfseries\rightmark}}
\rhead[\fancyplain{}{\bfseries\leftmark}]{\fancyplain{}{\bfseries\thepage}}
\cfoot{}

\onehalfspacing

\input{introduction/0_abstract.tex}

\chapter{Introduction}

\input{introduction/1_context.tex}
\input{introduction/2_aim.tex}
\input{introduction/5_plan.tex}

\input{state_of_art/main.tex}
\input{impl/main.tex}
\input{conclusions/main.tex}

\clearpage
\printglossary[type=\acronymtype]
\printglossary

\bibliographystyle{plain} % We choose the "plain" reference style
\bibliography{refs}

\end{document}

\documentclass[11pt,a4paper,openany]{book}

\usepackage{amsmath}
\usepackage{graphicx}
\usepackage{hyperref}
\usepackage[utf8]{inputenc}
\usepackage[acronym, toc]{glossaries}

\usepackage{setspace}	%double spacing for text, single for captions, footnotes, etc.
\usepackage{natbib}		% substituye a 'hypernat' que funciona en Windows.
\usepackage[english]{babel}
\usepackage[utf8]{inputenc}
\usepackage{color}
\usepackage{hhline} 		% extended styles for tables
\usepackage{multirow}
\usepackage{subfigure}
\usepackage{amsmath,amsmath,amssymb} 
\usepackage{fancyhdr}
\usepackage{epsfig, amsmath}
\usepackage{algorithm}
\usepackage{algorithmic}

\usepackage[section]{placeins}

% general settings
\hypersetup{
	linktocpage=true,
	colorlinks=true,
	linkcolor=blue,
	citecolor=blue,
}
\definecolor{Hgray}{gray}{0.6}

\newenvironment{definition}[1][Definition]{\begin{trivlist}
\item[\hskip \labelsep {\bfseries #1}]}{\end{trivlist}}

\setlength{\topmargin}{0cm}
\setlength{\textheight}{23cm}
\setlength{\textwidth}{17cm}
\setlength{\oddsidemargin}{0cm}
\setlength{\evensidemargin}{0cm}
\setlength{\headheight}{1cm}

% indica que las 'sub-sub-sections' sean numeradas y aparezcan en el indice
\setcounter{secnumdepth}{3}
\setcounter{tocdepth}{2}

% settings for code
\renewcommand{\algorithmicrequire}{\textbf{Entrada: }}
\renewcommand{\algorithmicensure}{\textbf{Salida: }}

\makeglossaries
\input{closing/glossary.tex}

\begin{document}

% Portada
\input{0_tittle.tex}


%\tableofcontents

\pagenumbering{roman} 
\setcounter{page}{1} 
\pagestyle{plain}

%%%%%%%%%%%%%
%%% FICHA %%%
%%%%%%%%%%%%%
\chapter*{FICHA DEL TRABAJO FINAL}
\begin{table}[ht]
	\centering{}
	\renewcommand{\arraystretch}{2}
	\begin{tabular}{r | l}
		\hline
		Título del trabajo: & Artificial MRI brain images creation with Variational Autoencoders\\
		\hline
        Nombre del autor: & Miguel Tablado León\\
		\hline
        Nombre del Tutor/a de TF: & Baris Kanber\\
		\hline
        Nombre del/de la PRA: & Ferran Prados Carrasco\\
		\hline
        Fecha de entrega: & 02/2023\\
		\hline
        Titulación o programa: & Máster en Ciencia de Datos\\
		\hline
        Área del Trabajo Final: & Medicine Area (TFM-Med)\\
		\hline
        Language: & English\\
		\hline
        Keywords & Deep Learning, Brain MRI, Variational Autoencoder\\
		\hline
	\end{tabular}
\end{table}

\pagestyle{fancy}
\renewcommand{\chaptermark}[1]{ \markboth{#1}{}}
\renewcommand{\sectionmark}[1]{\markright{ \thesection.\ #1}}
\lhead[\fancyplain{}{\bfseries\thepage}]{\fancyplain{}{\bfseries\rightmark}}
\rhead[\fancyplain{}{\bfseries\leftmark}]{\fancyplain{}{\bfseries\thepage}}
\cfoot{}

% indice
\cleardoublepage
\phantomsection
\addcontentsline{toc}{chapter}{Index}
\tableofcontents
% listado de figuras
\cleardoublepage
\phantomsection
\addcontentsline{toc}{chapter}{Figure List}
\listoffigures
% listado de tablas
\cleardoublepage
\phantomsection
\addcontentsline{toc}{chapter}{Table List}
\listoftables

\thispagestyle{empty}

\pagenumbering{arabic}

\pagestyle{fancy}
\renewcommand{\chaptermark}[1]{ \markboth{#1}{}}
\renewcommand{\sectionmark}[1]{\markright{ \thesection.\ #1}}
\lhead[\fancyplain{}{\bfseries\thepage}]{\fancyplain{}{\bfseries\rightmark}}
\rhead[\fancyplain{}{\bfseries\leftmark}]{\fancyplain{}{\bfseries\thepage}}
\cfoot{}

\onehalfspacing

\input{introduction/0_abstract.tex}

\chapter{Introduction}

\input{introduction/1_context.tex}
\input{introduction/2_aim.tex}
\input{introduction/5_plan.tex}

\input{state_of_art/main.tex}
\input{impl/main.tex}
\input{conclusions/main.tex}

\clearpage
\printglossary[type=\acronymtype]
\printglossary

\bibliographystyle{plain} % We choose the "plain" reference style
\bibliography{refs}

\end{document}

\documentclass[11pt,a4paper,openany]{book}

\usepackage{amsmath}
\usepackage{graphicx}
\usepackage{hyperref}
\usepackage[utf8]{inputenc}
\usepackage[acronym, toc]{glossaries}

\usepackage{setspace}	%double spacing for text, single for captions, footnotes, etc.
\usepackage{natbib}		% substituye a 'hypernat' que funciona en Windows.
\usepackage[english]{babel}
\usepackage[utf8]{inputenc}
\usepackage{color}
\usepackage{hhline} 		% extended styles for tables
\usepackage{multirow}
\usepackage{subfigure}
\usepackage{amsmath,amsmath,amssymb} 
\usepackage{fancyhdr}
\usepackage{epsfig, amsmath}
\usepackage{algorithm}
\usepackage{algorithmic}

\usepackage[section]{placeins}

% general settings
\hypersetup{
	linktocpage=true,
	colorlinks=true,
	linkcolor=blue,
	citecolor=blue,
}
\definecolor{Hgray}{gray}{0.6}

\newenvironment{definition}[1][Definition]{\begin{trivlist}
\item[\hskip \labelsep {\bfseries #1}]}{\end{trivlist}}

\setlength{\topmargin}{0cm}
\setlength{\textheight}{23cm}
\setlength{\textwidth}{17cm}
\setlength{\oddsidemargin}{0cm}
\setlength{\evensidemargin}{0cm}
\setlength{\headheight}{1cm}

% indica que las 'sub-sub-sections' sean numeradas y aparezcan en el indice
\setcounter{secnumdepth}{3}
\setcounter{tocdepth}{2}

% settings for code
\renewcommand{\algorithmicrequire}{\textbf{Entrada: }}
\renewcommand{\algorithmicensure}{\textbf{Salida: }}

\makeglossaries
\input{closing/glossary.tex}

\begin{document}

% Portada
\input{0_tittle.tex}


%\tableofcontents

\pagenumbering{roman} 
\setcounter{page}{1} 
\pagestyle{plain}

%%%%%%%%%%%%%
%%% FICHA %%%
%%%%%%%%%%%%%
\chapter*{FICHA DEL TRABAJO FINAL}
\begin{table}[ht]
	\centering{}
	\renewcommand{\arraystretch}{2}
	\begin{tabular}{r | l}
		\hline
		Título del trabajo: & Artificial MRI brain images creation with Variational Autoencoders\\
		\hline
        Nombre del autor: & Miguel Tablado León\\
		\hline
        Nombre del Tutor/a de TF: & Baris Kanber\\
		\hline
        Nombre del/de la PRA: & Ferran Prados Carrasco\\
		\hline
        Fecha de entrega: & 02/2023\\
		\hline
        Titulación o programa: & Máster en Ciencia de Datos\\
		\hline
        Área del Trabajo Final: & Medicine Area (TFM-Med)\\
		\hline
        Language: & English\\
		\hline
        Keywords & Deep Learning, Brain MRI, Variational Autoencoder\\
		\hline
	\end{tabular}
\end{table}

\pagestyle{fancy}
\renewcommand{\chaptermark}[1]{ \markboth{#1}{}}
\renewcommand{\sectionmark}[1]{\markright{ \thesection.\ #1}}
\lhead[\fancyplain{}{\bfseries\thepage}]{\fancyplain{}{\bfseries\rightmark}}
\rhead[\fancyplain{}{\bfseries\leftmark}]{\fancyplain{}{\bfseries\thepage}}
\cfoot{}

% indice
\cleardoublepage
\phantomsection
\addcontentsline{toc}{chapter}{Index}
\tableofcontents
% listado de figuras
\cleardoublepage
\phantomsection
\addcontentsline{toc}{chapter}{Figure List}
\listoffigures
% listado de tablas
\cleardoublepage
\phantomsection
\addcontentsline{toc}{chapter}{Table List}
\listoftables

\thispagestyle{empty}

\pagenumbering{arabic}

\pagestyle{fancy}
\renewcommand{\chaptermark}[1]{ \markboth{#1}{}}
\renewcommand{\sectionmark}[1]{\markright{ \thesection.\ #1}}
\lhead[\fancyplain{}{\bfseries\thepage}]{\fancyplain{}{\bfseries\rightmark}}
\rhead[\fancyplain{}{\bfseries\leftmark}]{\fancyplain{}{\bfseries\thepage}}
\cfoot{}

\onehalfspacing

\input{introduction/0_abstract.tex}

\chapter{Introduction}

\input{introduction/1_context.tex}
\input{introduction/2_aim.tex}
\input{introduction/5_plan.tex}

\input{state_of_art/main.tex}
\input{impl/main.tex}
\input{conclusions/main.tex}

\clearpage
\printglossary[type=\acronymtype]
\printglossary

\bibliographystyle{plain} % We choose the "plain" reference style
\bibliography{refs}

\end{document}


\clearpage
\printglossary[type=\acronymtype]
\printglossary

\bibliographystyle{plain} % We choose the "plain" reference style
\bibliography{refs}

\end{document}

\documentclass[11pt,a4paper,openany]{book}

\usepackage{amsmath}
\usepackage{graphicx}
\usepackage{hyperref}
\usepackage[utf8]{inputenc}
\usepackage[acronym, toc]{glossaries}

\usepackage{setspace}	%double spacing for text, single for captions, footnotes, etc.
\usepackage{natbib}		% substituye a 'hypernat' que funciona en Windows.
\usepackage[english]{babel}
\usepackage[utf8]{inputenc}
\usepackage{color}
\usepackage{hhline} 		% extended styles for tables
\usepackage{multirow}
\usepackage{subfigure}
\usepackage{amsmath,amsmath,amssymb} 
\usepackage{fancyhdr}
\usepackage{epsfig, amsmath}
\usepackage{algorithm}
\usepackage{algorithmic}

\usepackage[section]{placeins}

% general settings
\hypersetup{
	linktocpage=true,
	colorlinks=true,
	linkcolor=blue,
	citecolor=blue,
}
\definecolor{Hgray}{gray}{0.6}

\newenvironment{definition}[1][Definition]{\begin{trivlist}
\item[\hskip \labelsep {\bfseries #1}]}{\end{trivlist}}

\setlength{\topmargin}{0cm}
\setlength{\textheight}{23cm}
\setlength{\textwidth}{17cm}
\setlength{\oddsidemargin}{0cm}
\setlength{\evensidemargin}{0cm}
\setlength{\headheight}{1cm}

% indica que las 'sub-sub-sections' sean numeradas y aparezcan en el indice
\setcounter{secnumdepth}{3}
\setcounter{tocdepth}{2}

% settings for code
\renewcommand{\algorithmicrequire}{\textbf{Entrada: }}
\renewcommand{\algorithmicensure}{\textbf{Salida: }}

\makeglossaries
\newglossaryentry{kpi}
{
    name=KPI,
    description={Key Performance Indicator}
}

\newglossaryentry{mnist}
{
    name=MNIST,
    description={The MNIST database of handwritten digits used for training models}
}

\newglossaryentry{kdd}
{
    name=KDD,
    description={KDD cup 1999 network intrusion dataset}
}

\newglossaryentry{ixi}
{
    name=IXI,
    description={IXI dataset is a collection of 600 Brain MR Images, in 3-D, collected from 3 different hospitals in London. The images are scanned in 1.5 and 3 Tesla which is the range of the quality that we could usually find in hospitals around the world}
}

\newacronym{pca}{PCA}{Principal Component Analysis}

\newacronym{mri}{MRI}{Magnetic Resonance Image}

\newacronym{vqvae}{VQ-VAE}{Vector-Quantized Variational Autoencoder}

\newacronym{cevae}{ceVAE}{Context-Encoder Variational Autoencoder}

\newacronym{strega}{StRegA}{Segmentation Regularised Anomaly}


\begin{document}

% Portada
\newpage
\thispagestyle{empty}

\baselineskip 2em

%\vspace*{1cm}

\centerline{\includegraphics[width=0.6\textwidth]{images/UOC-logo}}
\begin{center}
\textsc{Universitat Oberta de Catalunya (UOC) \\
 Máster Universitario en Ciencia de Datos (\textit{Data Science})\\}

%\centerline {\pic{UOC}{4cm}}

\vspace*{1.5cm}

\textsc{\Large TRABAJO FINAL DE MÁSTER}

\vspace*{0.5cm}

\textsc{\large Área: YYY}


%\textbf{\Huge VirtualTechLab Model: }

\vspace*{2.0cm}

\title{\Large Artificial MRI brain images creation with Variational Autoencoders}

\vspace{2.5cm}
\baselineskip 1em

\baselineskip 2em
-----------------------------------------------------------------------------\\
Autor:      Miguel Tablado\\
Tutor:      Baris Kanber\\
Profesor:   Nombre del profesor responsable del área de TF\\
-----------------------------------------------------------------------------\\
\vspace*{1.5cm}
Barcelona, \today

\author{Miguel Tablado}

\end{center}



%\tableofcontents

\pagenumbering{roman} 
\setcounter{page}{1} 
\pagestyle{plain}

%%%%%%%%%%%%%
%%% FICHA %%%
%%%%%%%%%%%%%
\chapter*{FICHA DEL TRABAJO FINAL}
\begin{table}[ht]
	\centering{}
	\renewcommand{\arraystretch}{2}
	\begin{tabular}{r | l}
		\hline
		Título del trabajo: & Artificial MRI brain images creation with Variational Autoencoders\\
		\hline
        Nombre del autor: & Miguel Tablado León\\
		\hline
        Nombre del Tutor/a de TF: & Baris Kanber\\
		\hline
        Nombre del/de la PRA: & Ferran Prados Carrasco\\
		\hline
        Fecha de entrega: & 02/2023\\
		\hline
        Titulación o programa: & Máster en Ciencia de Datos\\
		\hline
        Área del Trabajo Final: & Medicine Area (TFM-Med)\\
		\hline
        Language: & English\\
		\hline
        Keywords & Deep Learning, Brain MRI, Variational Autoencoder\\
		\hline
	\end{tabular}
\end{table}

\pagestyle{fancy}
\renewcommand{\chaptermark}[1]{ \markboth{#1}{}}
\renewcommand{\sectionmark}[1]{\markright{ \thesection.\ #1}}
\lhead[\fancyplain{}{\bfseries\thepage}]{\fancyplain{}{\bfseries\rightmark}}
\rhead[\fancyplain{}{\bfseries\leftmark}]{\fancyplain{}{\bfseries\thepage}}
\cfoot{}

% indice
\cleardoublepage
\phantomsection
\addcontentsline{toc}{chapter}{Index}
\tableofcontents
% listado de figuras
\cleardoublepage
\phantomsection
\addcontentsline{toc}{chapter}{Figure List}
\listoffigures
% listado de tablas
\cleardoublepage
\phantomsection
\addcontentsline{toc}{chapter}{Table List}
\listoftables

\thispagestyle{empty}

\pagenumbering{arabic}

\pagestyle{fancy}
\renewcommand{\chaptermark}[1]{ \markboth{#1}{}}
\renewcommand{\sectionmark}[1]{\markright{ \thesection.\ #1}}
\lhead[\fancyplain{}{\bfseries\thepage}]{\fancyplain{}{\bfseries\rightmark}}
\rhead[\fancyplain{}{\bfseries\leftmark}]{\fancyplain{}{\bfseries\thepage}}
\cfoot{}

\onehalfspacing

\input{introduction/0_abstract.tex}

\chapter{Introduction}

\section{Context and project justification}

Artificial Intelligence has arrived to change the world in almost (if not all) any field. Today, we are surrounded by (and we are using many) AI products like smartphones’ face recognition capabilities, home cleaning robots or cars with autopilot options.

From the different AI fields, computer vision is maybe the most popular one and the one which is usually used for explaining AI capabilities to general public. Identifying a cat in a picture could perfectly be the example used in every AI presentation to welcome people to AI.

Image processing is intuitively matched with medical diagnosis by anyone having or not any expertise on the field. Every single citizen will have heard of magnetic resonance imaging, and anyone easily transposes ‘cat detection’ to ‘anomaly detection’, being the anomaly a tumor or anything else.

There are different techniques to scan people and create images for clinical diagnosis like X-Ray or Magnetic Resonance Imaging. In this project, we will work with Magnetic Resonance Images (MRI) which are images created by a machine with a large bore that scans people lying inside it. The MR technique is non-invasive, it produces no radiation, and is used to scan almost any part of the body from which we will focus on brain images.

Combining AI diagnosis capabilities on image processing and MRI images, we can think of helping doctors to identify the presence of anomalies or looking for concrete diagnosis for a specific disease.

Obviously, these projects are not easy at all and they face a lot of challenges. One of the first challenges that such AI project faces is the difficulty to obtain a large set of brain images that are needed to train an accurate AI model. Scanning people is too costly and requires a lot of time. and the challenge of having a large dataset gets harder once we understand that brain images may differ depending on age or gender.

Today, there is a clear limitation on how to reproduce or obtain healthy brain images for AI-based diagnosis projects while the appearance of new AI techniques known as Autoencoders and Variational Autoencoders introduces a new area of investigation to mitigate the gap.

This project aims to be a proof of concept to create AI-created images with new variational autoencoders that would serve to augment any existing MRI dataset and that will help to improve the accuracy of brain anomaly detection projects, including those which could be used for overall anomaly detection to others more specific which would help on concrete disease diagnosis.

Personal motivation comes from various angles:

\begin{itemize}
    \item One is to prove myself that I can work with new AI architectures demonstrating that I have acquired the knowledge needed (deep enough) to be productive and to be able to innovate in the health sector.
    \item Deep Learning has been the subject which I enjoyed the most, hence continuing with Autoencoders seems natural to me as the next step on AI adoption
    \item At no doubts, if I can contribute to help on brain issue detection or diagnosis, I will feel my life been completely fulfilled
\end{itemize}
\section{Aim of the project}

The aim of this project is to serve as a Proof of Concept on how MRI brain images can be artificially generated with Variational Autoencoders which ultimately would serve to enhance existing or new datasets to improve model accuracy by having larger samples of data.

Foreseen projects objectives are:

\begin{itemize}
    \item Obtain basic knowledge about MRI images and the NIFTI file format
    \item Obtain and visualise 2-D images from 3-D images in the dataset
    \item Select what range of slices (2-D images) from brain to be created
    \item Test different existing networks and choose the one to be used
    \item Tune network parameters
    \item Compare generated brain images against real ones, qualitatively
\end{itemize}
\section{Project Plan}

\subsection{Resources}

\begin{enumerate}
    \item 1 Data Scientist: Miguel Tablado will be playing this role and will dedicate 300h
    \item 1 Coach/Tutor: Baris Kanber will be assisting Miguel Tablado during the project
    \item MRI images and demographic information from IXI Dataset
    \item GPU resources are needed to train and test the network
\end{enumerate}

\subsection{High Level Plan}

The plan will be executed in 3 different phases with the listed tasks:
\begin{enumerate}
    \item Phase 1: Analysis
    \begin{enumerate}
        \item Gain knowledge on MRI and NIFTI protocol
        \item Describe images and dataset
        \item Extract 2D images
        \item Pre-processing images
    \end{enumerate}
    \item Phase 2: MR Images creation
    \begin{enumerate}
        \item Test different Network architectures
        \item Tune-up architectural network
    \end{enumerate}
    \item Phase 3: Project documentation
    \begin{enumerate}
        \item Write conclusions
        \item Create project documentation
        \item Create project presentation
    \end{enumerate}
\end{enumerate}

\begin{figure}[ht]
    \hspace*{-1.2in}
    \centering
    \includegraphics[width = 20cm, height = 6cm]{images/project-plan.png}
    \caption[]{Project Plan. source: https://www.onlinegantt.com}
    \label{fig:project-plan}
\end{figure}

\newpage
\subsection{Tasks}
\subsubsection*{Phase 1: Analysis}

During this face the different tasks will be executed to prepare the work and includes:
\begin{enumerate}
    \item Gain knowledge on MRI and the NIFTI file format: This activity consists of reading papers and documents to gain sufficient knowledge to execute the project. There is no need to become an expert on the matter but understanding how those files are and how to process them.
    \item Describe images and dataset: During this activity, a description of the dataset will be generated with a view of the quality of the dataset for the aim of the project and any findings which could result.
    \item Extract 2D images: Load 3D images and extract 2D slices from the original dataset, which will be depicted with code.
    \item Pre-processing images: Decide which transformations on the 2D images would help the project like pixel changes or applying gray-scale transformations.
\end{enumerate}

\subsubsection*{Phase 2: MR Images creation}

\begin{enumerate}
    \item Test different Network architectures: This task will take few existing network architectures and be tested with the dataset so that one of them will be selected to be improved and used as the project architecture.
    \item Tune-up architectural network: Tune the selected architecture with useful techniques like changing network layers
\end{enumerate}

\documentclass[11pt,a4paper,openany]{book}

\usepackage{amsmath}
\usepackage{graphicx}
\usepackage{hyperref}
\usepackage[utf8]{inputenc}
\usepackage[acronym, toc]{glossaries}

\usepackage{setspace}	%double spacing for text, single for captions, footnotes, etc.
\usepackage{natbib}		% substituye a 'hypernat' que funciona en Windows.
\usepackage[english]{babel}
\usepackage[utf8]{inputenc}
\usepackage{color}
\usepackage{hhline} 		% extended styles for tables
\usepackage{multirow}
\usepackage{subfigure}
\usepackage{amsmath,amsmath,amssymb} 
\usepackage{fancyhdr}
\usepackage{epsfig, amsmath}
\usepackage{algorithm}
\usepackage{algorithmic}

\usepackage[section]{placeins}

% general settings
\hypersetup{
	linktocpage=true,
	colorlinks=true,
	linkcolor=blue,
	citecolor=blue,
}
\definecolor{Hgray}{gray}{0.6}

\newenvironment{definition}[1][Definition]{\begin{trivlist}
\item[\hskip \labelsep {\bfseries #1}]}{\end{trivlist}}

\setlength{\topmargin}{0cm}
\setlength{\textheight}{23cm}
\setlength{\textwidth}{17cm}
\setlength{\oddsidemargin}{0cm}
\setlength{\evensidemargin}{0cm}
\setlength{\headheight}{1cm}

% indica que las 'sub-sub-sections' sean numeradas y aparezcan en el indice
\setcounter{secnumdepth}{3}
\setcounter{tocdepth}{2}

% settings for code
\renewcommand{\algorithmicrequire}{\textbf{Entrada: }}
\renewcommand{\algorithmicensure}{\textbf{Salida: }}

\makeglossaries
\input{closing/glossary.tex}

\begin{document}

% Portada
\input{0_tittle.tex}


%\tableofcontents

\pagenumbering{roman} 
\setcounter{page}{1} 
\pagestyle{plain}

%%%%%%%%%%%%%
%%% FICHA %%%
%%%%%%%%%%%%%
\chapter*{FICHA DEL TRABAJO FINAL}
\begin{table}[ht]
	\centering{}
	\renewcommand{\arraystretch}{2}
	\begin{tabular}{r | l}
		\hline
		Título del trabajo: & Artificial MRI brain images creation with Variational Autoencoders\\
		\hline
        Nombre del autor: & Miguel Tablado León\\
		\hline
        Nombre del Tutor/a de TF: & Baris Kanber\\
		\hline
        Nombre del/de la PRA: & Ferran Prados Carrasco\\
		\hline
        Fecha de entrega: & 02/2023\\
		\hline
        Titulación o programa: & Máster en Ciencia de Datos\\
		\hline
        Área del Trabajo Final: & Medicine Area (TFM-Med)\\
		\hline
        Language: & English\\
		\hline
        Keywords & Deep Learning, Brain MRI, Variational Autoencoder\\
		\hline
	\end{tabular}
\end{table}

\pagestyle{fancy}
\renewcommand{\chaptermark}[1]{ \markboth{#1}{}}
\renewcommand{\sectionmark}[1]{\markright{ \thesection.\ #1}}
\lhead[\fancyplain{}{\bfseries\thepage}]{\fancyplain{}{\bfseries\rightmark}}
\rhead[\fancyplain{}{\bfseries\leftmark}]{\fancyplain{}{\bfseries\thepage}}
\cfoot{}

% indice
\cleardoublepage
\phantomsection
\addcontentsline{toc}{chapter}{Index}
\tableofcontents
% listado de figuras
\cleardoublepage
\phantomsection
\addcontentsline{toc}{chapter}{Figure List}
\listoffigures
% listado de tablas
\cleardoublepage
\phantomsection
\addcontentsline{toc}{chapter}{Table List}
\listoftables

\thispagestyle{empty}

\pagenumbering{arabic}

\pagestyle{fancy}
\renewcommand{\chaptermark}[1]{ \markboth{#1}{}}
\renewcommand{\sectionmark}[1]{\markright{ \thesection.\ #1}}
\lhead[\fancyplain{}{\bfseries\thepage}]{\fancyplain{}{\bfseries\rightmark}}
\rhead[\fancyplain{}{\bfseries\leftmark}]{\fancyplain{}{\bfseries\thepage}}
\cfoot{}

\onehalfspacing

\input{introduction/0_abstract.tex}

\chapter{Introduction}

\input{introduction/1_context.tex}
\input{introduction/2_aim.tex}
\input{introduction/5_plan.tex}

\input{state_of_art/main.tex}
\input{impl/main.tex}
\input{conclusions/main.tex}

\clearpage
\printglossary[type=\acronymtype]
\printglossary

\bibliographystyle{plain} % We choose the "plain" reference style
\bibliography{refs}

\end{document}

\documentclass[11pt,a4paper,openany]{book}

\usepackage{amsmath}
\usepackage{graphicx}
\usepackage{hyperref}
\usepackage[utf8]{inputenc}
\usepackage[acronym, toc]{glossaries}

\usepackage{setspace}	%double spacing for text, single for captions, footnotes, etc.
\usepackage{natbib}		% substituye a 'hypernat' que funciona en Windows.
\usepackage[english]{babel}
\usepackage[utf8]{inputenc}
\usepackage{color}
\usepackage{hhline} 		% extended styles for tables
\usepackage{multirow}
\usepackage{subfigure}
\usepackage{amsmath,amsmath,amssymb} 
\usepackage{fancyhdr}
\usepackage{epsfig, amsmath}
\usepackage{algorithm}
\usepackage{algorithmic}

\usepackage[section]{placeins}

% general settings
\hypersetup{
	linktocpage=true,
	colorlinks=true,
	linkcolor=blue,
	citecolor=blue,
}
\definecolor{Hgray}{gray}{0.6}

\newenvironment{definition}[1][Definition]{\begin{trivlist}
\item[\hskip \labelsep {\bfseries #1}]}{\end{trivlist}}

\setlength{\topmargin}{0cm}
\setlength{\textheight}{23cm}
\setlength{\textwidth}{17cm}
\setlength{\oddsidemargin}{0cm}
\setlength{\evensidemargin}{0cm}
\setlength{\headheight}{1cm}

% indica que las 'sub-sub-sections' sean numeradas y aparezcan en el indice
\setcounter{secnumdepth}{3}
\setcounter{tocdepth}{2}

% settings for code
\renewcommand{\algorithmicrequire}{\textbf{Entrada: }}
\renewcommand{\algorithmicensure}{\textbf{Salida: }}

\makeglossaries
\input{closing/glossary.tex}

\begin{document}

% Portada
\input{0_tittle.tex}


%\tableofcontents

\pagenumbering{roman} 
\setcounter{page}{1} 
\pagestyle{plain}

%%%%%%%%%%%%%
%%% FICHA %%%
%%%%%%%%%%%%%
\chapter*{FICHA DEL TRABAJO FINAL}
\begin{table}[ht]
	\centering{}
	\renewcommand{\arraystretch}{2}
	\begin{tabular}{r | l}
		\hline
		Título del trabajo: & Artificial MRI brain images creation with Variational Autoencoders\\
		\hline
        Nombre del autor: & Miguel Tablado León\\
		\hline
        Nombre del Tutor/a de TF: & Baris Kanber\\
		\hline
        Nombre del/de la PRA: & Ferran Prados Carrasco\\
		\hline
        Fecha de entrega: & 02/2023\\
		\hline
        Titulación o programa: & Máster en Ciencia de Datos\\
		\hline
        Área del Trabajo Final: & Medicine Area (TFM-Med)\\
		\hline
        Language: & English\\
		\hline
        Keywords & Deep Learning, Brain MRI, Variational Autoencoder\\
		\hline
	\end{tabular}
\end{table}

\pagestyle{fancy}
\renewcommand{\chaptermark}[1]{ \markboth{#1}{}}
\renewcommand{\sectionmark}[1]{\markright{ \thesection.\ #1}}
\lhead[\fancyplain{}{\bfseries\thepage}]{\fancyplain{}{\bfseries\rightmark}}
\rhead[\fancyplain{}{\bfseries\leftmark}]{\fancyplain{}{\bfseries\thepage}}
\cfoot{}

% indice
\cleardoublepage
\phantomsection
\addcontentsline{toc}{chapter}{Index}
\tableofcontents
% listado de figuras
\cleardoublepage
\phantomsection
\addcontentsline{toc}{chapter}{Figure List}
\listoffigures
% listado de tablas
\cleardoublepage
\phantomsection
\addcontentsline{toc}{chapter}{Table List}
\listoftables

\thispagestyle{empty}

\pagenumbering{arabic}

\pagestyle{fancy}
\renewcommand{\chaptermark}[1]{ \markboth{#1}{}}
\renewcommand{\sectionmark}[1]{\markright{ \thesection.\ #1}}
\lhead[\fancyplain{}{\bfseries\thepage}]{\fancyplain{}{\bfseries\rightmark}}
\rhead[\fancyplain{}{\bfseries\leftmark}]{\fancyplain{}{\bfseries\thepage}}
\cfoot{}

\onehalfspacing

\input{introduction/0_abstract.tex}

\chapter{Introduction}

\input{introduction/1_context.tex}
\input{introduction/2_aim.tex}
\input{introduction/5_plan.tex}

\input{state_of_art/main.tex}
\input{impl/main.tex}
\input{conclusions/main.tex}

\clearpage
\printglossary[type=\acronymtype]
\printglossary

\bibliographystyle{plain} % We choose the "plain" reference style
\bibliography{refs}

\end{document}

\documentclass[11pt,a4paper,openany]{book}

\usepackage{amsmath}
\usepackage{graphicx}
\usepackage{hyperref}
\usepackage[utf8]{inputenc}
\usepackage[acronym, toc]{glossaries}

\usepackage{setspace}	%double spacing for text, single for captions, footnotes, etc.
\usepackage{natbib}		% substituye a 'hypernat' que funciona en Windows.
\usepackage[english]{babel}
\usepackage[utf8]{inputenc}
\usepackage{color}
\usepackage{hhline} 		% extended styles for tables
\usepackage{multirow}
\usepackage{subfigure}
\usepackage{amsmath,amsmath,amssymb} 
\usepackage{fancyhdr}
\usepackage{epsfig, amsmath}
\usepackage{algorithm}
\usepackage{algorithmic}

\usepackage[section]{placeins}

% general settings
\hypersetup{
	linktocpage=true,
	colorlinks=true,
	linkcolor=blue,
	citecolor=blue,
}
\definecolor{Hgray}{gray}{0.6}

\newenvironment{definition}[1][Definition]{\begin{trivlist}
\item[\hskip \labelsep {\bfseries #1}]}{\end{trivlist}}

\setlength{\topmargin}{0cm}
\setlength{\textheight}{23cm}
\setlength{\textwidth}{17cm}
\setlength{\oddsidemargin}{0cm}
\setlength{\evensidemargin}{0cm}
\setlength{\headheight}{1cm}

% indica que las 'sub-sub-sections' sean numeradas y aparezcan en el indice
\setcounter{secnumdepth}{3}
\setcounter{tocdepth}{2}

% settings for code
\renewcommand{\algorithmicrequire}{\textbf{Entrada: }}
\renewcommand{\algorithmicensure}{\textbf{Salida: }}

\makeglossaries
\input{closing/glossary.tex}

\begin{document}

% Portada
\input{0_tittle.tex}


%\tableofcontents

\pagenumbering{roman} 
\setcounter{page}{1} 
\pagestyle{plain}

%%%%%%%%%%%%%
%%% FICHA %%%
%%%%%%%%%%%%%
\chapter*{FICHA DEL TRABAJO FINAL}
\begin{table}[ht]
	\centering{}
	\renewcommand{\arraystretch}{2}
	\begin{tabular}{r | l}
		\hline
		Título del trabajo: & Artificial MRI brain images creation with Variational Autoencoders\\
		\hline
        Nombre del autor: & Miguel Tablado León\\
		\hline
        Nombre del Tutor/a de TF: & Baris Kanber\\
		\hline
        Nombre del/de la PRA: & Ferran Prados Carrasco\\
		\hline
        Fecha de entrega: & 02/2023\\
		\hline
        Titulación o programa: & Máster en Ciencia de Datos\\
		\hline
        Área del Trabajo Final: & Medicine Area (TFM-Med)\\
		\hline
        Language: & English\\
		\hline
        Keywords & Deep Learning, Brain MRI, Variational Autoencoder\\
		\hline
	\end{tabular}
\end{table}

\pagestyle{fancy}
\renewcommand{\chaptermark}[1]{ \markboth{#1}{}}
\renewcommand{\sectionmark}[1]{\markright{ \thesection.\ #1}}
\lhead[\fancyplain{}{\bfseries\thepage}]{\fancyplain{}{\bfseries\rightmark}}
\rhead[\fancyplain{}{\bfseries\leftmark}]{\fancyplain{}{\bfseries\thepage}}
\cfoot{}

% indice
\cleardoublepage
\phantomsection
\addcontentsline{toc}{chapter}{Index}
\tableofcontents
% listado de figuras
\cleardoublepage
\phantomsection
\addcontentsline{toc}{chapter}{Figure List}
\listoffigures
% listado de tablas
\cleardoublepage
\phantomsection
\addcontentsline{toc}{chapter}{Table List}
\listoftables

\thispagestyle{empty}

\pagenumbering{arabic}

\pagestyle{fancy}
\renewcommand{\chaptermark}[1]{ \markboth{#1}{}}
\renewcommand{\sectionmark}[1]{\markright{ \thesection.\ #1}}
\lhead[\fancyplain{}{\bfseries\thepage}]{\fancyplain{}{\bfseries\rightmark}}
\rhead[\fancyplain{}{\bfseries\leftmark}]{\fancyplain{}{\bfseries\thepage}}
\cfoot{}

\onehalfspacing

\input{introduction/0_abstract.tex}

\chapter{Introduction}

\input{introduction/1_context.tex}
\input{introduction/2_aim.tex}
\input{introduction/5_plan.tex}

\input{state_of_art/main.tex}
\input{impl/main.tex}
\input{conclusions/main.tex}

\clearpage
\printglossary[type=\acronymtype]
\printglossary

\bibliographystyle{plain} % We choose the "plain" reference style
\bibliography{refs}

\end{document}


\clearpage
\printglossary[type=\acronymtype]
\printglossary

\bibliographystyle{plain} % We choose the "plain" reference style
\bibliography{refs}

\end{document}


\clearpage
\printglossary[type=\acronymtype]
\printglossary

\bibliographystyle{plain} % We choose the "plain" reference style
\bibliography{refs}

\end{document}


\clearpage
\printglossary[type=\acronymtype]
\printglossary

\bibliographystyle{plain} % We choose the "plain" reference style
\bibliography{refs}

\end{document}
